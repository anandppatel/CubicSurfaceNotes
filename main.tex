\documentclass[12pt,reqno]{amsart}
\usepackage[margin = 1.3 in]{geometry}
\usepackage[frenchmath,defaultmathsizes]{mathastext}
\usepackage{
  hyperref,
  amsmath,
  amssymb,
  tikz,
  amsthm,
  thmtools,
  microtype,
  enumitem,
  stmaryrd,
  tikz-cd,
  mathrsfs,
}
\usepackage{graphicx}

\usepackage{import}
\usepackage{xifthen}
\usepackage{pdfpages}
\usepackage{transparent}

\newcommand{\incfig}[1]{%
    \def\svgwidth{\columnwidth}
    \import{./figures/}{#1.pdf_tex}
}

\usepackage{showlabels}

\linespread{1.1}
\usepackage[all]{xy}
\usepackage{eucal}


\title{Counting Cubic Surfaces}
\author{Anand Deopurkar, Anand Patel
  \& Dennis Tseng}





%-----------------------------------------------
\def\labelitemi{--}
\newcommand{\todo}[1]{\fbox{ToDo: #1}}
\renewcommand{\k}{k}
\DeclareMathOperator{\id}{id}
\DeclareMathOperator{\Bl}{Bl}
\DeclareMathOperator{\Orb}{\overline{Orb}}
\DeclareMathOperator{\Polar}{Polar}
\DeclareMathOperator{\res}{Res}
\DeclareMathOperator{\Pole}{Pole}
\DeclareMathOperator{\Hilb}{Hilb}
\DeclareMathOperator{\Sing}{Sing}
\DeclareMathOperator{\M}{\mathcal{M}}
\renewcommand{\to}{{\longrightarrow}}

\renewcommand{\sectionautorefname}{\S}
\renewcommand{\subsectionautorefname}{\S}
% Let us keep this minimial
% Let us also define things only if they are previously undefined.

% Common theorem-like environments
\ifcsname theorem\endcsname{}\else\declaretheorem[parent=subsection]{theorem}\fi
\ifcsname corollary\endcsname{}\else\declaretheorem[sibling=theorem]{corollary}\fi
\ifcsname lemma\endcsname{}\else\declaretheorem[sibling=theorem]{lemma}\fi
\ifcsname proposition\endcsname{}\else\declaretheorem[sibling=theorem]{proposition}\fi
\ifcsname conjecture\endcsname{}\else\declaretheorem[sibling=theorem]{conjecture}\fi
\ifcsname problem\endcsname{}\else\declaretheorem[sibling=theorem]{problem}\fi
\ifcsname question\endcsname{}\else\declaretheorem[sibling=theorem]{question}\fi
\ifcsname definition\endcsname{}\else\declaretheorem[sibling=theorem, style=definition]{definition}\fi
\ifcsname exercise\endcsname{}\else\declaretheorem[sibling=theorem, style=definition]{exercise}\fi
\ifcsname example\endcsname{}\else\declaretheorem[sibling=theorem, style=definition]{example}\fi
\ifcsname remark\endcsname{}\declaretheorem[sibling=theorem, style=remark]{remark}\fi

% Common abbreviations

% Absolutely standard rings and fields
\providecommand {\N}{{\bf N}}
\providecommand {\Z}{{\bf Z}}
\providecommand {\Q}{{\bf Q}}
\providecommand {\R}{{\bf R}}
\providecommand {\C}{{\bf C}}

% Common spaces grassmannian
\renewcommand {\P}{{\bf P}}
\providecommand {\Gr}{{\bf Gr}}
\providecommand {\A}{{\bf A}}

% Groups
\providecommand{\SL}{\operatorname{SL}}
\providecommand{\GL}{\operatorname{GL}}
\providecommand{\PGL}{\operatorname{PGL}}
\providecommand{\Gm}{{\bf G}_m}

% f \from G \to H reads much better than f \colon G \to H
\providecommand {\from}{{\colon}}

% Absolutely standard notation
\providecommand{\spec}{\operatorname{Spec}}
\providecommand{\proj}{\operatorname{Proj}}
\providecommand{\coker}{\operatorname{coker}}
% Kernel is already defined
\providecommand{\Blowup}{\operatorname{Bl}}
\providecommand{\Hom}{\operatorname{Hom}}
\providecommand{\Ext}{\operatorname{Ext}}
\providecommand{\Tor}{\operatorname{Tor}}
\providecommand{\End}{\operatorname{End}}
\providecommand{\Aut}{\operatorname{Aut}}
\providecommand{\codim}{\operatorname{codim}}
% Dim is already defined
\providecommand{\Pic}{\operatorname{Pic}}
\providecommand{\Sym}{\operatorname{Sym}}
\providecommand{\rk}{\operatorname{rk}}
\declaretheorem[sibling=theorem,style=remark]{remark}
\numberwithin{equation}{section}
\declaretheorem[title=Theorem]{maintheorem}
\renewcommand{\themaintheorem}{\Alph{maintheorem}}


\renewcommand{\O}{\mathcal O}
\newcommand{\G}{\mathbf G}
\newcommand{\F}{\mathbf F}
\newcommand{\td}{\widetilde}
\newcommand{\V}{\mathcal V}
\newcommand{\cP}{\mathcal P}
\newcommand{\cC}{\mathcal{C}}
\newcommand{\cX}{\mathcal{X}}
\newcommand{\hL}{\widehat{\mathcal{L}}}
\newcommand{\hpsi}{\widehat{\psi}}
\newcommand{\fm}{\mathfrak m}
\newcommand{\smvee}{\raise0.5ex\hbox{$\scriptscriptstyle\vee$}}
\newcommand{\hM}{\widehat{\M}}
\newcommand{\compl}[1]{\widehat{#1}}
\newcommand{\Spec}{{\text{\rm Spec}\,}}
\newcommand{\Spf}{{\text{\rm Spf}\,}}
\renewcommand {\o}[1]{\overline{#1}}
\newcommand{\Proj}{{\text{\rm Proj}\,}}
\newcommand{\git}{\sslash}
% -----------------------------------------------

%%% BEGIN DOCUMENT

\begin{document}




\maketitle

\section{Introduction}
\label{sec:intro}

The beautiful enumerative geometry inherent to cubic surfaces has
captured the imagination ever since, in 1849, Cayley and Salmon
illuminated the structure of the $27$ lines.  We add to this long
history by finding an enumerative formula which counts the number of
times the {\sl moduli} of a general cubic surface arises in suitably
generic, complete, $4$ parameter families. Applying the formula to a
few simple examples yields:
\begin{enumerate}
\item[(A)] \label{fact:RanestadSturmfelds} {\sl A general cubic
    surface's moduli arises $96120$ times in a general $4$ dimensional
    linear system of cubic surfaces,} and
\item[(B)] {\sl A general cubic surface's moduli arises $42120$ times
    as a hyperplane section of a general cubic threefold.}
\end{enumerate}

Our formula, along with its precise hypotheses, is stated in the
following theorem.

\begin{theorem}\label{theorem:main}
  Let $\pi: X \to B$ be a good family of cubic surfaces over a proper
  base $B$, with semi-stable generic fiber. If
  $v_{i} \in A^{i}(B), i=1, 2,3, 4$, are the Chern classes of its
  natural rank $4$ vector bundle $\V := \pi_{*}(\omega_{\pi}^{-1})$,
  then
  \begin{align}
    \label{eq:MAIN}
   \int_{B} 1080v_{1}^{2}v_{2} - 1080v_{1}v_{3}+9720v_{4} = \deg \mu,
  \end{align}
  where $\mu: B \dashrightarrow \M$ is the induced map to the moduli
  space of cubic surfaces.
\end{theorem}


\subsection{Strategy}
\label{sec:strategy}

The overall strategy of the proof is straightforward. We argue that a
priori there must exist a formula of the form

\begin{align}
  \label{eq:P}
  \int_{B} a_{1^4}v_{1}^{4} + a_{1^{2}2}v_{1}^{2}v_{2} + a_{13}v_{1}v_{3} + a_{2^2}v_{2}^{2} + a_{4}v_{4} = \deg \mu,
\end{align}
as found in \autoref{theorem:main} -- this is merely an instance of a
foundational theorem in equivariant enumerative geometry \cite[].
Then, we determine the five coefficients $a_{\bullet}$ by exhibiting
several good families for which we know all $v_{i}$ {\sl and}
$\deg \mu$.  The construction of these good families requires some
effort and occupies most of the paper.  The bases $B$ of our families
are : $\P^4$, $\overline{\M}_{0,7}$ and a Hassett-weighted variant of
it, a blow up of the Hilbert scheme of two points on a quintic del
Pezzo surface, and amusingly, the classifying stack $B\G_m$.


In this way, we obtain a system of linear equations in the five
unknowns $a_{\bullet}$, which turns out to have a unique solution.  In
fact, we exhibit {\sl nine} total families/equations; the redundancy
provides extra checks on the end result.
  


%---------------------------------------------------------------------------


\section{Preliminaries}
\label{sec:good}

After establishing notation, conventions, and classical facts about
cubic surfaces, this section introduces the notion of a {\sl good}
family of cubic surfaces and then provides the main tool
(\autoref{prop:good}) for verifying the goodness of several families
occurring in the paper.


\subsection{Notation and conventions}
\label{sec:notation-conventions}


\subsection{Classical facts about cubic surfaces}
\label{sec:classical-facts}

\begin{theorem}[Cayley, Salmon]
  \label{theorem:cayleysalmon} Every smooth cubic surface contains
  exactly $27$ lines.
\end{theorem}

An ordered tuple of six lines $(\ell_1, \dots, \ell_6)$ on a cubic
surface is called a {\sl six} if they are pairwise disjoint.  An
unordered pair of sixes
$\{(\ell_1, \dots, \ell_6), (\ell_1', \dots, \ell_6')\}$ is called a
{\sl double-six} if
\begin{itemize}
\item $\ell_{i}$ and $\ell'_{i}$ are disjoint for all $i$, and
\item $\ell_{i}$ and $\ell'_{j}$ meet at a point whenever $i \neq j$.
  \end{itemize}

\begin{theorem}[Schl\"{a}fli]
  \label{theorem:schlafli} Each six on a smooth cubic surface has a
  unique extension to a double-six, and every smooth cubic surface
  contains $36$ double-sixes.
\end{theorem}

\subsubsection{Moduli spaces of cubic surfaces}
\label{sec:modulispacescubic}

\begin{align}
  \label{eq:modulispaces}
  \sigma: \M^{\dagger} \to \M\\
  (S, \underline{\ell}) \mapsto S. \nonumber
\end{align}

\autoref{theorem:schlafli} implies that
\begin{align}
  \label{eq:degreesigma}
  \deg (\sigma) = 72 \times 6! = 51840.
\end{align}


\begin{theorem}[Schubert]
  \label{theorem:Schubert} There are $20$ cuspidal cubic curves
  containing all of five general points in $\P^{2}$, and flexed at a
  sixth general point.
\end{theorem}


\subsection{Good families}
\label{sec:good-families}

\begin{definition}
\label{def:goodfamily}
Let $\pi :X \to B$ be a flat, projective morphism of relative
dimension $2$, with $B$ a $4$-dimensional scheme. We say $\pi$ is {\sl
  a good family of cubic surfaces} if:
\begin{itemize}
\item There exists a $\P^3$-bundle $\rho: P \to B$ and a closed
  embedding $\iota: X \hookrightarrow P$ embedding each fiber $X_b$,
  $b \in B$, as a cubic surface in $P_b$ and $\pi = \rho \circ \iota$,
\item There does {\sl not} exist a $b \in B$ such that the cubic
  surface $X_b$ is represented by a point in
  $\partial \Orb(F) \subset \P^{19}$.
\end{itemize}
\end{definition}

The base $B$ of a good family carries the {\sl natural} rank $4$
vector bundle
\begin{align}
  \label{eq:V}
  \V := \pi_{*}\omega_{X/B}^{-1}.
\end{align}
Furthermore, assuming the general fiber of $\pi$ is smooth, $B$ admits
a rational map
\begin{align}
  \label{eq:mu}
  \mu: B \dashrightarrow \M
\end{align}
to the coarse moduli variety of cubic surfaces
$\mathcal{M} := \Sym^{3}E^{\smvee} \sslash GL(E)$.




\begin{definition}
  \label{def:admissible} A length $6$ subscheme $Z \subset \P^{2}$
  with ideal sheaf $\mathcal{I}_{Z}$ is {\sl admissible} if:
  \begin{enumerate}
  \item $Z$ is curvilinear, i.e $Z$ is contained in a smooth curve,
  \item $h^{0}(\mathcal{I}_Z(3)) = 4$, and
  \item $Z$ is the scheme cut out by the vector space of cubics
    $H^0(\P^2, \mathcal{I}_Z(3))$.
  \end{enumerate}
\end{definition}

\begin{definition}
  \label{def:cubicZ} Let $Z \subset \P^2$ be an admissible length $6$
  subscheme.  The {\sl cubic surface associated to $Z$}, denoted
  $X_Z \subset \P^3$ is the image of the anti-canonical map
  $\kappa: \Bl_{Z}\P^2 \to \P^3$ induced by the linear system
  $H^0(\P^2,\mathcal{I}_Z(3))$.
\end{definition}

\begin{proposition}
  \label{prop:good} Let $Z \subset \P^2$ be an admissible, length $6$
  subscheme with $X_Z \subset \P^3$ its associated cubic surface. Then
  $\dim \Aut(\P^2,Z) > 0$ if and only if $\dim \Aut(\P^3,X_Z) > 0$.
\end{proposition}

\begin{proof} The implication
  $\dim \Aut(\P^2,Z) > 0 \implies \dim \Aut(\P^3,X_Z) > 0$ is simpler:
  Any automorphism $g \in \Aut(\P^2,Z)$ naturally lifts to an
  automorphism $g' \in \Aut \Bl_{Z}\P^2$. This latter automorphism
  must preserve the anti-canonical line bundle
  $\omega_{\Bl_{Z}(\P^2)}^{-1}$, and therefore induces an element in
  $\Aut(\P^3, X_Z)$. This element is nontrivial if $g$ is, because
  $\kappa: \Bl_{Z}(\P^2) \to X_Z$ is birational. The implication
  follows.


  The opposite implication
  $\dim \Aut(\P^3,X_Z) > 0 \implies \dim \Aut(\P^2,Z) > 0$ is more
  intricate. First, note that $\Bl_{Z}\P^2$ is a surface with finitely
  many $A_n$ singularities, thanks to the curvilinearity of $Z$; let
  $\tau: Y \to \Bl_{Z}\P^2$ denote its minimal desingularization.
  Since $A_n$ singularities are canonical, we conclude that
  $\tau^{*}(\omega_{\Bl_{Z}\P^2}) = \omega_{Y}$.  The composite
  $\kappa':= \kappa \circ \tau: Y \to X_{Z}$ can only contract a curve
  $C \subset Y$ to a singular point of $X_Z$.  This follows because   
  $\kappa'^{*}\omega_{X_{Z}} = \omega_{Y}$. Thus,
  $\kappa': \kappa'^{-1}(X_{Z}^{sm}) \to X_{Z}^{sm}$ is an
  isomorphism. Our claim is that $\kappa':Y \to X_{Z}$ is also the
  minimal desingularization of $X_Z$.

  Since $Y$ is {\sl some} desingularization of $X_Z$,
  $\kappa':Y \to X_{Z}$ factors through the minimal desingularization
  $\widetilde{X_{Z}} \to X_{Z}$ of $X_{Z}$; let
  $\alpha: Y \to \widetilde{X_{Z}}$ denote the induced map. $\alpha$
  is a birational morphism between smooth surfaces. Therefore, if it
  is not biregular, there must be a smooth $(-1)$ rational curve
  $E \subset Y$ contracted by $\alpha$.  Since $\omega_{Y}$ is pulled
  back from $\widetilde{X_{Z}}$, it follows that $\omega_{Y}|_{E}$ is
  trivial. This, along with $E \cdot E = -1$, contradicts the
  adjunction formula.  Hence $\alpha$ is an isomorphism.

  Every automorphism of a surface lifts uniquely to its minimal
  desingularization. Therefore, if $\dim \Aut(\P^3,X_Z) >0 $, it
  follows that $\dim \Aut(Y) >0$. Under this assumption, let
  $G \subset \Aut(Y)$ denote the connected component containing
  identity.

  The Picard group $\Pic(Y)$ is discrete, so the connected group $G$
  acts trivially on $\Pic(Y)$.  In particular, the line bundle $L$ on
  $\Bl_{Z}\P^2$ inducing the morphism $Y \to \Bl_{Z}\P^2 \to \P^2$ is
  preserved by $G$. Therefore, $G$ induces an action on
  $\P H^0(Y,L)= \P^2$.  This action renders the blow-down map
  $\Bl_{Z}\P^2 \to \P^2$ $G$-equivariant, from which it follows that
  the action of $G$ on $\P^2$ preserves the subscheme $Z$, which is
  what we sought to show.

  
\end{proof}

%-------------------------------------------------------------------------
\section{Test families}\label{sec:testfamilies}

\subsection{Family $B_1$}
\label{sec:first-test-family}

In this section, we construct a good family over $B_{1} = \P^{4}$.
Let $S$ denote the blow up of $\P^{2}$ at two points $y_1,y_2$. We let
$E_{1}, E_{2}$ denote the corresponding exceptional lines on $S$ and
we let $L \subset S$ denote the proper transform of a line in $\P^2$
passing through the two points $y_1,y_2$. Finally, we pick a general
conic through $y_{2}$ and let $C \subset S$ denote its proper
transform. Let $\omega_{S}^{-1}$ denote the anti-canonical line bundle
on $S$.  

\subsubsection{Construction of $B_1$}
\label{sec:construction-b_1}


The base of our first family is
\begin{align}
  \label{eq:B1}
  B_1 := \Hilb^{4}C \simeq \P^{4}.
\end{align}
$B_1$ carries a good family of cubic surfaces over it.  In short, to
each degree $4$ divisor $Z \subset C$, we assign the cubic surface
$X_{Z}$ which is the image of $\Bl_{Z}S$ under its anti-canonical
map. The following proposition justifies this construction:

\begin{proposition}
  \label{prop:B1} Let $Z \subset S$ be any length $4$ subscheme of $C$
  with ideal sheaf $\mathcal{I}_{Z} \subset \mathcal{O}_{S}$. Then
  \begin{enumerate}
  \item $h^{0}(S, \mathcal{I}_{Z} \otimes \omega_{S}^{-1}) = 4$, and
  \item $Z \subset S$ is the subscheme defined by the vector space of
    sections $H^{0}(S, \mathcal{I}_{Z} \otimes \omega_{S}^{-1})$.
  \end{enumerate}
\end{proposition}

\begin{proof}
  We translate (1) and (2) into geometric statements about the
  anti-canonical embedding $S \subset \P^{7}$.  The line bundle
  $\omega_{S}^{-1}(-C)$ on $S$ is basepoint free, as its divisor class
  is represented by proper transforms of lines through
  $y_{1} \in \P^2$, which become separated in the blow up.  The
  geometric consequence of this is: {\sl If
    $\langle C \rangle \simeq \P^5 \subset \P^7$ is the linear span of
    $C$, then $\langle C \rangle \cap S = C$, scheme theoretically.}

  Now $C$ is a rational normal quintic curve in its span
  $\langle C \rangle$. The following is a basic fact about rational
  normal curves: {\sl Every length $m \leq n$ subscheme of rational
    normal curve of degree $n$ spans a linear space of dimension $m-1$
    which, in turn, intersects the rational normal curve precisely at
    the length $m$ scheme.}  Both (1) and (2) follow directly from
  this.
\end{proof}

Let $\mathcal{Z} \subset B_1 \times C$ denote the universal
length $4$ subscheme, and consider $\mathcal{Z}$ as a closed subscheme
of the product $B_1 \times S$. Let $\pi_{1}, \pi_{2}$ denote the
projections of $B_1 \times S$ to its respective factors.
Then we set
\begin{align}
  \label{eq:X1}
  X' := \Bl_{\mathcal{Z}} \left( B_1 \times S \right).
\end{align}
Abusing notation, we continue to let $\pi_{1}$ and $\pi_{2}$ denote
the natural maps from $X'$ to $B_1$ and $S$, respectively.  We
let $E \subset X'$ denote the exceptional divisor over $\mathcal{Z}$.


Consider the line bundle
$\mathcal{L} := \pi_{2}^{*}\omega_{S}^{-1} (-E) =
\omega_{X'/B_{1}}^{-1}$ on $X'$.  By \autoref{prop:B1},
\begin{align}
  \label{eq:V1}
  \V := \pi_{1 *}\mathcal{L}
\end{align}
is a rank $4$ vector bundle on $B_1$, and the canonical evaluation map
\begin{align}
  \label{eq:eval1}
  \pi_{1}^{*}\V \to \mathcal{L}
\end{align}
is surjective.  Therefore, we obtain a map
\begin{align}
  \label{eq:X11}
  \kappa: X' \to \P \V^{\smvee}
\end{align}
whose image $X := \kappa(X')$ is a family of cubic surfaces
parametrized by $B_1$.  Let $\pi: X \to B_1$ denote the natural map.
By tracing through the construction, one can see that the vector
bundle $\V$ is equal to $\pi_{*} \omega_{X/B_1}'$, i.e. $\V$ is the
natural bundle for $\pi: X \to B_1$.

\subsubsection{Chern classes of $\V$}
\label{sec:chern-classes-b_1}

In order to calculate the Chern classes of $\V$, we express it as an
extension of a trivial rank $2$ vector bundle by the Lazarsfeld-Mukai
bundle on $B_1$ associated to the line bundle
$\omega_{S}^{-1}|_{C}$. By identifying the conic $C$ with $\P^{1}$,
this latter line bundle is $\O_{\P^{1}}(5)$ -- we will switch back and
forth between $C$ and $\P^{1}$ in what follows.  We also identify
$B_{1}$ with $\P^{4}$.


The trivial sub-bundle of $\V$ arises as follows: The fiber $\V_{[Z]}$
over a point $[Z] \in B_1$, corresponding to a subscheme
$Z \subset C$, is the vector space
$H^{0}(S, \mathcal{I}_{Z} \otimes \omega_{S}^{-1})$.  The latter
vector space contains the fixed $2$ dimensional vector space
$H^{0}(S, \omega_{S}^{-1}(-C))$.  This yields a $2$-dimensional
trivial subbundle of $\V$, whose quotient we denote by $\mathcal{K}$.


The bundle $\mathcal{K}$ is naturally identified with the kernel in
the exact sequence
\begin{align}
  \label{eq:LM}
  0 \to \mathcal{K} \to H^{0}(\P^1, \O_{\P^{1}}(5)) \otimes \O_{\P^4} \to \O_{\P^{1}}(5)^{[4]} \to 0 
\end{align}

Here, the rank four bundle $\O_{\P^{1}}(5)^{[4]}$ is the tautological
bundle whose fiber over a point $[Z] \in B_{1}$, corresponding to a
length $4$ subscheme $Z \subset \P^{1}$, is the vector space
$H^{0}(Z, \O_{\P^{1}}(5)|_{Z})$.  The map on the right side of
\eqref{eq:LM} is simply restriction of sections.

It is an easy exercise to show that
$\mathcal{K} \simeq \O_{\P^{4}}(-1) \oplus \O_{\P^{4}}(-1)$.  The
following lemma summarizes the relevant Chern class computations of
$\V$ -- for brevity's sake, we equate a $0$-cycle with its degree.

\begin{lemma}
  \label{lemma:chernB1}
  For the natural bundle $\V$ over $B_1= \P^{4}$, we have:
  \begin{align}\nonumber
    v_{1}^{4} = 16,\\\nonumber
    v_{1}^{2}v_{2} = 4,\\\nonumber
    v_{1}v_{3} =0,\\\nonumber
    v_2^{2} = 1,\\\nonumber
    v_{4} = 0.
  \end{align}
\end{lemma}

\begin{proof}
  By the Whitney sum formula, the Chern classes of $\V$ are
  $v_{1} = -2h$, $v_{2}=h^2$, and $v_{3}= v_{4}=0$, where $h$ denotes
  the hyperplane class on the projective space $B_1 = \P^{4}$.
\end{proof}









\subsubsection{Enumerating cubics in $B_1$}
\label{sec:enum-cubics-b_1}


%\begin{figure}[!]
 % \centering
  %\incfig{FamilyB1}
  %\caption{Schematic for the family parametrized by $B_1$. The points
   % $Z$ vary freely on the conic $C$, and we blow up the plane at the
    %six points $Z \cup \{y_{1},y_{2}\}$.}
    %\label{fig:FamilyB1}
%\end{figure}


For $B_1$ to be of any use, we must {\sl a priori} know the degree of
the map to moduli $\mu: B_1 \dashrightarrow \M$.  First note that
$\mu$ factors through the quasi-finite, degree
$72 \times \frac{6!}{4!}$ cover $\M^{\dagger}/S_{4} \to \M$.  Here,
the symmetric group $S_4$ acts by permuting the last four lines in a
six $(\ell_{1}, \dots, \ell_{6})$ on a cubic surface.


The moduli map $\mu: B_{1} \dashrightarrow \M$ naturally factors
through a map $\mu': B_1 \dashrightarrow \M^{\dagger}/S_{4}$ by
sending a general subscheme $Z \subset C$ to the marked cubic surface
$(\Bl_{Z}S, E_1, E_2, E_{Z})$. Here, $E_Z$ denotes the union of four
exceptional lines above the points of $Z$.


Thus, in order to compute $\deg \mu$, we merely need to compute
$\deg \mu'$.

  \begin{lemma}
    \label{lemma:degreemudaggerB1}
    $\mu': B_1 \dashrightarrow \M^{\dagger}/S_{4}$ has degree $2$.
  \end{lemma}

  \begin{proof}
    Beginning with a general marked cubic surface
    $(X, \ell_{1}, \ell_2, \dots, \ell_{6})$ we can blow down the six
    to produce a map $\varphi: X \to \P^2$. The map $\varphi$ is only
    well-defined up to the action of $\Aut(\P^{2})$.  By composing
    with an automorphism, we may assume that
    $\varphi(\ell_{1}) = y_{1}$ and $\varphi(\ell_{2}) = y_{2}$.

    The $4$ dimensional subgroup $H \subset \Aut(\P^{2})$ fixing the
    points $y_{1}, y_{2}$ acts on the $\P^{4}$ of conics passing
    through $y_{2}$. Let $G \subset H$ denote the stabilizer of $C$,
    viewed as a conic through $y_{2}$.  We will now show that $|G|=2$,
    from which the lemma follows.

    Any automorphism $h \in H$ preserving $C$ must fix the three
    points $y_{1},y_{2}$ and the point
    $y_{3}:= (\overline{y_{1},y_{2}} \cap C) \setminus \{y_{2}\}.$
    Therefore, $h$ must also fix $p := \Pole(y_{2},y_{3}) \in \P^{2}$.
    At the same time, $h$ must preserve the un-ordered pair of points
    $\Polar(y_{1}) \subset C$. It is now a simple exercise to see that
    there is a unique involution on $\P^{2}$ fixing $p$, preserving
    $C$ and interchanging the points of $\Polar(y_{1})$. Hence
    $|G|=2$, as desired.
  \end{proof}




\subsubsection{Goodness}
\label{sec:goodness1}

\begin{proposition}
  \label{prop:B1good}
  The family $\pi: X \to B_1$ is good.
\end{proposition}

\begin{proof}
  We use \autoref{prop:good}. If $[Z] \in B_{1}$ is any point,
  corresponding to a length $4$ subscheme $Z \subset C$, we
  immediately get a length $6$ admissible subscheme (see \cite[] )
  $Z' \subset \P^2$ by the rule
  \begin{align*}
    Z' = (Z+y_2) \cup \{y_{1}\}.
  \end{align*}
  Here, we view $C$ as a conic in $\P^{2}$ containing $y_{2}$, and the
  sum in $Z + y_{2}$ is as divisors on $C$.  The cubic surface
  $X_{[Z]} = \pi^{-1}([Z])$ is the same as the cubic surface $X_{Z'}$
  associated to $Z'$.

  In order to use \autoref{prop:good}, we next verify that the group
  $\Aut(\P^{2},Z')$ is finite.  First observe that $C \subset \P^{2}$
  must be the unique conic containing the length $5$ subscheme
  $ Z + y_{2} \subset Z'$.  Therefore, any element
  $h \in \Aut(\P^{2},Z')$ must preserve $C$.  As $y_{1}$ does not lie
  on $C$, we conclude that $h(y_{1}) = y_{1}$, and therefore $h$ must
  preserve the pair of points $\Polar(y_{1}) \subset C$. Furthermore,
  if we assume $h$ also fixes $y_{2}$, which we can do by passing to
  the connected component of the identity element in
  $\Aut(\P^{2},Z')$, then $h$ would have to preserve three distinct
  points on $C$, from which finiteness follows.
\end{proof}


\subsubsection{First relation}
\label{sec:firstrelation}

The results in this section, taken together, yield the following
relation on the indetermined coefficients in \eqref{eq:P}:
  

\begin{align}
  \label{eq:relation1}
  16 \cdot a_{1^4} + 4 \cdot a_{1^2 2} + a_{2^2} = 72 \times \frac{6!}{4!} \times 2 = 4320.
\end{align}

%------------------------------------------------------------------------------------

\subsection{Family $B_2$}
\label{sec:family-b_2}


This section constructs and analyzes a good family over the base
$$B_2 := \overline{\M}_{0,7},$$
the moduli space of Deligne-Mumford stable $7$-pointed genus $0$
curves.  To avoid unnecessary accumulation of symbols, we reset
notation such as $C,S,X, \dots$ from previous sections, and start
anew.



Before diving into details, we summarize the construction. Let
$C \subset \P^{2}$ be a smooth conic with $7$ distinct marked points
$s_{0}, s_{1}, \dots, s_{5}, s_{\infty}$ on $C$, and let
$t \in \P^{2} \setminus C$ denote the intersection point of the
tangent lines of $C$ at $s_{0}$ and $s_{\infty}$, i.e.
$t = \Pole(s_{0},s_{\infty})$.  Then the scheme
$Z := \{s_{1}, \dots , s_{5}, t \}$ is an admissible length $6$
subscheme of $\P^{2}$, and its associated cubic surface $X_{Z}$ has a
finite automorphism group.  The main objective of this section is to
demonstrate how this construction {\sl can be extended over the
  boundary of $\overline{\M}_{0,7}$, yielding a good family of cubic
  surfaces.}

\subsubsection{Basics of $\overline{\M}_{0,7}$}
\label{sec:basics-overlinem_0-7}

We denote a general stable $7$-pointed genus zero curve by
$$(C, s_{0}, s_{1}, \dots, s_{5}, s_{\infty}),$$
and we let 
\begin{align}
  \label{eq:PM07}
  \varphi: \mathcal{C} \to \overline{\M}_{0,7}
\end{align}
denote the universal stable curve, with universal marking sections
denoted $$\sigma_{0}, \sigma_{1}, \dots, \sigma_{5}, \sigma_{\infty} \subset \mathcal{C}.$$

We let $\mathcal{L}$ denote the line bundle
$\O_{\mathcal{C}}(-\sigma_{\infty})|_{\sigma_{\infty}} \simeq
\omega_{\varphi}|_{\sigma_{\infty}}$ -- we view it as a line bundle on
$\overline{\M}_{0,7}$, as is customary.  Following standard notation,
we set $\psi_{\infty} := c_{1}\mathcal{L}$.  The sole enumerative
result about $\overline{\M}_{0,7}$ we will use in this section is:

\begin{theorem}[ ???2 ]
  \label{theorem:psi}
  $$\int_{\overline{\M}_{0,7}} \psi_{\infty}^{4} = 1.$$
\end{theorem}

\subsubsection{Construction of the family}
\label{sec:construction-familyB2}


We move on to the construction of our good family over $B_2$. Start
with the rank $2$ vector bundle
$$\mathcal{A} := \varphi_{*}\O_{\mathcal{C}}(\sigma_{\infty})$$ on
$\overline{\M}_{0,7}$.  From the fact that the line bundle
$\O_{\mathcal{C}}(\sigma_{\infty})$ restricts to $\mathcal{L}^{-1}$ on
$\sigma_{\infty}$ and $\O_{\sigma_{0}}$ on $\sigma_{0}$, it follows
that
\begin{align*}
  \mathcal{A} \simeq \O_{\overline{\M}_{0,7}} \oplus \mathcal{L}^{-1}.
\end{align*}

Next, put $\o{\mathcal{C}} := \P \mathcal{A}^{\smvee}$;
$\overline{\mathcal{C}}$ is a $\P^{1}$-bundle over
$\overline{\M}_{0,7}$ with structural map
$$\o{\varphi}: \o{\cC} \to \o{\M}_{0,7}$$ and the surjection
$\varphi^{*}\mathcal{A} \to \O_{\mathcal{C}}(\sigma_{\infty})$ induces
a {\sl contraction map}

\begin{align}
  \label{eq:contract}
  \gamma: \mathcal{C} \to \overline{\mathcal{C}}.
\end{align}

The effect of $\gamma$ on a stable curve $(C, s_0, \dots, s_{\infty})$
is to contract all irreducible components not containing the marked
point $s_{\infty}$, leaving the unique $\P^{1}$ containing the marked
point $s_{\infty}$.  This remaining $\P^{1}$ is the corresponding
fiber of the $\P^{1}$-bundle $\overline{\mathcal{C}}$.

Under $\gamma$, the sections $\sigma_{i}$ map to sections, which we
denote by $\overline{\sigma_{i}}$, of $\o{\varphi}$.  These sections
have the following properties:
\begin{itemize}
\item $\overline{\sigma}_{\infty}$ is disjoint from all other sections, and
\item in each fiber of $\o{\varphi}$, the union of all seven sections
  is supported on at least three points.
\end{itemize}




The next step is to invoke the relative $2$-veronese embedding
$$\iota: \overline{\mathcal{C}} \hookrightarrow \cP:= \P \Sym^{2}\mathcal{A}^{\smvee}.$$
We let $\rho: \cP \to \o{\M}_{0,7}$ denote the structural map of this
$\P^{2}$-bundle.  In this way, $\overline{\mathcal{C}} \subset \cP$ is
a family of smooth conics with $7$ marked points
$\overline{\sigma}_{0}, \dots, \overline{\sigma}_{\infty}$ satisfying
the two properties just mentioned above.

\begin{definition}
  \label{def:tau} Let $\tau \subset \cP$ denote the section obtained
  by taking, fiber-by-fiber, the point polar (with respect to the
  conic $\iota(\overline{\mathcal{C}})$) to the line spanned by
  $\overline{\sigma}_{0}$ and $\overline{\sigma}_{\infty}$ in each
  fiber.
\end{definition}

Observe that, since $\o{\sigma}_{0}$ and $\o{\sigma}_{\infty}$ are {\sl
    disjoint}, it follows that the three sections
  $\tau, \o{\sigma}_{0},$ and $\o{\sigma}_{\infty}$ are non-collinear
  in every fiber of $\rho: \cP \to \o{\M}_{0,7}$.

\begin{lemma}
  \label{lemma:tau}
  The section $\tau \subset \cP$ is induced by an inclusion of the
  form
  $${\mathcal{L}} \hookrightarrow \Sym^{2} \mathcal{A}^{\smvee}.$$
\end{lemma}

\begin{proof}
  Let $\mathcal{T} \hookrightarrow \Sym^{2} \mathcal{A}^{\smvee}$
  denote the line sub-bundle corresponding to the section $\tau$ -- we
  must show $\mathcal{T} \simeq \mathcal{L}$.

  The sections
  $\o{\sigma_{0}}$ and $\o{\sigma}_{\infty}$ correspond, respectively,
  to inclusions of the form
  \begin{align*}
    \O \hookrightarrow \Sym^{2} \mathcal{A}^{\smvee}, \\
    \mathcal{L}^{2} \hookrightarrow \Sym^{2} \mathcal{A}^{\smvee}. \\
  \end{align*}

  Since $\tau$, $\o{\sigma}_{0}$, and $\o{\sigma}_{\infty}$ are
  non-collinear in every fiber of $\rho$, we deduce that the induced
  map
  $$\mathcal{T} \oplus \mathcal{L}^{2} \oplus \O \to \Sym^{2}
  \mathcal{A}^{\smvee}$$ is an isomorphism.  Since
  $\Sym^{2} \mathcal{A}^{\smvee} \simeq \O \oplus \mathcal{L} \oplus
  \mathcal{L}^{2}$, the claim now follows by considering determinants.
\end{proof}


\begin{definition}
  \label{def:ZB2} We let $\Sigma \subset \iota(\o{\mathcal{C}})$
  denote the union of the sections
  $\o{\sigma}_{1}, \dots, \o{\sigma}_{5}$.  We let
  $\mathcal{Z} \subset \cP$ denote the (disjoint) union
  $\Sigma \bigsqcup \tau$.
\end{definition}

The reader can immediately check that $\mathcal{Z} \subset \cP$
registers an admissible length $6$ subscheme in each fiber of
$\rho$. In light of this, we let
\begin{align}
  \label{eq:VB2}
  \mathcal{V} := \rho_{*}(\mathcal{I}_{\mathcal{Z}} \otimes \omega_{\rho}^{-1});
\end{align}
$\V$ is the natural rank $4$ vector bundle for the family of cubic
surfaces $$\pi: X \to \o{\M}_{0,7}$$ obtained by taking, for each
fiber of $\rho$, the cubic associated to the admissible subscheme
$\mathcal{Z} \subset \cP$.

\begin{proposition}
  \label{prop:goodnessB2} The family $\pi: X \to \o{\M}_{0,7}$ is
  good.
\end{proposition}

\begin{proof}
  Apply \autoref{prop:good}, after noting that in
  each fiber of $\rho$ the subscheme
  $\Sigma \subset \iota(\o{\mathcal{C}})$ is supported on at least
  three points.
\end{proof}

\subsubsection{Chern classes of $\V$}
\label{sec:chern-classes-v2}

We must now determing the Chern polynomial of the vector bundle
$\V = \rho_{*}(\mathcal{I}_{\mathcal{Z}} \otimes
\omega_{\rho}^{-1})$. In light of the exact sequence
\begin{align}
  \label{eq:exact2}
  0 \to \V \to \rho_{*}\omega_{\rho}^{-1} \to \rho_{*}(\omega_{\rho}^{-1}|_{\mathcal{Z}}) \to 0
\end{align}
it clearly suffices to determine the Chern polynomials of the rank
$10$ bundle $\rho_{*}(\omega_{\rho}^{-1})$ and the rank $6$ bundle
$\rho_{*}(\omega_{\rho}^{-1}|_{\mathcal{Z}})$.

\begin{lemma}
  \label{lemma:rank10}
  The bundle $\omega_{\rho}^{-1}$ is isomorphic to
  $\O_{\mathcal{P}}(3) \otimes \rho^{*}\mathcal{L}^{3}$, and
  \begin{align}
    \label{eq:rnk10}
    \rho_{*}(\omega_{\rho}^{-1}) \simeq \Sym^{3}(\Sym^{2} \mathcal{A})\otimes \mathcal{L}^{3}.
  \end{align}
\end{lemma}

\begin{proof}
  It suffices to prove the first assertion, that
  $\omega_{\rho}^{-1} \simeq \O_{\mathcal{P}}(3) \otimes
  \rho^{*}\mathcal{L}^{3}$.  This follows directly from the relative
  Euler exact sequence for $\rho$:
  \begin{align}
    \label{eq:eulerexact}
    0 \to \Omega_{\mathcal{P}/\o{\M}_{0,7}} \to \rho^{*}(\Sym^{2}\mathcal{A}^{\smvee})(-1) \to \O_{\cP} \to 0.
  \end{align}
\end{proof}

\begin{corollary}
  \label{cor:chern10} The Chern polynomial of
  $\rho_{*}(\omega_{\rho}^{-1})$ is
  \begin{align}
    \label{eq:chern10}
    (1-9\psi_{\infty}^{2}t^{2})(1-4\psi_{\infty}^{2}t^{2})(1-\psi_{\infty}^{2}t^{2})^{2}.
  \end{align}
\end{corollary}

\begin{proposition}
  \label{prop:omegarestrictZ} The Chern polynomial of $\rho_{*}(\omega_{\rho}^{-1}|_{\mathcal{Z}})$ is
  \begin{align}
    \label{eq:rank6}
    (1+3\psi_{\infty}t) (1+2\psi_{\infty}t) (1+\psi_{\infty}t)  (1-\psi_{\infty}t)
  \end{align}
\end{proposition}

\begin{proof}
  First, because $\mathcal{Z}$ is the disjoint union of
  $\Sigma \subset \o{\mathcal{C}}$ and $\tau$, we get a direct sum
  decomposition
  \begin{align}
    \label{eq:dirsum}
    \rho_{*}(\omega_{\rho}^{-1}|_{\mathcal{Z}}) = \rho_{*}(\omega_{\rho}^{-1}|_{\Sigma}) \oplus \rho_{*}(\omega_{\rho}^{-1}|_{\tau}).
  \end{align}
  The line bundle $\rho_{*}(\omega_{\rho}^{-1}|_{\tau})$ is easy to
  identify -- in fact it is trivial.  Indeed, \autoref{lemma:tau}
  implies that $\O_{\cP}(1)|_{\tau} \simeq \mathcal{L}^{-1}$, and the
  first claim in \autoref{lemma:rank10} then implies
  $\rho_{*}(\omega_{\rho}^{-1}|_{\tau}) \simeq \mathcal{L}^{-3}
  \otimes \mathcal{L}^{3} = \O_{\o{\M}_{0,7}}$.

  Thus, the Chern polynomial we need is the same as the Chern
  polynomial of $\rho_{*}(\omega_{\rho}^{-1}|_{\Sigma})$ -- we spend
  the rest of the proof extracting it.  Since
  \begin{align*}
    \rho_{*}(\omega_{\rho}^{-1}|_{\Sigma}) = \o{\varphi}_{*}(\iota^{*}\omega_{\rho}^{-1}|_{\Sigma})
  \end{align*}
  we may instead work within the $\P^{1}$-bundle $\o{\cC}$, where
  $\Sigma$ is a divisor.  Since
  $$\iota^{*}\O_{\cP}(1) = \O_{\o{\cC}}(2) = \O_{\o{\cC}}(2\o{\sigma}_{\infty}),$$
  and since $\Sigma$ is disjoint from $\o{\sigma}_{\infty}$, it follows that
  $$\omega_{\rho}^{-1}|_{\Sigma} \simeq \o{\varphi}^{*}(\mathcal{L}^{3})|_{\Sigma},$$ and therefore that
  $$\rho_{*}(\omega_{\rho}^{-1}|_{\Sigma}) \simeq \o{\varphi}_{*}(\o{\varphi}^{*}(\mathcal{L}^{3})|_{\Sigma}) = (\o{\varphi}_{*}\O_{\Sigma})\otimes \mathcal{L}^{3}.$$

  This means we must find the Chern polynomial of
  $\o{\varphi}_{*}\O_{\Sigma}$.  For this, we use a little trick.  In
  the $\P^{1}$-bundle $\o{\cC}$, every section $\o{\sigma}_{i}$ other
  than $\o{\sigma}_{\infty}$ is linearly equivalent to any other such
  section -- we leave it to the reader to deduce this from the
  observation that $\o{\sigma}_{\infty}$ is disjoint from all other
  sections $\o{\sigma}_{i}$.  So, there exists a one-parameter family
  of divisors interpolating between $\Sigma$ and, say, the divisor
  $5\o{\sigma}_{0}$.  Since the Chow ring of $\o{\M}_{0,7}$ is
  finitely generated, it follows that the Chern polynomials of
  $\o{\varphi}_{*}(\O_{\Sigma})$ and
  $\o{\varphi}_{*}(\O_{5 \o{\sigma}_{0}})$ are identical.

  In order to identify the Chern polynomial of
  $\o{\varphi}_{*}(\O_{5 \o{\sigma}_{0}})$, we iteratively use the
  sequences
  $$0 \to \mathcal{I}_{\o{\sigma}_{0}}^{n-1}/\mathcal{I}_{\o{\sigma}_{0}}^{n} \to \O_{n\o{\sigma}_{0}} \to \O_{(n-1)\o{\sigma}_{0}} \to 0$$ along with the isomorphism of line bundles
  $$\mathcal{I}_{\o{\sigma}_{0}}^{n-1}/\mathcal{I}_{\o{\sigma}_{0}}^{n} \simeq (\mathcal{I}_{\o{\sigma}_{0}}/\mathcal{I}_{\o{\sigma}_{0}}^{2})^{\otimes n-1}$$ to deduce that
  \begin{align}
    \label{eq:rk5}
    c(\o{\varphi}_{*}(\O_{5 \o{\sigma}_{0}})) = c(\O_{\o{\M}_{0,7}})c(\mathcal{I}_{\o{\sigma}_{0}}/\mathcal{I}_{\o{\sigma}_{0}}^{2})c((\mathcal{I}_{\o{\sigma}_{0}}/\mathcal{I}_{\o{\sigma}_{0}}^{2})^{2})\cdots c((\mathcal{I}_{\o{\sigma}_{0}}/\mathcal{I}_{\o{\sigma}_{0}}^{2})^{4}).
  \end{align}

  It remains to identify the conormal bundle
  $\mathcal{I}_{\o{\sigma}_{0}}/\mathcal{I}_{\o{\sigma}_{0}}^{2}$ and
  its relationship with the class $\psi_{\infty}$. For this, we leave
  it to the reader to verify: {\sl if a $\P^{1}$-bundle has two
    disjoint sections, then their conormal bundles are inverse to each
    other.}  Therefore,
  $$c(\o{\varphi}_{*}(\O_{5 \o{\sigma}_{0}})) = c(\O)c(\mathcal{L}^{-1})c(\mathcal{L}^{-2})c(\mathcal{L}^{-3})c(\mathcal{L}^{-4}).$$

  The proposition now follows by tracing backwards, starting with this
  latest identity.
\end{proof}

\begin{corollary}
  \label{cor:chernpolyV2} The Chern polynomial of $\V$ is
  $(1-3\psi_{\infty}t)(1-2
  \psi_{\infty}t)(1-\psi_{\infty}^{2}t^{2})$. Therefore,
  \begin{align}
    \label{eq:chernV2}
    v_1 = -5\psi_{\infty}\\
    v_2 = 5 \psi_{\infty}^{2}\\ \nonumber
    v_3 = 5 \psi_{\infty}^{3}\\ \nonumber
    v_{4} = -6 \psi_{\infty}^{4}.\\ \nonumber
  \end{align}
\end{corollary}

\subsubsection{Enumerating cubics in $B_{2}$}
\label{sec:enum-cubics-b_2}


%\begin{figure}[!]
   % \centering
   %\incfig{FamilyB2}
    %\caption{Schematic for our second family. We blow up the plane at
     % the points $\{s_{1}, \dots, s_{5}, t\}$.}
    %\label{fig:FamilyB2}
    % \end{figure}


Now that we've identified the $v_{i}$, we need only establish
$\deg \mu$, where $\mu: B_{2} \dashrightarrow \M$ is the map to
moduli.  Clearly, $\mu$ factors through
$\mu':B_{2} \dashrightarrow \M^{\dagger}$: $\mu'$ sends a general
$7$-pointed stable curve $(C,s_0,s_1, \dots, s_5, s_{\infty})$ to the
marked cubic surface
$(Bl_{\{s_1, \dots, s_{5}, t\}}\P^{2}, E_1, \dots, E_5, E_t)$. Here,
$C$ is embedded in the plane as a conic, and $E_i$ denotes the
exceptional divisor over the corresponding point.

Clearly, $\deg \mu = 51840 \times \deg \mu'$.  Thus our analysis will
be complete once we show:

\begin{lemma}
  \label{lemma:muprime2} $\deg \mu' = 2.$
\end{lemma}

\begin{proof}
  We need to show that a general marked cubic surface
  $(X, \ell_{1}, \dots, \ell_{6}) \in \M^{\dagger}$ has two preimages
  in $\o{\M_{0,7}}$.
  
  
  Starting with $(X, \ell_{1}, \dots, \ell_{6})$, let
  $(s_1, \dots, s_5, t) \subset \P^{2}$ denote the images of
  $\ell_{i}$ under the map blowing down the six, well defined up to
  $\Aut \P^{2}$.  The five points $s_{i}$ lie on a unique smooth conic
  $C$ on which there is a unique un-ordered pair of points $\{x,y\}$
  whose tangent lines intersect at $t$.  The two stable curves
  $(C,x,s_1, \dots, s_5, y)$ and $(C,y,s_1, \dots, s_5, x)$ are then
  the two claimed preimage points in $\o{\M}_{0,7}$, proving the
  lemma.
\end{proof}


\subsubsection{Second relation}
\label{sec:secondrelation}

By assembling the information in this section, using
\autoref{theorem:psi} to compute the various monomials in the $v_{i}$,
we find our second relation:
\begin{align}
  \label{eq:relation2}
  625 a_{1^{4}} + 125 a_{1^{2}2} - 25a_{13} + 25a_{2^2} - 6 a_{4} = 2 \times 51840.
\end{align}

%------------------------------------------------------------------------------


\subsection{Family $B_3$}
\label{sec:family-b_3}


%\begin{figure}[!]
  %  \centering
   % \incfig{FigureB3}
    %\caption{Third family schematic. The point $\nu(s_{\infty})$ is
     % the point at infinity. It, along with
      %$\nu(s_{i}), i=1, \dots, 5$ are to be blown up.}
    %\label{fig:FigureB3}
%\end{figure}


Let us reset notation once again.  Our third family is a variant of
the family in the previous section \autoref{sec:family-b_2}.  It is
parametrized by the base variety
$$B_{3} := \hM_{0,7}.$$

Here, $\hM_{0,7}$ is a particular Hassett space parametrizing
${\bf w}$-stable $7$-pointed genus $0$ curves, where
$${\bf w} = \left(1-\epsilon, \epsilon, \dots, \epsilon, 1-3 \epsilon \right), 0 < \epsilon \ll 1$$
is the weight vector on the marked points
$(s_{0}, s_{1}, \dots, s_{5}, s_{\infty})$ on a genus $0$ curve.

Before getting buried in details, we explain the basic construction of
the family of cubic surfaces over the interior, $\M_{0,7}$.  Begin
with $(C, s_{0}, s_{1}, \dots, s_{5}, s_{\infty})$, a smooth rational
curve with $7$ distinct marked points, and choose a map
$$\nu: C \to \P^{2},$$ birational onto its image
$\nu(C)$, which is a {\sl cuspidal cubic curve with cusp point
  $\nu(s_{0})$ and unique flex point $\nu(s_{\infty})$.}  Let
$Z = \nu(\{s_{1}, \dots, s_{5}, s_{\infty}\}) $, and now assign the
cubic surface $\Bl_{Z}\P^{2}$.  The main objective of this section is
to demonstrate that this assignment extends over the boundary of
$\hM_{0,7}$, yielding a good family of cubic surfaces from which we
can extract one more relation.

Before starting the analysis, we make some remarks: The need for using
the alternate compactification $\hM_{0,7}$ ultimately traces to the
requirement that our family must be good.  There are several possible
variants of the construction, but the reader will eventually realize
that, very often, there will be a few non-good fibers present.


Finally, in the interest of brevity, {\sl we assume the reader is
  familiar with the intersection theory of $\o{\M}_{0,n}$, as
  explained in, for example \cite[??].}  Therefore, we will at time
translate intersection products on $\hM_{0,7}$ into calculations on
$\o{\M}_{0,7}$, and then leave it to the reader to check the results
there.

\subsubsection{The spaces $\hM_{0,n}$}
\label{sec:spaces-hm_0-n}

In order to carry out the calculations needed for our third relation,
we need to investigate the basic geometry of a series of
Hassett-spaces which we denote by $\hM_{0,n}$.  We do that in this
section.

\begin{definition}
  \label{def:hM0n} For each $n \geq 4$, we define $\hM_{0,n}$ to be
  the compactification of $\M_{0,n}$ by ${\bf w}$-stable, $n$-pointed,
  genus $0$ curves, where
  $${\bf w} = \left(1- \epsilon, \epsilon, \dots, \epsilon, 1-(n-4)\epsilon \right), 0 < \epsilon \ll 1.$$
\end{definition}

We will denote a ${\bf w}$-stable curve by
$(C, s_{0}, s_{1}, \dots, s_{n-2}, s_{\infty})$.  The indices are
meant to indicate that the two points $s_{0}$ and $s_{\infty}$ are two
``more interesting'' points, as will become apparent in the next
paragraph.

The pointed curves $(C, s_{0}, s_{1}, \dots, s_{n-2}, s_{\infty})$
parametrized by $\hM_{0,n}$ are actually very simple to describe:

\begin{itemize}
\item $C$ is a {\sl smooth} rational curve,
\item $s_{0}$ and $s_{\infty}$ must be distinct,
\item at most one of the $s_{i}$, $i=1, \dots, n-2$ can coincide with
  $s_{0}$,
\item at most $n-4$ of the $s_{i}$, $i=1, \dots, n-2$ can coincide
  with $s_{\infty}$.
\item there are no constraints on how the points $s_{i}$,
  $i=1, \dots, n-2$ can coincide with each other.
\end{itemize}
Therefore, the universal curve $$\varphi: \cC \to \hM_{0,n}$$ is in
fact a $\P^{1}$-bundle, the projectivization of a particular rank $2$
vector bundle $\mathcal{W}$ on $\hM_{0,n}$. We let
$\sigma_{i} \subset \cC$,
$i \in \left\{0,1,2, \dots, n-2, \infty\right\}$ denote the universal
sections providing the marked points, and we denote by
\begin{align}
  \label{eq:stab}
  \zeta: \overline{\M}_{0,n} \to \hM_{0,n}
\end{align}
the stabilization morphism.  We will sometimes use $\zeta$ to
pull-back enumerative calculations on $\hM_{0,n}$ back to
$\overline{\M}_{0,n}$, where they are better understood.

\subsubsection{Divisor classes on $\hM_{0,n}$}
\label{sec:divisor-classes-hm_0}

Certain divisors on $\hM_{0,n}$  play an essential role in our
computations, so we take this section to introduce them.

\begin{definition}
  \label{def:psihats} We let $\hL_{0}, \hL_{\infty}$ denote the line
  bundles
  $\omega_{\varphi}|_{\sigma_{0}},
  \omega_{\varphi}|_{\sigma_{\infty}}$, respectively, and let
  $\hpsi_{0}, \hpsi_{\infty} \in A^{1}(\hM_{0,n})$ denote their first
  Chern classes.
\end{definition}

Since the sections $\sigma_{0}$ and $\sigma_{\infty}$ are disjoint, it
follows that
\begin{align}
  \label{eq:opposites}
  \hpsi_{0} + \hpsi_{\infty} = 0.
\end{align}


Finally, we indicate the simple relationship between $\hpsi_{0}$ on
$\hM_{0,n}$ with $\psi_{0}$ on $\o{\M}_{0,n}$.  Let
$\delta_{\{i,0\}}, i=1, \dots, n-2$ denote the boundary divisor in
$\o{\M}_{0,n}$ whose general point corresponds to the union of two
marked $\P^{1}$'s, one of which contains only $0$-th and $i$-th marked
points.



Then,
\begin{align}
  \label{eq:relationpsis}
\zeta^{*}(\hpsi_{0}) = \psi_{0} - \sum_{i \in \{1, \dots, n-2\}} \delta_{\{i,0\}}.  
\end{align}













\begin{definition}
  \label{def:boundary}
  For each distinct pair of indices
  $i,j \in \{0, 1, 2, \dots, n-2, \infty \}$, we define
  $\Delta_{i,j} \subset \hM_{0,n}$ to be the divisor over which the
  sections $\sigma_{i}$ and $\sigma_{j}$ intersect, and we set
  $$\Delta_{\infty} = \sum_{i \in \{1, \dots, n-2 \}}
  \Delta_{i,\infty}.$$
\end{definition}

The specific boundary divisors $\Delta_{i, \infty}$, where the section
$\sigma_{i}$ collides with $\sigma_{\infty}$ will play an especially
important role in our calculations will be important for us, so we
briefly collect some observations about them:
\begin{itemize}\label{item:boundaryprops}
\item Each boundary divisor $\Delta_{i,\infty} \subset \hM_{0,n}$ is
  naturally isomorphic to $\hM_{0,n-1}$ via a map
  $$\eta_{i}: \hM_{0,n-1} \to \hM_{0,n}$$ that sends
  $(C, s_{0}, \dots, s_{n-3}, s_{\infty})$ to
  $(C, s_{0}, \dots, s_{i-1}, s_{i}=s_{\infty}, s_{i+1} \dots,
  s_{\infty})$.

  In the same way, if $i \neq j$, then the intersection
  $\Delta_{i,\infty} \cap \Delta_{j, \infty}$ is isomorphic to
  $\hM_{0,n-2}$ via a similarly defined map
  $$\eta_{i,j}: \hM_{0,n-2} \to \hM_{0,n},$$ and so on.  This provides a
  nice inductive structure among the Hassett spaces $\hM_{0,n}$.
  
\item Under $\eta$, we clearly have $\eta^{*}\hpsi_{0} = \hpsi_{0}$
  and $\eta^{*}\hpsi_{\infty} = \hpsi_{\infty}.$
\item We have:
\begin{align}\label{eq:pullbackdelta}
  \eta^{*}([\Delta_{i,\infty}]) = -\hpsi_{\infty} \in A^{1}(\hM_{0,n-1}).
\end{align}
To see this, first note that
$\Delta_{i, \infty} = \varphi_{*}(\sigma_{i}\sigma_{\infty})$, and so
$\eta^{*}([\Delta_{i,\infty}]) = \varphi_{*}(\sigma_{\infty}^{2})$,
where the latter equation is happening in the divisor class group of
$\hM_{0,n-1}$.  But
$\varphi_{*}(\sigma_{\infty}^{2}) = - \hpsi_{\infty}$, and the claim
follows.
\end{itemize}


\subsubsection{Computing integrals in $\hM_{0,n}$}

These observations are enough to compute, using an inductive method,
any top degree integral involving $\hpsi_{0}, \hpsi_{\infty},$ and the
boundary divisors $\Delta_{i, \infty}$.

\begin{example}
  \label{ex:toppsi}
  Let's first indicate how to
  compute $$\int_{\hM_{0,n}}\hpsi_{0}^{n-3}.$$ Using
  \eqref{eq:relationpsis}, this amounts to computing
  $$\int_{\o{\M}_{0,n}} \left(\psi_{0} - \sum_{i \in \{1, \dots, n-2\}} \delta_{\{i,0\}}\right)^{n-3}.$$
  Now, in the Chow ring of $\o{\M}_{0,n}$, recall that:
  \begin{itemize}
  \item $\psi_{0} \cdot \delta_{\{i,0\}} = 0$, 
  \item $\delta_{\{i,0\}} \cdot \delta_{\{j,0\}} = 0$ for $i \neq j$,
    and
  \item $\int_{\o{\M}_{0,n}}\delta_{\{i,0\}}^{n-3} = (-1)^{n}$, for
    all $i$, and
  \item $\int_{\o{\M}_{0,n}} \psi_{0}^{n-3} = 1$.
  \end{itemize}

  Using these, we deduce:
  \begin{align}
    \label{eq:hpsitop}
    \int_{\hM_{0,n}}\hpsi_{0}^{n-3} = -(n-3). 
  \end{align}
\end{example}

Next, we do an example involving $\Delta_{i, \infty}$'s.

\begin{example}
  Consider
  $$\int_{\hM_{0,7}} \Delta_{i, \infty}^{2}\Delta_{j, \infty}^{2},$$
  where $i \neq j$.  Using the parametrization
  $\eta_{i,j} : \hM_{0,5} \to \Delta_{i,\infty} \cap
  \Delta_{j,\infty}$, we see that this integral is the same as
  $$\int_{\hM_{0,5}}\eta_{i,j}^{*}[\Delta_{i, \infty}] \cdot \eta_{i,j}^{*}[\Delta_{j, \infty}],$$
  which in turn is equal to
  $$\int_{\hM_{0,5}}(-\hpsi_{\infty})^{2} = \int_{\hM_{0,5}}\hpsi_{0}^{2} = -2,$$
  as seen by combining \eqref{eq:hpsitop} with previous observations
  on how boundary divisors pull back along $\eta_{i}$ maps.
\end{example}

The intersection theory described in this subsection yields:

\begin{lemma}
  \label{lemma:monomials} On $\hM_{0,7}$ we have:
  \begin{align}
    \int_{\hM_{0,7}}\widehat{\psi}_{0}^{4}& = -4, \\ \nonumber
    \int_{\hM_{0,7}}\widehat{\psi}_{0}^{3}\Delta_{\infty}& = -15, \\ \nonumber
    \int_{\hM_{0,7}}\widehat{\psi}_{0}^{2}\Delta_{\infty}^{2}& = -55, \\ \nonumber
    \int_{\hM_{0,7}}\widehat{\psi}_{0}\Delta_{\infty}^{3}& = -195, \\ \nonumber
    \int_{\hM_{0,7}}\Delta_{\infty}^{4}& = -655
  \end{align}
\end{lemma}

\subsubsection{Construction of the family over $\hM_{0,7}$}
\label{sec:constr-family-hM07}

Let us proceed to the formal construction of our family, starting with
the universal curve
$$\varphi: \cC \to \hM_{0,7}.$$
The idea is to map $\cC$ to a relative cuspidal cubic curve in a
$\P^{2}$-bundle so that $\sigma_{0}$ becomes the cusp point and
$\sigma_{\infty}$ becomes the flex point.


Consider the rank $2$
vector bundle $\mathcal{A} = \varphi_{*}\O_{\cC}(\sigma_{\infty})$ --
since $\sigma_{\infty}$ and $\sigma_{0}$ are disjoint, and since
$\O_{\cC}(\sigma_{\infty})|_{\sigma_{\infty}} = \hL_{\infty}^{-1}$ and
$\O_{\cC}(\sigma_{0})|_{\sigma_{0}} \simeq \O_{\hM_{0,7}}$, it follows
that
$$\mathcal{A} \simeq \O_{\hM_{0,7}} \oplus \hL_{\infty}^{-1}.$$
We emphasize that the above direct sum decomposition is induced by
applying $\varphi_{*}$ to the restriction map
$\O_{\cC}(\sigma_{\infty}) \to \O_{\cC}(\sigma_{\infty})|_{\sigma_{0}
  \cup \sigma_{\infty}}.$

  Next, consider the subbundle
  \begin{align}
    \label{eq:S}
    \mathcal{S} = :\O_{\hM_{0,7}} \oplus \hL_{\infty}^{-2} \oplus \hL_{\infty}^{-3} \subset \Sym^{3} \mathcal{A} = \O_{\hM_{0,7}} \oplus \hL_{\infty}^{-1} \oplus \hL_{\infty}^{-2} \oplus \hL_{\infty}^{-3},
  \end{align}
  and denote by
  \begin{align}
    \label{eq:nu}
    \nu: \cC \to \P \mathcal{S}^{\smvee}
  \end{align}
  the $\hM_{0,7}$-morphism induced by the surjection
  $\varphi^{*}\mathcal{S} \to \O_{\cC}(3\sigma_{\infty})$. Denote by
  $\rho: \P \mathcal{S}^{\smvee} \to \hM_{0,7}$ the $\P^{2}$-bundle
  structural map.  Fiber-by-fiber, the effect of $\nu$ is to map the
  fibers of $\cC$ to a cuspidal cubic curve, with $\nu(\sigma_{0})$
  the cusp point and $\nu(\sigma_{\infty})$ the flex point.

  We denote by $\mathcal{Z} \subset \cC$ the union of the six sections
  $\sigma_{1}, \dots, \sigma_{\infty}$.  Because at most one of the
  sections $\sigma_{i}, i \in \{1, \dots, \infty\}$ is allowed to
  intersect $\sigma_{0}$ (which becomes the cusp under $\nu$), it
  follows that $\nu|_{\mathcal{Z}}:\mathcal{Z} \to \nu(\mathcal{Z})$
  is an isomorphism. As will be demonstrated in
  \autoref{sec:goodness3}, every member of the family of length $6$
  subschemes $\nu(\mathcal{Z}) \subset \P \mathcal{S}^{\smvee}$ is
  admissible, and has a {\sl good} associated cubic surface.
  Therefore, our attention is brought upon the natural rank $4$ bundle
  \begin{align}
    \label{eq:V3}
    \V := \rho_{*}\left(\omega_{\rho}^{-1} \otimes \mathcal{I}_{\nu(\mathcal{Z})}\right)
  \end{align}
  of our claimed good family $$\pi: \cX \to \hM_{0,7},$$ with
  $\cX := \Bl_{\nu(\mathcal{Z})}\P \mathcal{S}^{\smvee}$.


\subsubsection{Goodness}
\label{sec:goodness3}

Here we address the goodness of the family
$\pi: \mathcal{X} \to B_{3}$. We let $$T \subset B_{3} = \hM_{0,7}$$
denote the finite set parametrizing marked curves
$(C, s_{0}, s_{1}, \dots, s_{5}, s_{\infty})$ where three of the
$s_{i}, i=1, \dots, 5$ coincide with $s_{\infty}$.  $T$ consists of
${5 \choose 3 } = 10$ points.  Using \autoref{prop:good}, it will be
relatively easy to show that $\pi: \mathcal{X} \to B_{4}$ is good over
the open set $B_{3} \setminus T$.  The argument showing that the
fibers over $T$ are also good is quite a bit more involved.  In the
interest of exposition, we have placed it in \autoref{sec:goodnessA5}.




\autoref{theorem:goodcusp} below essentially states that
$\pi : \mathcal{X} \to B_{3}$ is a good family over the open set
$B_{3} \setminus T$. The following elementary lemma, which we leave to
the reader, contains simple observations about automorphisms of
$\P^{2}$.

\begin{lemma}
  \label{lemma:identity} Suppose $g \in \Aut(\P^{2})$ satisfies one of:
  \begin{enumerate}
  \item $g$ restricts to the identity on a line $L \subset \P^{2}$ and
    also fixes two points outside $L$, or
  \item $g$ restricts to the identity on a line $L$, preserves another
    line $L'$ and fixes a point outside $L \cup L'$, or
  \item $g$ fixes $4$ points, no three collinear.
  \end{enumerate}
  Then $g = \id$.
\end{lemma}

\begin{proof}
  Omitted.
\end{proof}

\begin{theorem}
  \label{theorem:goodcusp}
  Let $C \subset \P^{2}$ be a cuspidal cubic, and let
  $s_{i} \in C, i \in \{1, \dots, 5, \infty \}$ be six, not
  necessarily distinct, points satisfying:
  \begin{enumerate}
  \item At most one $s_{i}$ is the cusp point,
  \item $s_{\infty}$ is the flex point of $C$, and
  \item at most {\bf two} $s_{i}, 1 \leq i \leq 5$ are equal to $s_{\infty}$.
  \end{enumerate}
  
  The length $6$ subscheme $Z := s_{1} + \dots + s_{5} + s_{\infty}$,
  interpreted as a divisor on $C$, is admissible and the associated
  cubic $X_{Z}$ is good.
\end{theorem}

\begin{proof}
  First note that since no length $4$ subscheme of $Z$ is contained in
  a line, it follows that the length $5$ subscheme
  $Z' := s_1 + s_2 + s_3 +s_4 +s_5$ is contained in a unique conic,
  which we denote by $Q$.

  The verification proceeds in cases, depending on the rank of the
  conic $Q$ and its relative position with respect to the cuspidal
  cubic $C$.  Observe first that every element in the identity
  component of $\Aut(\P^{2},Z)$ must preserve the conic $Q$. So, in
  light of \autoref{prop:good} it suffices to show that there are only
  finitely many elements in $\Aut(\P^{2},Z)$ preserving $Q$.  Let
  $g \in \Aut(\P^{2},Z)$ denote an arbitrary element in the identity
  component -- we must deduce in every case that $g = \id$.  This
  strategy works in the first four cases. The fifth case proceeds
  differently, and appeals to the classification of semi-stable cubic
  surfaces.

  
  {\bf Case 1}: {\sl $Q$ is smooth and $s_{\infty} \notin Q$.}    Then
  $g$ must preserve the pair of points $\Polar(s_{\infty})$. Since $Z$
  is a subscheme of the irreducible cubic curve $C$, it follows that
  $Z'$ must have multiplicity at most $2$ at each of the pair of
  points $\Polar(s_{\infty})$.  This implies that $Z' \subset Q$ is
  supported on at least three points, and so $g$ is an automorphism of
  $Q$ fixing three points, i.e. $g = \id$.

  

  {\bf Case 2}: {\sl $Q$ is smooth and $s_{\infty} \in Q$.}  Let
  $C^{\circ}$ denote the smooth part of $C$ -- The classical chord and
  tangent method makes $C^{\circ}$ isomorphic to the additive group
  $\G_{a}$, with origin corresponding to $s_{\infty}$. If
  $Z \subset Q$ were supported on less than three points, necessarily
  $s_{\infty}$ and, say $s_{1}$, then, given condition (3) we deduce
  that $s_{1}$ must be contained in $C^{\circ}$ and must be a torsion
  point in the group law, meaning $s_{1} = s_{\infty}$. Thus, $Z$
  would have to be entirely supported on $s_{\infty}$, which is
  contrary to assumption.  Since $Z$ has at least three points of
  support, we deduce $g =\id$.

  
  {\bf Case 3}: {\sl $Q$ is twice a line $L$.} $L$ cannot be the flex
  line at $s_{\infty}$, because it would force $Z$ to be
  $6 s_{\infty}$, violating condition (3).  Nor can $L$ be the unique
  line joining $s_{\infty}$ and the cusp of $C$, since then the
  support of $Z$ would have to contain that cusp with multiplicity at
  least $2$, contrary to condition (1). Thus, $L$ meets $C$
  transversely at three distinct smooth points, including
  $s_{\infty}$.  Without loss of generality, we may assume
  $Z = 2s_1 + 2s_2 + 2s_{\infty}$, where
  $L \cap C = s_{1}+s_{2} + s_{\infty}$.

  
  Now, $g \in \Aut(\P^{2},Z)$ fixes three points on $L$, and hence
  restricts to the identity on $L$. Let
  $q_{1,2}, q_{1, \infty}, q_{2,\infty}$ denote the corresponding
  pairwise intersections of the three tangent lines to $C$ at
  $s_1, s_2, s_{\infty}$. $q_{1,2}, q_{1, \infty},$ and
  $ q_{2,\infty}$ are distinct: If two coincided at a point $p$, then
  $p$ must be a point on the flex line, and must therefore not lie on
  $C$. Projecting $C$ from $p$ would yield an order $2$ ramification
  point at $s_{\infty}$, two simple ramification points at 
  $s_{1}, s_{2}$ and another ramification point at the cusp.  This
  exceeds the length $4$ ramification scheme given by the
  Riemann-Hurwitz formula for a degree $3$ map between rational
  curves.  Thus, $g$ fixes $L$, and fixes at least two other points
  outside $L$; by \autoref{lemma:identity}, $g =\id$.

  
  {\bf Case 4}: {\sl $Q$ is the union of two distinct lines
    $L \cup M$, neither of which is the flex line at $s_{\infty}$.}
  Let $x$ denote the point $L \cap M$, and {\sl first assume that
    $x \notin C$.} $Z'$ must contain at least two points on each of
  the lines. Since $g$ must fix $x$, we can apply
  \autoref{lemma:identity} to deduce $g =\id$ in this case.

  Next, assume $x \in C$: $x$ is not the cusp point, because then
  there is no room for $Z'$, given condition (1). We leave it to the
  reader to check all combinatorial possibilities in this situation,
  depending on whether $L$ and $M$ are tangent to $C$ or not.  All
  situations here are accounted for by \autoref{lemma:identity}.

  {\bf Case 5}: {\sl $Q$ is the union of two distinct lines
    $L \cup M$, where $M$ is the flex line through $s_{\infty}$.} This
  is the most interesting case.  Once again, let $x = L\cap M$.  If
  $x \in C$, $x$ must be the flex point $s_{\infty}$, and $Z$ would
  have to have multiplicity at least $4$ at $s_{\infty}$, violating
  condition (3).

  So, we can assume $x \notin C$.  There is now only one option for
  $Z$: $Z$ must equal $Q \cap C$.  In this case, $\Aut(\P^{2}, Z)$ has
  positive dimensional automorphism group, and therefore
  \autoref{prop:good} does not apply.  Instead, the reader can check
  that $X_{Z}$ is a normal cubic surface with only $A_{1}$ and $A_{2}$
  singularities.  Hence $X_{Z}$ is GIT semi-stable.  Semi-stable
  surfaces are automatically good, finishing this case.
  
\end{proof}

  
It remains to prove that the finitely many fibers of
$\pi: \mathcal{X} \to B_{3}$ over $T \subset B_{3}$ are good.  Since
this argument is rather involved (and contains a method of independent
interest), we have moved it to \autoref{sec:goodnessA5}.

\begin{corollary}
  \label{cor:good3} $\pi: \cX \to B_{3}$ is good.
\end{corollary}

\subsubsection{The bundle $\V$ and its Chern classes}
\label{sec:bundle-v3}

As usual, we must compute the Chern classes of
$\V = \pi_{*}\left( \omega_{\pi}^{-1} \right)$.  The analysis runs
parallel to that found in \autoref{sec:chern-classes-v2}, and so we
skip the details, and report the results:

\begin{proposition}
  \label{prop:chernV3} The total Chern polynomial of $\V$ is:
  \begin{align}
    \label{eq:chernV3}
    c(\V) = \frac{1+\hpsi_{0}t}{1-4 \hpsi_{0}t}\left(1-\Delta_{\infty}t \right) \left(1 + (\hpsi_{0}-\Delta_{\infty})t \right)  \left(1 + (2\hpsi_{0}-\Delta_{\infty})t \right) \left(1 + (3\hpsi_{0}-\Delta_{\infty})t \right).
  \end{align}

  Furthermore, we have:
  \begin{align}
    \label{eq:monomialsV3}
    v_{1}^{4}& = 3436, \\ \nonumber
    v_{1}^{2}v_{2}& = 1076,\\ \nonumber
    v_{2}^{2}& = 316, \\ \nonumber
    v_{1}v_{3}& = 116, \\ \nonumber
    v_{4}& = 0. \nonumber
  \end{align}
\end{proposition}

\begin{proof}
  Mimic the analysis done in \autoref{sec:chern-classes-v2} to
  identify $c(V)$, and use the calculations in
  \autoref{lemma:monomials} to compute the monomials.
\end{proof}

\subsubsection{Third relation}
\label{sec:third-relation}

The degree of the natural map
$\mu': B_{3} \dashrightarrow \M^{\dagger}$ is $20$, by
\autoref{theorem:Schubert}. Therefore, by compiling all that we have
calculated in this section, we finally obtain our third relation:
\begin{align}
  \label{eq:relation3}
  3436a_{1^4} + 1076a_{1^{2}2} + 116 a_{13} + 316 a_{2^{2}} = 20 \times 51840.
\end{align}




%----------------------------------------------------------------------------



\subsection{Family $B_4$}




\label{sec:family-b_4}

We construct our fourth family $\pi: X \to B_4$ by starting with a
quintic del Pezzo surface $S$ (the blow up of the plane at four
general points) and, informally, ``blowing up $S$ at two points which
vary freely.''  The complication arises because we get a non-good
cubic surface by blowing up the surface $S$ at a pair of points
contained in one of its ten lines.  Most effort is concentrated on
resolving this issue.

The surface $S$ has $10$ exceptional lines which we denote by
$L_{1}, \dots, L_{10}$.  Each line $L_{i}$ meets exactly three other
(mutually disjoint) lines, and we let $T_{i} \subset S$ denote their
union. Furthermore, we let $\tau_{i} \subset L_{i}$ denote
$T_{i} \cap L_{i}$.

We let $S^{[2]}$ denote the Hilbert scheme of
pairs of points on $S$.  $S^{[2]}$ contains $10$ disjoint $\P^{2}$'s,
$\Lambda_1, \dots, \Lambda_{10}$, namely the Hilbert schemes
$L_{i}^{[2]}$ parametrizing degree $2$ subschemes contained in the
lines $L_{i}$. Let $\Lambda = \cup_{i}\Lambda_{i}$.  The crux of this
section is to show that the blow up $\Bl_{\Lambda}S^{[2]}$ hosts a
good family of cubic surfaces.


We let $\pi_{1}, \pi_{2}$ denote the projections of $S^{[2]} \times S$
to either factor, and denote by
$$\mathcal{Z} \subset S^{[2]} \times S$$ the universal subscheme. 

\subsubsection{Construction of $B_4$}
\label{sec:construction-b_4}





First, it is easy to see (we leave the verification to the reader)
that the open set
\begin{align} 
  \label{eq:UB3}
  U := S^{[2]} \setminus \Lambda
\end{align}
supports a good family of cubic surfaces, namely
\begin{align}
  \label{eq:XUB3}
  X^{\circ} := \Bl_{\mathcal{Z}_{U}}(U \times S).
\end{align}


However, when we extend $X^{\circ}$ in the naive way over $S^{[2]}$
the family acquires non-good fibers, namely unions of quadrics and
planes, and therefore we must modify $S^{[2]}$ along $\Lambda$.

Let
\begin{align}
  \label{eq:S2tilde}
  \widetilde{S^{[2]}} := \Bl_{\Lambda}S^{[2]}
\end{align}
denote the blow-up with blow-down map
$\beta: \widetilde{S^{[2]}} \to S^{[2]}$.  Let
$E_{i} = \beta^{-1}(\Lambda_{i})$, $i=1, \dots, 10$ denote the
components of the of the exceptional divisor of the blow-up; $\beta$
expresses $E_{i}$ as a $\P^{1}$-bundle over $\Lambda_{i}$.

Next, let $\widetilde{Z} := \widetilde{S^{[2]}} \times_{S^{[2]}} Z$
denote the fiber-product; $\widetilde{Z}$ is a closed subscheme of the
product $\widetilde{S^{[2]}} \times S$, and is a finite, flat, degree
two cover of $\widetilde{S^{[2]}}$ under the first projection
$\tilde{\pi}_{1} : \widetilde{Z} \to \widetilde{S^{[2]}}.$

Inside the product $\widetilde{S^{[2]}} \times S$ we define the
smooth, closed, codimension $2$ subschemes
\begin{align}
  \label{eq:Fi}
  F_{i} := E_{i} \times L_{i};
\end{align}
we let $F := \cup_{i}F_{i}$.  We perform one more blow-up along $F$ to
get
\begin{align}
  \label{eq:Xtilde}
  \mathcal{Y} := \Bl_{F}\left( \widetilde{S^{[2]}} \times S \right),
\end{align}
and denote by $\eta: \mathcal{Y} \to \widetilde{S^{[2]}} \times S $
the blowdown.  The next lemma is critical to our construction.

\begin{lemma}
  The closed immersion
  $i: \widetilde{\mathcal{Z}} \hookrightarrow \widetilde{S^{[2]}}
  \times S$ lifts to a closed immersion
  $j: \widetilde{\mathcal{Z}} \hookrightarrow  \mathcal{Y}$.
\end{lemma}

\begin{proof}
  The main observation is that we have an equality of schemes
  \begin{align}
    \label{eq:Cartier}
    F_i \cap \widetilde{\mathcal{Z}} = \widetilde{\pi}_{1}^{-1}(E_i) \cap \widetilde{\mathcal{Z}}
  \end{align}
  for all $i=1, \dots, 10$, {\sl and therefore
    $F \cap \widetilde{\mathcal{Z}}$ is a Cartier divisor on
    $\widetilde{\mathcal{Z}}$}.  Thus, the natural blowdown map

  \begin{align}
    \label{eq:blowdown}
    \Bl_{F \cap \td{\mathcal{Z}}}\td{\mathcal{Z}} \to \td{\mathcal{Z}}
  \end{align}
  is an isomorphism.  But, by functoriality of the blow-up, we also
  have a natural induced map
  $\Bl_{F \cap \td{\mathcal{Z}}}\td{\mathcal{Z}} \to \mathcal{Y}$,
  which must be an isomorphism onto its image, because composing with
  $\eta$ yields the blowdown \eqref{eq:blowdown}, which is an
  isomorphism.  Thus, we get our desired lifting.
\end{proof}

The following diagram summarizes the situation so far:

%%%%%%%% diagram
\begin{center}
\begin{tikzcd} & \mathcal{Y} \arrow[d, "\eta"] & & \\ \td{\mathcal{Z}} \arrow[r,
  hook, "i"] \arrow[ur, dotted, hook, "j"] & \td{S^{[2]}} \times S \arrow[r,
  "{(\beta \times \id)}"] \arrow[d, "\td{\pi}_{1}"] & S^{[2]}\times S \arrow[d,
  "\pi_{1}"] \\ & \td{S^{[2]}} \arrow[r, "\beta"] &
  S^{[2]}
\end{tikzcd}
\end{center}

Next, we let $\mathcal{F} \subset \mathcal{Y}$ denote the exceptional
divisor of $\eta$ -- it has $10$ disjoint components, corresponding to
each $F_{i}$, which in turn correspond to the ten lines $L_{i}$.

\begin{proposition}
  \label{prop:geometry}
  Let $e \in E_{i}$ be any point, and let
  $\mathcal{Y}_{e} = (\td{\pi}_{1} \circ \eta)^{-1}(e)$.
  \begin{enumerate}
  \item $\mathcal{Y}_{e}$ is the union of two surfaces,
    $S \cup \mathcal{F}_{e}$, meeting transversely along
    $L_{i} \subset S$.  The surface $\mathcal{F}_{e}$ is isomorphic to
    the Hirzebruch surface $\F_{1}$, and $L_{i} \subset \F_{1}$ is a
    line, i.e. a section with self-intersection $1$.
  \item The identification of $\mathcal{F}_{e}$ with $\F_{1}$ is such
    that $\eta|_{\mathcal{F}_{e}} : \mathcal{F}_{e} \to L_{i}$ is the
    natural projection expressing $\F_{1}$ as a $\P^{1}$-bundle over
    $L_{i}$.
  \item View the length $2$ scheme $\td{\mathcal{Z}}_{e}$ as a closed
    subscheme of $\mathcal{Y}_{e}$ via the inclusion
    $j : \td{\mathcal{Z}} \hookrightarrow \mathcal{Y}$.  Then
    $\td{\mathcal{Z}}_{e} \subset \mathcal{F}_{e}$, and
    $\td{\mathcal{Z}}_{e} \cap S = \emptyset$.

  \item The subscheme $\td{\mathcal{Z}}_{e} \subset \F_{1}$ maps
    isomorphically onto its image in $L_{i}$ under $\eta$.
  \end{enumerate}
\end{proposition}


\begin{proof}
  (1): The claim follows after understanding the geometry of the
  blowup $\mathcal{Y} = \Bl_{F} \left( \td{S^{[2]}} \times S
  \right)$. In order to do this, we must understand the normal bundle
  $N_{F/ \td{S^{[2]}} \times S }$.  It suffices to focus
  on the particular component $F_{i} = E_{i} \times L_{i}$.  Clearly,
  $N_{F_{i}/ \td{S^{[2]}} \times S } = \td{\pi}_{1}^{*}
  N_{E_{i}/\td{S^{[2]}}} \oplus \td{\pi}_{2}^{*}N_{L_{i}/S}$.  Here,
  $\td{\pi}_{2}: E_{i} \times L_{i} \to L_{i}$ denotes the second
  projection.

  The component of $\mathcal{F}$ lying above $F_{i}$ is the
  projectivization $\P N_{F_{i}/ \td{S^{[2]}} \times S }$; restricting
  this to $L_{i} = \{e\} \times L_{i}$ yields
  $\P (\O_{L_{i}} \oplus \O_{L_{i}}(-1))$.  Thus,
  $\mathcal{F}_{e} \simeq \F_{1}$.  Furthermore, since
  $\{e\} \times S \cap F_{i} = \{e\} \times L_{i}$ is a Cartier
  divisor on $S = \{e\} \times S$, the proper transform of $S$ in
  $\mathcal{Y}$ is just $S$ again.  Furthermore, it meets
  $\mathcal{F}_{e} = \P (\O_{L_{i}} \oplus \O_{L_{i}}(-1)) \simeq
  \F_{1}$ along the section corresponding to the summand
  $N_{L_{i}/S} = \O_{L_{i}}(-1)$. Altogether, we get the description of $\mathcal{Y}_{e}$ provided in (1).

  (2): This also follows from the geometry of the blow up described in
  the proof of part (1).

  (3): The lift $j: \td{\mathcal{Z}} \hookrightarrow \mathcal{Y}$ is
  produced by using compatibility of blowups.  This implies that
  \[j^{-1}(\mathcal{F}) = F \cap \td{\mathcal{Z}},\] since the latter
is the exceptional divisor of the blowup (i.e. identity morphism)
$\td{\mathcal{Z}} = \Bl_{F}(\td{\mathcal{Z}}) \to \td{\mathcal{Z}}$.
But
$F \cap \td{\mathcal{Z}} = \td{\pi}_{1}^{-1}(E) \cap
\td{\mathcal{Z}}$.  Restricting attention to a specific point
$e \in E_{i}$ implies
$j(\td{\mathcal{Z}_{e}}) \subset \mathcal{F}_{e}$. Finally consider
the intersection product of cycles:
$[\td{\mathcal{Z}}] \cdot [\mathcal{Y}_{e}]$.  On the one hand, since
$\td{\pi}_{1}: \td{\mathcal{Z}} \to \td{S^{[2]}} \times S$ is finite,
flat and has degree $2$, this intersection product is $2$.  On the
other hand, by part (1), it is equal to
$[\td{\mathcal{Z}}] \cdot ([\mathcal{F}_{e}]+ [S]),$ and since
$\td{\mathcal{Z}}_{e} \subset \mathcal{F}_{e}$, it follows that
$[\td{\mathcal{Z}}] \cdot [S] = 0$, and therefore
$\td{\mathcal{Z}} \cap S = \emptyset$, concluding part (3).

(4): The restriction of $\eta$ to $j(\td{\mathcal{Z}})$ is the
identity map $\td{\mathcal{Z}} \to \td{\mathcal{Z}}$.  Part (4)
  immediately follows.

\end{proof}


Why have we created the variety $\mathcal{Y}$? Recall that we are
trying to extend a family of good cubic surfaces over $U$ across the
divisor $E \subset \td{S^{[2]}}$. If $e \in E_{i}$ is any point, then
by \autoref{prop:geometry}, the fiber $\mathcal{Y}_{e}$ is a
$2$-component surface $S \cup_{L_{i}} \F_{1}$, with the $\F_{1}$
containing a distinquished length $2$ subscheme, namely
$\mathcal{Z}_{e}$. However, there are three further distinguished
points on $\F_{1}$, namely $\tau_{i} \subset L_{i} \subset \F_{1}$.
Therefore, there is now a natural length $5$ subscheme
\[\tau_{i} \cup \td{\mathcal{Z}}_{e} \subset \F_{1}.\]
{\sl The idea now is to extend our good family which lives over $U$ by
  assigning to $e \in E_{i}$ the cubic surface which is the
  anticanonical image of
  $\Bl_{\tau_{i} \cup \td{\mathcal{Z}}_{e}}\F_{1}$.}

First, we must understand how $\td{\mathcal{Z}}_{e}$ is situated
inside $\F_{1}$.

\begin{proposition}
  \label{prop:nodirectrix} Maintain the setting above.  The length two
  subscheme $\td{\mathcal{Z}}_{e} \subset \F_{1}$ is not entirely
  contained in the directrix $D$ of $\F_{1}$.
\end{proposition}

\begin{proof}
  The main idea is to unravel what a point $e \in E_{i}$ represents.
  The point $\beta(e) \in S^{[2]}$ represents a length two subscheme
  of $L_{i}$ -- in fact, it is naturally identified with
  $\td{\mathcal{Z}}_{e}$. Let us first identify the $2$-dimensional
  vector space $N_{\Lambda_{i}/S^{[2]}}\big|_{\beta(e)}$.  We leave it
  to the reader to check that
  $N_{\Lambda_{i}/S^{[2]}} = N_{L_{i}/S}^{[2]} \simeq
  \O_{L_{i}}(-1)^{[2]}$.  So,
  \begin{align}
    \label{eq:N2}
    N_{\Lambda_{i}/S^{[2]}}\big|_{\beta(e)} = H^{0}\left(\td{\mathcal{Z}}_{e}, N_{L_{i}/S}\big|_{\td{\mathcal{Z}}_{e}}\right). 
  \end{align}
  Thus, the point $e \in E_{i}$ corresponds to a $1$-dimensional
  linear subspace, denoted
  \[\langle e \rangle \subset H^{0}\left(\td{\mathcal{Z}}_{e},
      N_{L_{i}/S}\big|_{\td{\mathcal{Z}}_{e}}\right).\]

  In fact, since $N_{E_{i}/\td{S^{[2]}}}$ restricts to $\O(-1)$ on the
  $\P^{1}$ fibers of $\beta: E_{i} \to \Lambda_{i}$, we can and will
  identify this $1$-dimensional subspace with
  $N_{E_{i}/\td{S^{[2]}}}\big|_{e}$.

  Tracing through the construction, the inclusion
  $j: \td{\mathcal{Z}}_{e} \hookrightarrow \mathcal{F}_{e} =
  \P(N_{E_{i}/\td{S^{[2]}}}\big|_{e} \oplus N_{L_{i}/S})$
  (\autoref{prop:geometry}) is induced by the natural inclusion
  \begin{align}
    \label{eq:naturalinclusion}
    \O_{\td{\mathcal{Z}}_{e}} \otimes N_{E_{i}/\td{S^{[2]}}}\big|_{e} \hookrightarrow \left(\O_{\td{\mathcal{Z}}_{e}} \otimes N_{E_{i}/\td{S^{[2]}}}\big|_{e} \right) \oplus N_{L_{i}/S}\big|_{\td{\mathcal{Z}}_{e}}
  \end{align}
  induced by $(\id, \epsilon)$, where $\epsilon$ is obtained by taking
  the composite
  \[N_{E_{i}/\td{S^{[2]}}}\big|_{e} \hookrightarrow
    H^{0}\left(\td{\mathcal{Z}}_{e},
      N_{L_{i}/S}\big|_{\td{\mathcal{Z}}_{e}}\right) \to
    N_{L_{i}/S}\big|_{\td{\mathcal{Z}}_{e}}\] (the latter being
  evaluation) and tensoring with $ \O_{\td{\mathcal{Z}}_{e}}$. Note
  that the latter evaluation map is an isomorphism of $k$-vector
  spaces.

  The proposition is equivalent to the claim that the composition of
  the inclusion \eqref{eq:naturalinclusion} with the projection to the
  factor $ N_{L_{i}/S}\big|_{\td{\mathcal{Z}}_{e}}$ is not identically
  $0$.  It can only be zero if the original $1$-dimensional subspace
  $\langle e \rangle$ was $0$, which is absurd.  The proposition
  follows.
\end{proof}

Next, we define a line bundle on $\mathcal{Y}$ which will induce a
(rational map) to our family of cubic surfaces.   Let
\begin{align}
  \label{eq:linebundle}
  \mathcal{L} :=\eta^{*} \left( \td{\pi}_{2}^{*}\omega_{S}^{-1} \right)(-2\mathcal{F})\otimes \eta^{*}\td{\pi}_{1}^{*}\O_{\td{S^{[2]}}}(E).
\end{align}
Here again, $\td{\pi}_{1}, \td{\pi}_{2}$ are the projections of
$\td{S^{[2]}} \times S$ to respective factors. The specific reason for
choosing this line bundle will be elucidated as we go, but first we
identify the restriction of $\mathcal{L}$ to a fiber
$\mathcal{Y}_{e}$, $e \in E_{i}$.

\begin{proposition}
  \label{prop:LYe}
  Let $e \in E_{i}$ be an arbitrary point, $\mathcal{Y}_{e} = S \cup_{L_{i}} \F_{1}$ as above.  Then
  \begin{enumerate}
  \item $\mathcal{L}|_{S} \simeq \O_{S}(T_{i})$
  \item $\mathcal{L}|_{\F_{1}} \simeq \O_{\F_{1}}(3L_{i}-D)$.
  \end{enumerate}
  Furthermore, the restriction map
  \begin{align}
    \label{eq:restrL}
    H^{0}(\mathcal{Y}_{e}, \mathcal{L}|_{\mathcal{Y}_{e}}) \to H^{0}(\F_{1}, \O_{\F_{1}}(3L_{i}-D))
  \end{align}
  is an isomorphism onto the $6$ dimensional vector space
  $H^{0}\left(\F_{1}, \O_{\F_{1}}(3L_{i}-D) \otimes \mathcal{I}_{\tau_{i}}\right).$
\end{proposition}

Here, $D \subset \F_{1}$ denotes the directrix curve.

\begin{proof}
  The assertion about the restriction map follows from the
  identifications (1) and (2), after noting that $\O_{S}(T_{i})$ has a
  unique global section vanishing simply along the three disjoint
  lines $T_{i} \subset S$.

  By the definition of $\mathcal{L}$ and \autoref{prop:geometry} (1),
  we see that
  $\mathcal{L}|_{S} = \omega_{S}^{-1} \otimes \O_{S}(-2L_{i})$. Now
  since $c_{1}( \omega_{S}^{-1}) = T_{i} + 2L_{i}$, assertion (1) follows.

  Assertion (2) requires us to identify how
  $\O_{\mathcal{Y}}(\mathcal{F})$ restricts to $\mathcal{F}_{e}$. Let
  $\mathcal{S} \subset \mathcal{Y}$ denote the proper transform of
  $E \times S$ in $\mathcal{Y}$; note that
  $\eta: \mathcal{S} \to E \times S$ is an isomorphism.  Then on
  $\mathcal{Y}$ we get an equality of divisors:
  \[\mathcal{F} = \eta^{*}\td{\pi}_{1}^{*}(E) - \mathcal{S}.\]
  Restricting both sides to $\mathcal{F}_{e}$, while using
  \autoref{prop:geometry} part (1), we get:
  \[\O_{\F_{1}}(\mathcal{F}) = \O_{\F_{1}}(-L_{i}).\] Finally, letting
  $R \subset \F_{1}$ denote a ruling line (so $R+D = L_{i}$), we get
  that
  $\eta^{*}\td{\pi}_{2}^{*}(\omega_{S}^{-1})|_{\F_{1}} =
  \O_{\F_{1}}(R)$ because $\omega_{S}^{-1}$ has degree $1$ on $L_{i}$.
  Now, we simply use the definition of $\mathcal{L}$ to deduce
  assertion (2).
\end{proof}

Finally, we set
\begin{align}
  \label{eq:V4}
  \mathcal{W} := (\td{\pi}_{1} \circ \eta)_{*}\left(\mathcal{L} \otimes \mathcal{I}_{\td{\mathcal{Z}}}\right).
\end{align}

By combining \autoref{prop:geometry}, \autoref{prop:nodirectrix}, and
\autoref{prop:LYe}, we leave it to the reader to check that:

\begin{enumerate}\label{propertieskappa}
\item $\mathcal{W}$ is a rank $4$ vector bundle on $\td{S^{[2]}}$, and
\item for each $e \in E_{i}$, the base scheme of
  $\mathcal{W}|_{e} \subset
  H^{0}\left(\mathcal{Y}_{e},\mathcal{L}_{\mathcal{Y}_{e}}\right)$ is
  the scheme $T_{i} \cup \td{\mathcal{Z}}_{e}$.
\end{enumerate}

The natural evaluation map
$(\td{\pi}_{1} \circ \eta)^{*}\mathcal{W} \to \mathcal{L}$ on
$\mathcal{Y}$ induces a rational map
\begin{align}
  \label{eq:kappa4}
  \kappa: \mathcal{Y} \dashrightarrow \P \mathcal{W}.
\end{align}

The image of $\kappa$, by which we mean the closure of the image of
the locus on which $\kappa$ is defined, will be denoted
\[\mathcal{X} \subset \P \mathcal{W}, \] and $\pi: \mathcal{X} \to \td{S^{[2]}}$ the natural map. By \autoref{prop:LYe}, if
$e \in E_{i}$ is any point, the surface $\mathcal{X}_{e}$ is the
anti-canonical image of the surface
$\Bl_{\td{\mathcal{Z}}_{e} \cup \tau_{i}} \F_{1}$.  Note that
$\mathcal{X}_{e}$ is singular, because the proper transform of $L_{i}$
is a contracted $(-2)$-curve. We leave it to the reader to verify the
following, using \autoref{prop:geometry} (3) and (4), and
\autoref{prop:nodirectrix}:

\begin{proposition}
  \label{prop:good4} The family $\pi: \mathcal{X} \to \td{S^{[2]}}$ is
  good.
\end{proposition}

\begin{proof}
  Omitted.
\end{proof}

The attentive reader will ask: what role does the twist by
$\O_{\td{S^{[2]}}}(E)$ in the definition of $\mathcal{L}$
\eqref{eq:linebundle} play? (It has played no role in the analysis up
to this point.)  The next proposition explains this:

\begin{proposition}
  \label{prop:whyE} Maintain the setting above.  Then
  $\mathcal{W} = \pi_{*}\left( \omega_{\pi}^{-1} \right)$,
  i.e. $\mathcal{W} = \V$.
\end{proposition}

\begin{proof}
  The indeterminacy locus of $\kappa$ is a codimension $2$ subscheme
  of $\mathcal{Y}$, therefore, it makes sense to pull back line
  bundles under $\kappa$.  The content of the proposition is to show that:
  \[ \mathcal{L} = \kappa^{*}\left( \omega_{\pi}^{-1} \right).\]

  To see this, first note that these two line bundles agree over the
  open set $U \times S \subset \mathcal{Y}$.  The complement
  $\mathcal{Y} \setminus U \times S$ is the union of two divisors,
  $\mathcal{F}$ and $\mathcal{S}$ -- see the proof of
  \autoref{prop:LYe}.  (The pullback of $E$ to $\mathcal{Y}$ is
  $\mathcal{F} + \mathcal{S}$.)

   We conclude there is a relationship of the form
   \[c_{1} \mathcal{L} - c_{1} \left( \kappa^{*}\left(
         \omega_{\pi}^{-1} \right) \right) = a\mathcal{F} +
     b\mathcal{S}.\] We must show that $a = b = 0$.  To do so, we will
   restrict this relation to certain subschemes of $\mathcal{Y}$.

  First, choose any point $e \in E_{i}$, and restrict the relation to
  the surface $\mathcal{F}_{e} \simeq \F_{1}$.  The left hand side
  restricts to $\O_{\F_{1}}$.  The right side restricts to
  $\O\left( (b-a)L_{i} \right)$.  Thus, $a=b$.

  Thus,
  $c_{1} \mathcal{L} - c_{1} \left( \kappa^{*}\left( \omega_{\pi}^{-1}
    \right) \right)$ is the pullback of $\O_{\td{S^{[2]}}}(aE_{i})$
  for some integer $a$, and we must prove that $a = 0$.  In order to
  do this, we will restrict to a different surface.  First, choose a
  general fiber $P \subset E_{i}$ of the blow-down map
  $\beta: E_{i} \to \Lambda_{i}$; $P \simeq \P^{1}$.  Let
  $\mathcal{D} \subset \mathcal{Y}$ denote the surface which is traced
  out by the directrices of $\mathcal{F}_{e}$ as $e$ varies in $P$.
  Note that $\mathcal{D} \simeq P \times L_{i}$.  $\mathcal{D}$ is a
  divisor in the threefold
  $\mathcal{F}_{P} := \mathcal{F} \cap (\td{\pi_{1}} \circ
  \eta)^{-1}(P)$; the map $\eta$ expresses $\mathcal{F}_{P}$ as a
  $\P^{1}$-bundle over $P \times L$, namely the projectivization
  $\P \left( \O_{P}(-1) \oplus \O_{L_{i}}(-1) \right).$ (Here and in
  what follows, we will suppress pull-back notation when we pull line
  bundles back along the two projection maps
  $P \times L_{i} \to L_{i}$.)

  With regard to the $\P^{1}$-bundle structure of
  $\eta: \mathcal{F}_{P} \to P \times L_{i}$, the divisor
  $\mathcal{D} \subset \mathcal{F}_{P}$ is a section, corresponding to
  the inclusion of the first factor
  \[\O_{P}(-1) \hookrightarrow \O_{P}(-1) \oplus \O_{L_{i}}(-1). \]

  Therefore, letting $\O_{\mathcal{F}}(1)$ denote the natural line
  bundle of the projective bundle $\mathcal{F}$, this translates into
  $\O_{\mathcal{F}}(-1)|_{\mathcal{D}} = \O_{P}(-1)$. (Here we are
  identifying $\mathcal{D}$ with $P \times L_{i}$ in the natural way.)
  As is generally true about blow-ups, we know that
  $\O_{\mathcal{Y}}(\mathcal{F})\big|_{\mathcal{F}} =
  \O_{\mathcal{F}}(-1)$, and therefore,
  \begin{align}\label{eq:FtoD}
    \O_{\mathcal{Y}}(\mathcal{F}) \big|_{\mathcal{D}} = \O_{P}(-1).
  \end{align}

  Now we can identify $\mathcal{L}|_{\mathcal{D}}$: The three factors
  of $\mathcal{L}$ are (suppressing pullbacks):
  $\omega_{S}^{-1}, \O_{\mathcal{Y}}(\mathcal{F}),$ and $\O(E)$. These
  three restrict to $\mathcal{D}$ as: $\O_{L_{i}}(1), \O_{P}(-1),$ and
  $ \O_{P}(-1)$ respectively, thanks to \eqref{eq:FtoD}, and the fact
  that $\O_{E}(E) = \O_{E}(-1)$.  Thus,
  \begin{align}
    \label{eq:LtoD}
    \mathcal{L}|_{\mathcal{D}} = \O_{P}(1) \otimes \O_{L_{i}}(1).
  \end{align}

  Next, we identify $\kappa^{*}(\omega_{\pi})|_{\mathcal{D}}$.
  Restricting attention over $P$, the map
  $\kappa: \mathcal{F}_{P} \dashrightarrow \mathcal{X}_{P}$ is still
  defined outside a codimension $2$ subscheme of $\mathcal{F}_{P}$.
  The pullback of forms map
  $\kappa^{*}(\omega_{\mathcal{X}_{P}/P}) \to
  \omega_{\mathcal{F}_{P}/P}$ is, fiber by fiber over $P$, an
  isomorphism outside a set of codimension $2$, therefore,
  $\kappa^* \left( \omega_{\pi} \right) = \omega_{\mathcal{F}_{P}/P}.$
  The latter is
  $\omega_{\mathcal{F}_{P}/P \times L_{i}} + \omega_{P \times
    L_{i}/P}$, which, using the Euler exact sequence, is:
  $\O_{P}(-1) \otimes \O_{L_{i}}(-1)$.  This is exactly the opposite
  of $\mathcal{L}\big|_{\mathcal{D}}$, which is what we wanted to
  show.

  
\end{proof} 


\subsubsection{Chern classes of $\V$}
\label{sec:chern-classes-v4}

Now we move to the task of calculating the Chern classes of $\V$. 






\subsubsection{Relation $B_4$}
\label{sec:relation-b_4}

The relation we get from family $B_3$ ultimately appears to be:
\begin{align}
  \label{eq:relationB4}
  6a_{1^{4}} + 21a_{1^{2}2}+6a_{13}+16a_{2^{2}}+a_{4} = 36 \times 6! = 25920.
\end{align}

(There is an 80 percent chance I made a mistake in the long
calculation, so be careful.)

\appendix

\section{The goodness of the $A5$ cubic surface}
\label{sec:goodnessA5}




\bibliographystyle{plain}
\bibliography{references.bib}
\end{document}

%%% Local Variables:
%%% mode: latex
%%% TeX-master: t
%%% End:

