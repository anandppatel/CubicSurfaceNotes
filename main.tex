\documentclass[12 pt]{amsart}

%============================
%         Obligatory
%============================
\usepackage{mathtools}
\usepackage{amssymb}
\usepackage{amsmath}
\usepackage{amsthm}
\usepackage{mathrsfs}
\usepackage{tikz}
\usepackage{array}

%============================
%         Aesthetic
%============================
\usepackage[hidelinks]{hyperref} % For citations without hideous colored boxes
\usepackage{geometry} % For adjusting the page domensions if desired
%\usepackage{sectsty} % For centering Section/Subsection titles
\usepackage{marginnote} % Currently unused, but for notes in the margin
\usepackage{graphics} % For custom pretty fractions 
\usepackage{newtxmath} % Changes the math font to be bolder and prettier

\geometry{margin=1.66 in}
%\sectionfont{\centering}
%\subsectionfont{\centering}
\linespread{1.1}


\title{Counting Cubic Surfaces}
\author{Anand Deopurkar \and Anand Patel \and Dennis Tseng}


\newtheorem{theorem}{Theorem}[section]
\newtheorem{lemma}{Lemma}[section]
\newtheorem{proposition}{Proposition}[section]
\newcommand{\propositionautorefname}{Proposition}
\newtheorem{corollary}{Corollary}[section]
\newcommand{\corollaryautorefname}{Corollary}
\newtheorem{conjecture}{Conjecture}[section]
\newcommand{\conjectureautorefname}{Conjecture}
\newtheorem{definition}{Definition}[section]
\newcommand{\definitionautorefname}{Definition}

\newenvironment{example}{\textbf{Example:}}{}
\newenvironment{remark}{\textbf{Remark:}}{}

% chk is a stylized "dual" symbol created by altering the vee command 
\newcommand{\chk}[1]{#1^{\text{\scalebox{.8}[.6]{$\boldsymbol{\vee}$}}}}

% altfrac and altfraci are twin commands for alternate fractions. The difference between them is that altfrac is bogger and prettier, while altfraci fits in inline mode. In the future I plan to merge them into one command which checks the displaystyle and uses the appropriate form of the command. Note that this command is specialized to the newtxmath package, and without it the fraction shape could be distorted. This can be solved by adjusting the different numbers in the command.
\newcommand{\altfrac}[2]{\ifmmode\def\tmp{$}\else\def\tmp{}\fi
		\mbox{{\raisebox{.5\ht\strutbox}{\tmp#1\tmp}}
		\kern-5pt\scalebox{1.75}[1.5]{/}\kern-1.5pt
		\mbox{\raisebox{-.2\ht\strutbox}{\tmp#2\tmp}}}
}

\newcommand{\altfraci}[2]{\ifmmode\def\tmp{$}\else\def\tmp{}\fi
		\scalebox{.925}[.925]{{\mbox{{\raisebox{.25\ht\strutbox}{\tmp#1\tmp}}
		\kern-5pt\scalebox{1.5}[1.3]{\raisebox{-.1\ht\strutbox}{/}}\kern-0.5pt
		\mbox{\raisebox{-.15\ht\strutbox}{\tmp#2\tmp}}}}}
}


\renewcommand{\P}{\mathbb{P}}
\newcommand{\SL}{\operatorname{SL}}
\newcommand{\T}{\mathbb{T}}
\newcommand{\e}{\varepsilon}
\newcommand{\<}{\left\langle}
\renewcommand{\>}{\right\rangle}

\newcommand{\um}[1]{\color{blue}{#1}\color{black}}
\newcommand{\remove}[1]{\color{red}{#1}\color{black}}
\newcommand{\add}[1]{\color{green}{#1}\color{black}}

\newcommand{\keps}{\altfrac{k[\e]}{\left(\e^2\right)}}
\newcommand{\kepsi}{\altfraci{k[\e]}{\left(\e^2\right)}}
\newcommand{\mf}[1]{\mathfrak{#1}}

\DeclareMathOperator{\Bl}{Bl}
\DeclareMathOperator{\Sec}{Sec}
\DeclareMathOperator{\spec}{Spec}
\DeclareMathOperator{\proj}{Proj}
\DeclareMathOperator{\res}{res}
\DeclareMathOperator{\sym}{Sym}

\newcommand{\M}{\overline{M}_{0,7}}

\begin{document}

\maketitle


\section{Introduction}
\label{sec:intro}
Our primary objective in this paper is to determine an enumerative
formula which counts the number of times a general cubic surface
occurs in a (suitable) $4$-parameter family $ \pi: X \to B$ of cubic
surfaces.  We work over an algebraically closed, characteristic zero
ground field $k$.  Applying the formula to some natural families
yields:

\begin{theorem}
  \begin{itemize}
  \item[(A)] A general abstract cubic surface can be interpolated
    through $15$ general points in $\P^3$ in exactly $96120$ distinct
    ways.
  \item[(B)] Let $X \subset \P^{4}$ be a general cubic
    threefold. A general abstract cubic surface arises $42720$ times
    as a hyperplane section of $X$.
    \item[(C)]... 
    \end{itemize}
\end{theorem}

The first computation above provides a verification of \cite{}, which
used the numerical technique of homotopy continuation.

Before providing the formula, we clarify which families of
cubic surfaces it applies to:

\begin{definition}
\label{def:goodfamily}
Let $\pi :X \to B$ be a flat, projective morphism of relative
dimension $2$, with $B$ a $4$-dimensional scheme. We say $\pi$ is {\bf
  a good family of cubic surfaces} if for all $b \in B$, the fiber
$X_{b}$ is isomorphic to a cubic surface which is not an isotrivial
specialization of a general cubic surface.
\end{definition}

The base $B$ of a good family carries its {\sl natural} rank $4$
vector bundle
\begin{align}
  \label{eq:V}
  V := \pi_{*}\omega_{X/B}^{-1}.
\end{align}
Furthermore, assuming the general fiber of $\pi$ is smooth, $B$ admits
a rational map
\begin{align}
  \label{eq:mu}
  \mu: B \dashrightarrow M
\end{align}
to the coarse moduli variety $M$ of cubic surfaces.


\begin{theorem}\label{theorem:main}
  Let $\pi: X \to B$ be a good family of cubic surfaces with $B$
  proper and smooth generic fiber. If $v_{i}, i=1, \dots 4$ denote the
  Chern classes of $V = \pi_{*}(\omega_{\pi}^{-1})$, then
  \begin{align}
    \label{eq:MAIN}
   \deg( 1080v_{1}^{2}v_{2} - 1080v_{1}v_{3}+9720v_{4}) = \deg \mu,
  \end{align}
  where $\mu: B \dashrightarrow M$ is the induced map to the moduli
  space of cubic surfaces.
\end{theorem}


Our method of proof is to first argue that there exists a formula of the form

\begin{align}
  \label{eq:P}
  \deg (a_{1^4}v_{1}^{4} + a_{1^{2}2}v_{1}^{2}v_{2} + a_{13}v_{1}v_{3} + a_{2^2}v_{2}^{2} + a_{4}v_{4}) = \deg \mu,
\end{align}
as stated in \autoref{theorem:main}.  Then, we determine the five coefficients by exhibiting particular good
families for which we know all $v_{i}$ and $\deg \mu$.  We will
actually exhibit nine(?) such families, thereby increasing the
confidence in the correctness of our formula.

Each section focuses on one particular construction of a good
family.  After each section, we reset
notation.

\section{Family $B_1$}
\label{sec:first-test-family}

In this section, we let $S \subset \P^7$ denote a degree $7$ del Pezzo
surface. We let $E_{1}, E_{2}$ denote the two disjoint lines on $S$
and we let $L \subset S$ denote the third line on $S$, which meets
both $E_{1}$ and $E_{2}$. Finally, we choose a general rational normal
quintic curve $C \subset S$ meeting $E_{2}$ at a single point other
than $L \cap E_{2}$.

\subsection{Construction of $B_1$}
\label{sec:construction-b_1}


The base of our first family is
\begin{align}
  \label{eq:B1}
  B_1 := C^{[4]} \simeq \P^{4}.
\end{align}
$B_1$ carries a good family of cubic surfaces over it.  To see this,
we note that any length four subscheme $Z \subset C$ satisfies the
property that the three dimensional span $\langle Z \rangle$
intersects $S$ at precisely the scheme $Z$. Therefore, for each such
$Z$, the linear projection from $\langle Z \rangle$
$p_{\langle Z \rangle}: S \dashrightarrow \P^{3}$ has as image an
irreducible cubic surface.  Furthermore, one can check that each of
these cubics has a finite automorphism group, and hence we have a good
family.

\subsection{Chern classes on $B_1$}
\label{sec:chern-classes-b_1}

\begin{lemma}
  \label{lemma:chernB1}
  For the natural bundle $V$ over $B_1$, we have:
  \begin{align}\nonumber
    v_{1}^{4} = 16,\\\nonumber
    v_{1}^{2}v_{2} = 4,\\\nonumber
    v_{1}v_{3} =0,\\\nonumber
    v_2^{2} = 1,\\\nonumber
    v_{4} = 0.
  \end{align}
\end{lemma}

\begin{proof}
The Chern classes $v_{1} = -2h$, $v_{2}=h^2$, and $v_{3}= v_{4}=0$,
where $h$ denotes the hyperplane class on the projective space $B_1$.)
  
\end{proof}

\subsection{Enumerating cubics in $B_1$}
\label{sec:enum-cubics-b_1}


For $B_1$ to be of any use, we must {\sl a priori} know
$\deg P(v_{1},v_{2},v_{3},v_{4})$ for $B_1$. This is easily
achievable, given basic facts about the configuration of lines on a
general cubic surface. We let
\begin{align}
  \label{eq:nu}
  \nu: M^{\dagger} \to M
  \end{align}
  denote the variety of smooth {\sl marked} cubic surfaces:
  $M^{\dagger}$ parametrizes tuples $(X, E_{1}, \dots, E_{6})$ where
  the $E_{i}$ are six disjoint lines on $X$. It is a classical fact
  that $\deg \nu = 6! \times 72$. The symmetric group $\Sigma_6$ acts
  on $M^{\dagger}$ by permuting the marked lines.

  The natural rational map to moduli $\mu : B_1 \dashrightarrow M$
  clearly factors through a map
  $\mu^{\dagger}: B_1 \dashrightarrow M^{\dagger}/\Sigma_{4}$, where
  $\Sigma_{4}$ acts by permuting the last four lines
  $E_{3}, \dots, E_{6}$.

  \begin{lemma}
    \label{lemma:degreemudaggerB1}
    $\mu^{\dagger}: B_1 \dashrightarrow M^{\dagger}$ has degree $2$.
  \end{lemma}

Assuming this fact, we arrive at the first relation:

\begin{align}
  \label{eq:relation1}
  16 \cdot a_{1^4} + 4 \cdot a_{1^2 2} + a_{2^2} = 6! \times 3 = 4320.
\end{align}

\section{Family $B_2$}
\label{sec:family-b_2}


Following Anand D's idea, we construct a good family over the base
$B_2 = \overline{M}_{0,7}$. Before continuing, let us summarize the
construction. Let $C \subset \P^{2}$ be the $2$-Veronese image of
$\P^{1}$ with $7$ distinct marked points
$x_{0}, x_{1}, \dots, x_{5}, x_{\infty}$ on $C$. Let
$q \in \P^{2} \setminus C$ denote the intersection point of the
tangent lines of $C$ at $x_{0}$ and $x_{\infty}$.  Then the blow up of
$\P^{2}$ at the six points $\{x_{1}, \dots, x_{5}, q\}$ produces a
good cubic surface.  In this way, we have described a good family of
cubic surfaces over $M_{0,7}$.  Anand D's point is: this construction
can be naturally extended to a good family over $\M$.




Let us denote a general stable
$7$-pointed genus zero curve by
$$(C, p_{0}, p_{1}, \dots, p_{5}, p_{\infty}).$$  Denote
by $\mathbf{L}_{\infty}$ the cotangent line bundle on $\M$
corresponding to the marked point $p_{\infty}$. If
\begin{align}
  \label{eq:PM07}
  \pi: C \to \overline{M}_{0,7}
\end{align}
denotes the universal stable curve, we let
$\sigma_{0}, \dots, \sigma_{\infty}$ denote the universal sections in
$C$.

Next, let $C' = \proj \pi_{*}\mathcal{O}_{C}(\sigma_{\infty})$ denote
the $\P^{1}$-bundle over $\overline{M}_{0,7}$. Let $O_{C'}(1)$ denote
the natural hyperplane line bundle.  Then the sections $\sigma_{i}$
also produce sections (with the same name) of the bundle projection
$$\gamma : C' \to \M,$$ with
$\sigma_{0}^{*}O_{C'}(1) = \dots = \sigma_{5}^{*}O_{C'}(1) = O_{\M}$
and $\sigma_{\infty}^{*}O_{C'}(1) = \mathbf{L}^{-1}$.

Because $\sigma_{\infty}$ is disjoint from all other sections, it
follows that
$$\pi_{*}(O_{C}(\sigma_{\infty})) = \mathbf{L}^{-1} \oplus O_{\M}.$$


The next step is to invoke the relative $2$-veronese
embedding
$$\iota: C' \hookrightarrow P := \proj
\pi_{*}O_{C}(2\sigma_{\infty})$$ so $C' \subset P$ is a relative
family of smooth conics with $7$ marked points
$\sigma_{0}, \dots, \sigma_{\infty}$.  Importantly, $\sigma_{\infty}$
never intersects any of the other six sections.  Also, observe that
$$\pi_{*}O_{C}(2 \sigma_{\infty}) = \sym^{2}(\mathbf{L}^{-1} \oplus
O) = \mathbf{L}^{-2} \oplus \mathbf{L}^{-1} \oplus O.$$



To be continued...




\subsection{OLD $B_2$}


Our next family is a slight variant of $B_1$. Let $S$ be a degree $7$
del Pezzo surface, viewed as the blow up of the plane at two points,
with its two exceptional lines $E_1, E_2$ and its third exceptional
line $L$.  Next, fix a general conic $C \subset S$ which is {\sl
  tangent} to $L$ at a point $p$ distinct from the pair
$L \cap (E_1 \cup E_2)$.

The configuration space $C^{[4]} \simeq \P^{4}$ carries a family of
cubic surfaces $X/\P^{4}$.  Unfortunately, the configurations of the
form $x+y+2p$, $x,y \in C$, register reducible cubic surfaces, and so
$X/\P^{4}$ is not good.  This locus is a $2$-plane
$\Lambda \subset \P^{4}$.

\begin{proposition}
  \label{proposition:resolveB2} The blow up of $\P^{4}$ along
  $\Lambda$ carries a good family extending
  $X/(\P^{4} \setminus \Lambda)$.
\end{proposition}
(Actually, hold off: One needs to blow up at least one more time,
corresponding to the configuration $4p$.)



Next we must compute the degree of the map to moduli
$\mu: B_2 \dashrightarrow M$. As in the case of $B_1$, $\mu$ factors
through $\mu^{\dagger}: B_2 \dashrightarrow M^\dagger/\Sigma_{4}$.
However,  there is a distinction:

\begin{proposition}
  \label{proposition:deg-mudagger2}
  We have: $\deg \mu^{\dagger} = 2$.
\end{proposition}


\section{Family $B_3$}
\label{sec:family-b_3}

We construct our third family $B_3$ be starting with a quintic del
Pezzo surface $S$, which is the blow up of the plane at four general
points, and then ``blowing up all pairs of points on $S$.''

The surface $S$ has $10$ exceptional lines $L_{1}, \dots, L_{10}$.  We
let $S^{[2]}$ denote the Hilbert scheme of pairs of points on $S$.
$S^{[2]}$ contains $10$ disjoint $\P^{2}$'s,
$\Lambda_1, \dots, \Lambda_{10}$, namely the Hilbert schemes of pairs
of points contained in the lines $L_{i}$. Let
$\Lambda = \cup_{i}\Lambda_{i}$.

\subsection{Construction of $B_3$}
\label{sec:construction-b_3}

Let $Z \subset S^{[2]} \times S$ denote the universal degree two
subscheme; $Z$ is smooth and the first projection
$\pi_1 : Z \to S^{[2]}$ is finite, flat, and has degree two.



First, observe that the open set
\begin{align}
  \label{eq:UB3}
  U := S^{[2]} \setminus \Lambda
\end{align}
supports a good family of cubic surfaces, namely
\begin{align}
  \label{eq:XUB3}
  X^{\circ} := \Bl_{Z}(S^{[2]} \times S).
\end{align}

(We encourage the reader to check that $X^{\circ}/U$ is indeed good.)

However, when we extend $X^{\circ}$ in the natural way over $S^{[2]}$
the family acquires bad fibers, namely unions of quadrics and planes,
and therefore we must modify $S^{[2]}$ along $\Lambda$.

Let
\begin{align}
  \label{eq:S2tilde}
  \widetilde{S^{[2]}} := \Bl_{\Lambda}S^{[2]}
\end{align}
denote the blow-up with blow-down map
$\beta: \widetilde{S^{[2]}} \to S^{[2]}$.  Let
$E_{i} = \beta^{-1}(\Lambda_{i})$, $i=1, \dots, 10$ denote the
components of the of the exceptional divisor of the blow-up; $\beta$
expresses $E_{i}$ as a $\P^{1}$-bundle over $\Lambda_{i}$.

Next, let $\widetilde{Z} := \widetilde{S^{[2]}} \times_{S^{[2]}} Z$
denote the fiber-product; $\widetilde{Z}$ is a closed subscheme of the
product $\widetilde{S^{[2]}} \times S$, and is a finite, flat, degree
two cover of $\widetilde{S^{[2]}}$ under the first projection
$\tilde{\pi}_{1} : \widetilde{Z} \to \widetilde{S^{[2]}}.$

Inside the product $\widetilde{S^{[2]}} \times S$ we define the
smooth, closed subschemes
\begin{align}
  \label{eq:Fi}
  F_{i} := E_{i} \times L_{i};
\end{align}
we let $F := \cup_{i}F_{i}$.  We perform one more blow-up to get 
\begin{align}
  \label{eq:Xtilde}
  Y := \Bl_{F}\widetilde{S^{[2]}} \times S.
\end{align}
The natural projection $ Y \to \widetilde{S^{[2]}}$ is a family of
surfaces with simple normal crossing singular fibers over $E_{i}$.
The next lemma is critical.

\begin{lemma}
  The closed immersion
  $\widetilde{Z} \subset \widetilde{S^{[2]}} \times S$ lifts to a
  closed immersion $\widetilde{Z} \subset Y$.
\end{lemma}

\begin{proof}
  The main observation is that we have an equality of schemes
  \begin{align}
    \label{eq:Cartier}
    F_i \cap \widetilde{Z} = \widetilde{\pi}_{1}^{-1}(E_i) \cap \widetilde{Z}
  \end{align}
  for all $i=1, \dots, 10$, {\sl and therefore $F \cap \widetilde{Z}$
    is a Cartier divisor on $\widetilde{Z}$}. The claim now follows
  from a basic property of blow-ups.
\end{proof}



We are close to describing a family of cubic surfaces over
$\widetilde{S^{[2]}}$.  We first let
\begin{align}
  \label{eq:XB3}
  Y' := \Bl_{\widetilde{Z}}Y
\end{align}
denote the blow-up,
\begin{align}
  \label{eq:phiB3}
  \varphi: Y' \to Y
\end{align}
the blow-down map, and
\begin{align}
  \label{eq:piB3}
  \pi: Y' \to \widetilde{S^{[2]}}
\end{align}
the natural projection. Finally, let $F' \subset Y'$ denote the
exceptional divisor $\varphi^{-1}(F)$.
\begin{proposition}
  \label{proposition:resolv-b3}
  There exists a line bundle $\mathcal{L}$ on $Y'$ which induces a
  rational map
  \begin{align*}
    \rho: Y' \dashrightarrow \P(\pi_{*}\mathcal{L})
  \end{align*}
  whose image $X \subset \P(\pi_{*}\mathcal{L})$ is a good family of
  cubic surfaces over $\widetilde{S^{[2]}}$ with natural bundle
  $V = \pi_{*} \mathcal{L}$.
\end{proposition}

(By image, we mean ``closure of image''.)

(This one I still am confident in -- one simple blow-up resolves the situation.)


\subsection{Relation $B_3$}
\label{sec:relation-b_3}

The relation we get from family $B_3$ ultimately appears to be:
\begin{align}
  \label{eq:relationB3}
  6a_{1^{4}} + 21a_{1^{2}2}+6a_{13}+16a_{2^{2}}+a_{4} = 36 \times 6! = 25920.
\end{align}

(There is an 80 percent chance I made a mistake in the long
calculation, so be careful.)

\section{Family $B_4$}
\label{sec:family-b_4}

For this family, we fix a degree $6$ del Pezzo surface $S$, the blow
up of the plane at three non-collinear points, and then we fix a
general ``line'' (which is really a twisted cubic) $L$ in $S$ disjoint
from the three points blown up.  An open subset $U$ of the variety
$L \times L \times S$ carries a good family of cubic surfaces in the
obvious way. However, as in the case of $B_3$, the complement of $U$
is a union of surfaces which need to be blown up before obtaining a
good family of cubic surfaces.


\section{Family $B_5$}
\label{sec:family-b_5-1}

This family is similar to $B_1$. Fix a degree $7$ del Pezzo surface
$S$, and let $D \subset S$ denote a general hyperplane section; $D$ is
a degree $7$ elliptic normal curve.  Next observe that any degree four
subscheme $Z \subset D$ imposes independent conditions on sections of
$O_{S}(1)$, and that $\langle Z \rangle \cap S = Z$.  Furthermore, for
each $Z$, the projection map
$p_{\langle Z \rangle}: S \dashrightarrow \P^{3}$ has as image an
irreducible cubic, with finite automorphism group (check this!).


Therefore, the Hilbert scheme $B_{5} = D^{[4]}$ carries a good family
of cubic surfaces.  (The reason I like this family is that, unlike
$B_1$'s case, the natural bundle $V$ has more non-trivial chern
classes. Therefore, the relation we get this time is probably not the
same as for $B_1$.

\subsection{Difficulty with $B_5$}
\label{sec:difficulty-with-b_5}

The main obstacle in using this simple family is: I do not a priori
know the cubic surface count!!!

\section{Isotrivial family of surface with 3 $A_2$ singularities}
\label{sec:isotrivial3A2}

We outline the idea and give references, in order to fill in the computation relating to \eqref{eq:equality} later. 

Our starting point is the following fact:
\begin{proposition}
\label{prop:3A2}
The cubic surface defined by $X^3 + YZW = 0$ has singularity type $3A_2$ and is semistable.
\end{proposition}

\begin{remark}
\cite[Proposition 2.2]{N05} shows that the most special singularity types that correspond to semistable surfaces are $3A_2$ and $4A_1$. 
\end{remark}

\begin{proof}
The fact that $X^3 + YZW=0$ has singularity type $3A_2$ follows from \cite[Proof of 3.1.12]{N00}. Semistability follows from \cite[Proposition 2.3]{N05}.
\end{proof}

Given \autoref{prop:3A2}, we summarize the two facts about the $3A_2$ cubic surface that makes this family work:
\begin{proposition}
\label{prop:twofacts}
The cubic surface defined by $X^3 + YZW = 0$ has a nontrivial torus stabilizer and is not in the orbit closure of any stable cubic surface.
\end{proposition}

\begin{remark}
I think that this cubic surface is the most special surface that has a nontrivial torus stabilizer but is still semistable, which is means this method of finding a twisted isotrivial family of surfaces missing the general orbit closure only works for exactly this $3A_2$ surface. 
\end{remark}

\begin{proof}
The fact that the $3A_2$ cubic surface does not lie in the general cubic surface orbit closure follows for GIT. Indeed, we have the map $\pi: \mathbb{P}^{19}\dashrightarrow \mathbb{P}^{19}//PGL_4$ and by definition of semistability, the point defined by $X^3 + YZW = 0$ is not contained in its indeterminacy locus. After removing the indeterminancy locus, orbits of stable points are fibers of $\pi$, so they cannot contain the point defined by $X^3 + YZW = 0$. {\color{red} add GIT references to make sure}
\end{proof}

Finally, we obtain our test family:
\begin{proposition}
\label{prop:isotrivialfamily}
Given a base $B$ and line bundles $L_X, L_Y, L_Z, L_W$ such that $L_X^{\otimes 3}\cong L_Y\otimes L_Z\otimes L_W$, we can form the projective bundle $\mathbb{P}(L_X\oplus L_Y\oplus L_Z\oplus L_W)$. Inside of $\mathbb{P}(L_X\oplus L_Y\oplus L_Z\oplus L_W)$ we have the cubic surface $\mathcal{X}$ defined by $\{X^3 + YZW = 0\}$ over each trivialization of $L_X, L_Y, L_Z, L_W$ over $B$. Because $L_X^{\otimes 3}\cong L_Y\otimes L_Z\otimes L_W$, the cubic surface glues compatibly under transition maps.

The family $\mathcal{X}\to B$ is a nontrivial family of cubic surfaces where no member is a contained in the orbit closure of a stable cubic surface.
\end{proposition}

Thus, \autoref{prop:isotrivialfamily} affords an enumerative problem whose answer is zero. We worked with a general base $B$, but it suffices for our purposes to let $B=\mathbb{P}^4$. 

{\color{red} I suspect we can relate this to a collection of 6 points on $P^2$ with a stabilizer, so this family looks more like Anand's many families. From \cite[Proposition 3.3]{N05}, it looks like the length 6 scheme corresponding to the $3A_2$ surface is the intersection of the union of two lines with a triple line. Don't see it yet, but writing this down to not forget.}

\subsection{Isotrivial Relation}
\label{sec:isotrivial-relation}

The relation we obtain from Dennis's trick appears to be:

\begin{align}
  \label{eq:relationDennis}
  a_{2^{2}} = 0.
\end{align}

\section{More isotrivial families}
In this section, we exhibit a number of good isotrivial families.
All of these families will arise from a cubic surface $X$ with automorphism group $\mathbb G_m$.
To make sure that these families are good, we must prove that $X$ does not lie in the closure of the orbit of a generic cubic surface.
If $X$ is a semi-stable point of the $\SL_4$ action on the space of cubics $\P^{19}$, then this follows from GIT.
We now explain an extension of this idea that allows for even more examples.

Let $V$ be a 4-dimensional vector space.
Consider the natural action of $\SL(V)$ on $\P(\sym^3V) \times \P V$ (here $\P$ denotes the space of one-dimensional subspaces).
That is, consider the action of $\SL_4$ on the set of pairs $(X, H)$, where $X \subset \P^3$ is a cubic surface and $H \subset \P^3$ is a hyperplane.
For a pair of positive integers $a, b$, the action lifts uniquely to an action on the ample line bundle $\mathcal O(a)\boxtimes \mathcal O(b)$; the GIT of the resulting linearized action depends only on the ratio $t = a/b$.

\begin{proposition}\label{prop:nolimit}
  Let $t$ be a positive rational number.
  Suppose $S \subset \P^3$ is a cubic surface such that for every hyperplane $H$, the pair $(S,H)$ is $t$-stable.
  Let $X \subset \P^3$ be a cubic surface with stabilizer group $\mathbb G_m \subset \SL_4$ such that for some $\mathbb G_m$-fixed hyperplane $H \subset \mathbb P^3$, the pair $(X, H)$ is $t$-semistable.
  Then $X$ does not lie in the closure of the $\SL_4$-orbit of $S$.
\end{proposition}
\begin{proof}
  Suppose $X$ lies in the orbit closure of $S$.
  Then there exists a one-parameter subgroup $\alpha \colon \mathbb G_m \to \SL_4$ and a cubic surface $Y \subset \P^3$ such that $Y$ is in the $\SL_4$-orbit of $S$ and
  \[ \lim_{t \to 0} \alpha_t(Y) = X.\]
  Since the stabilizer of $X$ is assumed to be a particular $\mathbb G_m \subset \SL_4$, the map $\alpha$ must factor through this $\mathbb G_m \subset \SL_4$.
  Consider the $\alpha$-fixed hyperplane $H \subset \P^3$ such that $(X, H)$ is $t$-semistable.
  We have
  \[ \lim_{t \to 0} \alpha_t(Y, H) = (X, H).\]
  This contradicts that $(Y, H)$ is $t$-stable.
\end{proof}
\begin{corollary}\label{cor:nolimit}
  Let $X$ be as in \autoref{prop:nolimit} and assume $0 < t \leq 3/7$.
  Let $S$ be a smooth cubic surface without an Eckardt point.
  Then $X$ is not in the closure of the $\SL_4$ orbit of $S$.
\end{corollary}
\begin{proof}
  We use the classification of $t$-stable pairs $(S,H)$ due to Gallardo and Martinez-Garcia \cite{gal.mar:19}.
  For a smooth cubic surface $S$ without an Eckardt point and any hyperplane $H$, the curve $D = S \cap H$ is reduced and has $A_n$ singularities for $n \leq 3$ (at worst tacnodes).
  From \cite[Theorem~2]{gal.mar:19}, it follows that $(S, H)$ is $t$-stable.
  We now use \autoref{prop:nolimit}.
\end{proof}

\begin{corollary}
  The following surfaces with stabilizer $\mathbb G_m \subset \SL_4$ are not in the closure of the $\SL_4$-orbit of a generic cubic surface (the parenthesis describes the singularities):
  \begin{enumerate}
  \item $x_0x_1x_3 = x_2^3$ \quad $(3A_2)$
  \item $x_3(x_0x_2-x_1^2) = x_0x_1^2$ \quad $(A_3 + 2A_1)$
  \item $x_3(x_0x_2-x_1^2) = x_0^2x_1$ \quad $(A_4 + A_1)$
  \item $x_3x_0^2 = x_1^3 + x_2^3$ \quad $(D_4)$
  \end{enumerate}
\end{corollary}
\begin{proof}
  Together with a suitable $\mathbb G_m$-invariant hyperplane section $H$, these surfaces are $t$-semistable for $0 < t < 1$, $t = 1/5$, $t = 1/3$, and $t = 3/7$, respectively (\cite[Table~2]{gal.mar:19}).
  We now apply \autoref{cor:nolimit}.
\end{proof}

Let $X$ be a good cubic surface (not in the closure of a general orbit) with stabilizer group $\mathbb G_m$.
We have a good family
\[ \mathcal X = [X / \mathbb G_m] \xrightarrow{\pi} [. / \mathbb G_m] = B\mathbb G_m,\]
yielding a moduli map
\[\mu \colon B \mathbb G_m \to [\mathbb P^{19}/\SL_4].\]
By construction, the pull-back under $\mu$ of the closure of a generic orbit is empty.
Therefore, we must have
\[ a_{1^4} v_1^4 + a_{1^22}v_1^2v_2 + a_{13}v_1v_3 + a_{2^2}v_2^2 + a_4v_4 = 0.\]
By computing $v_1^4, \dots, v_4$, we get a linear relationship among the coefficients.

To compute $v_1^4, \dotso, v_4$, we must identify the vector bundle $\pi_* \left( \omega_{\mathcal X/B\mathbb G_m}^{-1}\right)$.
To do so, let $\mathcal V$ be the rank 4 vector bundle on $B \mathbb G_m$ corresponding to the representation $\mathbb G_m \to \SL_4$.
Then $\mathcal X \subset \P \mathcal V$ is a divisor, given by a global section of a line bundle
\[ \mathcal L = \mathcal O_{\P \mathcal V}(3) \otimes \pi^* M\]
for some line bundle $M$ on $B\mathbb G_m$.
By adjunction, we have
\begin{align*}
  \omega_{\mathcal X/B\mathbb G_m} &= \omega_{\P \mathcal V / \mathbb G_m} \otimes \mathcal L \big|_{\mathcal X}\\
                                   &= \mathcal O_{\P \mathcal V}(-4) \otimes \mathcal O_{\P \mathcal V}(3) \otimes \pi^* M \big|_{\mathcal X} \\
                                   &= \mathcal O_{\P \mathcal V}(-1) \otimes \pi^* M \big |_{\mathcal X},
\end{align*}
and hence
\begin{align*}
  \pi_* \left( \omega_{\mathcal X/B\mathbb G_m}  \right) &= \pi_* \left( \mathcal O_{\P \mathcal V}(1) \big|_{\mathcal X} \right) \otimes M^\vee \\
  &= \mathcal V \otimes M^\vee.
\end{align*}
Hence, by identifying $\mathcal V$ and $M$, we can easily compute the chern classes $v_i$.

Given $a \in \mathbb Z$, we denote by $\chi(a)$ the line bundle on $B\mathbb G_m$ corresponding to the character $t \mapsto t^a$.
Denote by $q \in A^1(B\mathbb G_m)$ the generator of the cohomology ring given by $q = c_1(\chi(1))$.
\subsection{The family $3A_2$}
For
\[ X: x_0x_1x_3 = x_2^3\]
a stabilizing $\mathbb G_m \subset \SL_4$ is given by the diagonal matrix with exponents $(a,b,0,c)$ where $a+b+c = 0$.
Thus, we have
\[\mathcal V = \chi(a) \oplus \chi(b) \oplus \chi(0) \oplus \chi(c).\]
The element $x_0x_1x_3 - x_2^3$ is a global section of $\sym^3 \mathcal V$.
In other words, $\mathcal X$ is a section of $\mathcal O_{\P \mathcal V}(3)$ and hence we have
\[ M = \chi(0).\]
We get
\[ v_1 = 0, \quad v_2 = (ab+bc+ca) \cdot q^2, \quad v_3 = abc \cdot q^3, \quad v_4 = 0,\]
and hence
\[ v_1^4 = 0, \quad v_1^2v_2 = 0, \quad v_1v_3 = 0,\quad v_2^2 \neq 0, \quad v_4 = 0.\]
Therefore, this family gives the relation
\begin{equation}\label{eqn:iso1}
  a_{2^2} = 0.
\end{equation}

\subsection{The family $A_3 + 2A_1$}
For
\[ X: x_3(x_0x_2-x_1^2) = x_0x_1^2,\]
the stabilizing $\mathbb G_m \subset \SL_4$ is given by the diagonal matrix with exponents $(-3,1,5,-3)$.
Thus, we have
\[ \mathcal V = \chi(-3) \oplus \chi(1) \oplus \chi(5) \oplus \chi(-3).\]
The element $x_3(x_0x_2-x_1^2) - x_0x_1^2$ defines a global section of $\sym^3 \mathcal V \otimes \chi(1)$.
In other words, we have
\[ M = \chi(1).\]
%((-4 -16 64 0) (256 -256 -256 256 0))
We get
\[ v_1 = -4q, \quad v_2 = -16q^2, \quad v_3 = 64q^3, \quad v_4 = 0,\]
and hence
\[ v_1^4 = 256 \quad v_1^2v_2 = -256, \quad v_1v_3 = -256, \quad v_2^2 = 256, \quad v_4 = 0.\]
Therefore, this family gives the relation
\begin{equation}\label{eqn:iso2}
  a_{1^4} - a_{1^2 2}- a_{13} +  a_{2^2}  = 0.
\end{equation}

\subsection{The family $A_4 + A_1$}
For
\[ X: x_3(x_0x_2-x_1^2) = x_0^2x_1,\]
the stabilizing $\mathbb G_m \subset \SL_4$ is given by the diagonal matrix with exponents $(1,-1,-3,3)$.
Thus, we have
\[ \mathcal V = \chi(1) \oplus \chi(-1) \oplus \chi(-3) \oplus \chi(3).\]
The element $x_3(x_0x_2-x_1^2) - x^2_0x_1$ defines a global section of $\sym^3 \mathcal V \otimes \chi(-1)$.
In other words, we have
\[ M = \chi(-1).\]
We get
\[ v_1 = 4q, \quad v_2 = -4q^2, \quad v_3 = -16q^3, \quad v_4 = 0,\]
and hence
\[ v_1^4 = 256 \quad v_1^2v_2 = -64, \quad v_1v_3 = -64, \quad v_2^2 = 16, \quad v_4 = 0.\]
Therefore, this family gives the relation
\begin{equation}\label{eqn:iso3}
  16a_{1^4} -4a_{1^2 2}- 4a_{13} + 1 a_{2^2}  = 0.
\end{equation}

\subsection{The family $D_4$}
For
\[ X: x_3x_0^2 = x_1^3+x_2^3,\]
the stabilizing $\mathbb G_m \subset \SL_4$ is given by the diagonal matrix with exponents $(5,1,1,-7)$.
Thus, we have
\[ \mathcal V = \chi(5) \oplus \chi(1) \oplus \chi(1) \oplus \chi(-7).\]
The element $x_3x_0^2-x_1^3-x_2^3$ defines a global section of $\sym^3 \mathcal V \otimes \chi(-3)$.
In other words, we have
\[ M = \chi(-3).\]
We get
\[ v_1 = -12q, \quad v_2 = -16q^2, \quad v_3 = 192q^3, \quad v_4 = -512q^4,\]
and hence
\[ v_1^4 = 20736 \quad v_1^2v_2 = 2304, \quad v_1v_3 = -2304, \quad v_2^2 = 256, \quad v_4 = -512.\]
Therefore, this family gives the relation
\begin{equation}\label{eqn:iso4}
  81a_{1^4} +9a_{1^2 2}- 9a_{13} +  a_{2^2}-2a_4  = 0.
\end{equation}

\subsection{Families over a base variety}
By choosing a suitable base $B$ and a map $B \to B \mathbb G_m$, we can construct an isotrivial family over $B$ by pulling back one of the families above.
We describe this explicitly in an example.

Let $B = \mathbb P^4$.
Set
\[V = \mathcal O(5) \oplus \mathcal O(1) \oplus \mathcal O(1) \oplus \mathcal O(-7).\]
Let $x_0, x_1, x_2, x_3$ be generators of the four summands on the standard $\mathbb A^4 \subset \mathbb P^4$.
We have the section $(x_3x_0^2-x_1^3-x_2^3)$ of $\sym^3 V = \mathcal O(3)^{\oplus 20}$.
Note that this section has a pole of order $3$ along the hyperplane at infinity.
As a result, it defines a section $\xi$ of $\sym^3 V \otimes \mathcal O(-3)$, which is nowhere vanishing.
Equivalently, $\xi$ is a section of $\mathcal O_{\P V}(3) \otimes \pi^* \mathcal O(-3)$.
Our family $\mathcal X \to B$ is defined by the zero-locus of $\xi$ in $\P V  \to B$.

\appendix

\section{Existance of a universal formula}

In this section, we recall statements from equivariant intersection theory that guarantee the existence of the formula \eqref{eq:P}. On this first pass, we will let $\mathbb{P}(V)$ denote the projective bundle of 1-dimensional subspaces in the fibers of $V$ and correct for the sign later if people want to work with duals. 

\begin{lemma}
\label{lem:Q_Zclass}
Let $W$ be an $r+1$-dimensional vector space and $Z\subset \operatorname{Sym}^d W^{\vee}$ be a subvariety. Then, the equivariant class $[\mathbb{P}Z]$ inside the equivariant Chow ring 
\begin{align}
    A^{\bullet}(\mathbb{P}_{GL_{r+1}}(\operatorname{Sym}^d W^{\vee}))\cong \frac{\mathbb{Z}[H, t_0,\ldots,t_{r}]^{S^{r+1}}}{\left(\displaystyle\prod_{\substack{i_0+\cdots+i_{r}=d\\ i_0,\ldots,i_r\geq 0}}(H-i_0 t_0 - \cdots - i_r t_r)\right)}\label{eq:equivariantSym}
\end{align}
can be expressed as
\begin{align*}
    Q_Z(t_0-\frac{H}{d},\ldots,t_{r}-\frac{H}{d}),
\end{align*}
for polynomial homogeneous $Q_Z$ with degree equal to the codimension of $Z$. 
\end{lemma}

The variables $t_i$'s in \autoref{lem:Q_Zclass} can either be viewed as chern roots or via the inclusion into the torus-equivariant Chow ring 
\begin{align*}
    A^{\bullet}(\mathbb{P}_{T}(\operatorname{Sym}^d W^{\vee}))\cong \frac{\mathbb{Z}[H, t_0,\ldots,t_{r}]}{\left(\displaystyle\prod_{\substack{i_0+\cdots+i_{r}=d\\ i_0,\ldots,i_r\geq 0}}(H-i_0 t_0 - \cdots - i_r t_r)\right)}
\end{align*}
where $T\subset GL_{r+1}$ is the standard torus.

\begin{proof}[Proof of \autoref{lem:Q_Zclass}]
The formula for the Chow ring of $A^{\bullet}(\mathbb{P}_{GL_{r+1}}(\operatorname{Sym}^d W^{\vee}))$ in \eqref{eq:equivariantSym} follows from the Chow ring of a projective bundle. %applied to the construction of an equivariant Chow ring, didn't find a quick reference for this than other people saying its obvious

The formula for $[\mathbb{P}Z]$ in terms of a polynomial $Q_Z$ follows from \cite[Theorem 6.1]{FNR05}, where $Q_Z$ is the formula for the class of $[Z]$ in $A^{\bullet}_{GL_r+1}(\operatorname{Sym}^d W^{\vee})\cong \mathbb{Z}[H, t_0,\ldots,t_{r}]$. 
\end{proof}

\begin{proposition}
In the context of \autoref{def:goodfamily} a universal polynomial $P$ as in \eqref{eq:P} exists.
\end{proposition}

\begin{proof}
Following the setup in \eqref{eq:P}, we start with a good family $\pi: X\to B$ and let $V=\pi_{*}\omega_{X/B}^{-1}$. Since each fiber of $X\to B$ is embedded into $\mathbb{P}(V)$ as a cubic surface via the anticanonical map, we get a section $\sigma$ of the bundle $\mathbb{P}(\operatorname{Sym}^3 V^{\vee})\to B$. The section $\sigma$ maps each point $b$ to the equation of the cubic surface $X_b$ lying above it.

Inside of $\mathbb{P}(\operatorname{Sym}^3 V^{\vee})$, the section $\sigma$ is induced by the inclusion of a line bundle $L=\det(V^{\vee})\hookrightarrow \operatorname{Sym}^3 V^{\vee}$. {\color{red} Check this, the main thing we need is that the line bundle can be expressed in terms of $c_1(V)$ only. I don't have a clean proof yet.}

Now, we apply \autoref{lem:Q_Zclass}, where $Z\subset \operatorname{Sym}^3(\mathbb{C}^{4\ast})$ is the $GL_4$ orbit closure of a general degree 3 homogenous equation in 4 variables. The vector bundle $V\to B$ induces a map $B\to [\operatorname{pt}/GL_4]$ under which $[\operatorname{Sym}^3(\mathbb{C}^{4\ast})/GL_4]$ pulls back to $\mathbb{P}(\operatorname{Sym}^3 V^{\vee})$. Under the pullback the equivariant class of the projectivized orbit closure $[\mathbb{P}Z]$ pulls back to the class of the relative orbit closure $\mathcal{Z}\subset \mathbb{P}(\operatorname{Sym}^3 V^{\vee})$, where each fiber of $\mathcal{Z}\to B$ is given by $Z$. By \autoref{lem:Q_Zclass}, the class of $\mathcal{Z}$ depends only on the chern classes of $V$. 

The polynomial $P$ in \eqref{eq:P} is the intersection number of $\sigma$ with the class of $\mathcal{Z}$. The class of $\sigma$, the Chow ring of $\mathbb{P}(\operatorname{Sym}^3 V^{\vee})$, and the class of $\mathcal{Z}$ are depend only on the chern classes of $V$. Thus, the final intersection number depends only on the chern classes of $V$ as desired. 
{\color{red} I don't think it'll be any harder to actually determine the formula of $P$ in \eqref{eq:P} in terms of the equivariant class of an orbit closure of a cubic surface, but we don't seem to need the formula. The point is that $P$ and the equivariant class of the orbit closure are equivalent by some substitution.}
\end{proof}

\section{Congruence relations from hypersurfaces with finite automorphisms}
As another another check for the class, I think we can get congruence relations from isotrivial families of hypersurfaces with finite linear automorphisms. This is the same idea as the $3A_2$ family, except the condition holds in $H_{G}(\operatorname{pt})$ with $G$ a finite group instead of $\mathbb{G}_m$. 

\begin{proposition}
\label{prop:isotrivial2}
Let $X\subset \mathbb{P}^r$ be a hypersurface with a linear automorphism group $G\subset \operatorname{Aut}(X)\subset GL_{r+1}$. Let $Q$ be the polynomial $Q_Z$ in \autoref{lem:Q_Zclass} where $Z$ \emph{any} equivariant cycle in the space of hypersurfaces not containing $X$. Then, under the linear representation $G\hookrightarrow GL_{r+1}$, $Q$ is sent to zero under the characteristic class map:
\begin{align*}
    A^{\bullet}_{GL_{r+1}}\cong \mathbb{Z}[t_0,\ldots,t_{r}]\to A^{\bullet}_{G}(\operatorname{pt}). 
\end{align*}
\end{proposition}

% \begin{proof}
% Continue: the key point is that the twisted family of hypersurfaces comes from descending a trivial section of Sym^d of the universal vector bundle pulled back from BG\to BGL_{r+1}, so intersecting this section with the orbit closure picks out constant term of the projective equivariant class. This constant term is exactly $Q$ in the proposition.

% This is just the general construction of the twisted family:
% Given a hypersurface $X\subset \mathbb{P}^r$ with a finite linear automorphism group $G=\operatorname{Aut}(X)\subset GL_{r+1}$, we want to construct a twisted version of $X$ over $BG$. To this end, take $X\times ((\mathbb{C}^{r+1})^N\backslash Z)\subset \mathbb{P}^r \times ((\mathbb{C}^{r+1})^N\backslash Z)$, where $N>>0$ and $Z$ is some closed set of high codimension. Here $G$ acts on $X$ and $\mathbb{P}^{r+1}$ by linear automorphisms and on each factor of $\mathbb{C}^{r+1}$ via $GL_{r+1}$. The closed set $Z$ is chosen to $G$ acts freely on $(\mathbb{C}^{r+1})^N\backslash Z$. 

% Taking the quotient by $G$ yields a nontrivial family of hypersurfaces $\mathcal{X}\to ((\mathbb{C}^{r+1})^N\backslash Z)/G$ inside the projectivization of some vector bundle $W$ over $((\mathbb{C}^{r+1})^N\backslash Z)/G$.
% \end{proof}
\begin{example}
\autoref{prop:isotrivial2} specializes to \autoref{sec:isotrivial3A2} when applied to the cubic surface $X^3=YZW$.
\end{example}

\begin{example}
Every coefficient of every class of a quartic plane curve we computed in \cite{LPT19} is divisible by 4 (even after dividing by the automorphism groups). I think this is a consequence of the Fermat quartic having automorphism group containing $(\mu_4)^3\hookrightarrow GL_3$. Similarly, every equivariant class of a cubic curve orbit is divisible by 3 in \cite[Appendix C]{LPT19}, and this should follow from the Fermat cubic. We will try to prove this in \autoref{prop:divd} below.
\end{example}

Given the example, we will try to flesh out the application of \autoref{prop:3A2} to the Fermat hypersurface applied to the case $G=(\mu_d)^{r+1}\subset GL_{r+1}$. In general, I don't know how to compute the map $A^{\bullet}_{GL_{r+1}}\to A^{\bullet}_{G}(\operatorname{pt})$ (google ``characteristic class of a linear representation''). 

\subsection{Fermat hypersurfaces}
A Fermat hypersurface $X\subset \mathbb{P}^r$ of degree $d$ has a linear automorphism group containing $(\mu_d)^{r+1}$. {\color{red} You can also permute the coordinates, curious what condition in $A^{\bullet}_{S_{r+1}}$ that gives...}

\begin{lemma}
\label{lem:mu_d}
Under the inclusion $\mu_d\hookrightarrow \mathbb{G}_m$, the map in group cohomology is given by 
\begin{align*}
    \mathbb{Z}[t]=A^{\bullet}_{\mathbb{G}_m}(\operatorname{pt})\to A^{\bullet}_{\mu_d}(\operatorname{pt})=\mathbb{Z}[t]/(dt)
\end{align*}
\end{lemma}

\begin{proof}
The ring structure of $A^{\bullet}_{\mu_d}(\operatorname{pt})$ is given by the cohomology ring of the Lens space 
\href{https://mathoverflow.net/questions/133974/reference-for-ring-structure-on-group-cohomology}{(mathoverflow link)}. To compute the map, we take an approximation of $B\mathbb{G}_m$ and $B\mu_d$ to compute the map in $A^{1}_{\mathbb{G}_m}(\operatorname{pt})\to A^{1}_{\mu_d}(\operatorname{pt})$. This reduces to a familiar geometric example of a cone over a rational normal curve.

We approximate using the map $B\mu_d\to B\mathbb{G}_m$ using
\begin{align*}
    \pi: S=\operatorname{Spec}(\mathbb{C}[x^d, x^{d-1}y,\ldots,y^d])=(\mathbb{A}^2\backslash\{0\})/\mu_d\to (\mathbb{A}^2\backslash\{0\})/\mathbb{G}_m=\mathbb{P}^1. 
\end{align*}
The map $\pi$ is the projection of the cone over a degree $d$ rational normal curve (minus the origin) onto the rational normal curve. The content of \autoref{lem:mu_d} is that a ruling of the cone $S$ over a rational normal curve generates the class group of the cone. 

This is true because $S\backslash L\cong \mathbb{A}^2$ where $L$ is a ruling. (The more nontrivial fact is that $A^{1}(S)\cong \mathbb{Z}/d\mathbb{Z}$, so $[L]$ is exactly $d$-torsion. This is implied by the cohomology of the lens space above, but one can probably prove it directly. For example, $[L]$ is nontrivial in the class group because it cannot be cut out by a single equation (since the tangent space at the cone is at least 3-dimensional for $d>1$).) 
\end{proof}

\begin{proposition}
\label{prop:divd}
If $Q$ is the polynomial associated to the equivariant class of \emph{any} equivariant class in the space of degree $d$ hypersurfaces not containing the Fermat hypersurface, then every coefficient of $Q$ is divisible by $d$.
\end{proposition}

\begin{proof}
This follows from \autoref{lem:mu_d} and \autoref{prop:isotrivial2}.
\end{proof}

\begin{example}
For a general cubic surface, our current guess is that:
$$Q = 25920 c_1^4+12960 c_1^2 c_2-6480 c_1 c_3+9720 c_4$$
 and
$$P = 1080 c_1^2 c_2-1080 c_1 c_3+9720 c_4,$$
where the $P$ and $Q$ are related by
\begin{align*}
P(t_0,t_1,t_2,t_3) &= Q(t_0 - \frac{t_0+\cdots+t_3}{3},\ldots, t_3 - \frac{t_0+\cdots+t_3}{3})\\
P(t_0 - (t_0+\cdots+t_3), \ldots, t_3 - (t_0+\cdots+t_3)) &= Q(t_0,\ldots,t_3)
\end{align*}
The coefficients of $Q$  are all divisible by 3 as predicted.
\end{example}
\bibliographystyle{plain}
\bibliography{references.bib}
\end{document}

%%% Local Variables:
%%% mode: latex
%%% TeX-master: t
%%% End:

