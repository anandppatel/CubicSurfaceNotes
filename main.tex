\documentclass[12 pt]{amsart}

%============================
%         Obligatory
%============================
\usepackage{mathtools}
\usepackage{amssymb}
\usepackage{amsmath}
\usepackage{amsthm}
\usepackage{mathrsfs}
\usepackage{tikz}
\usepackage{array}

%============================
%         Aesthetic
%============================
\usepackage[hidelinks]{hyperref} % For citations without hideous colored boxes
\usepackage{geometry} % For adjusting the page domensions if desired
%\usepackage{sectsty} % For centering Section/Subsection titles
\usepackage{marginnote} % Currently unused, but for notes in the margin
\usepackage{graphics} % For custom pretty fractions 
\usepackage{newtxmath} % Changes the math font to be bolder and prettier

\geometry{margin=1.66 in}
%\sectionfont{\centering}
%\subsectionfont{\centering}
\linespread{1.1}


\title{Counting Cubic Surfaces}
\author{Anand Patel \& Dennis Tseng}


\newtheorem{theorem}{Theorem}[section]
\newtheorem{lemma}{Lemma}[section]
\newtheorem{proposition}{Proposition}[section]
\newtheorem{conjecture}{Conjecture}[section]
\newtheorem{definition}{Definition}[section]

\newenvironment{example}{\textbf{Example:}}{}
\newenvironment{remark}{\textbf{Remark:}}{}

% chk is a stylized "dual" symbol created by altering the vee command 
\newcommand{\chk}[1]{#1^{\text{\scalebox{.8}[.6]{$\boldsymbol{\vee}$}}}}

% altfrac and altfraci are twin commands for alternate fractions. The difference between them is that altfrac is bogger and prettier, while altfraci fits in inline mode. In the future I plan to merge them into one command which checks the displaystyle and uses the appropriate form of the command. Note that this command is specialized to the newtxmath package, and without it the fraction shape could be distorted. This can be solved by adjusting the different numbers in the command.
\newcommand{\altfrac}[2]{\ifmmode\def\tmp{$}\else\def\tmp{}\fi
		\mbox{{\raisebox{.5\ht\strutbox}{\tmp#1\tmp}}
		\kern-5pt\scalebox{1.75}[1.5]{/}\kern-1.5pt
		\mbox{\raisebox{-.2\ht\strutbox}{\tmp#2\tmp}}}
}

\newcommand{\altfraci}[2]{\ifmmode\def\tmp{$}\else\def\tmp{}\fi
		\scalebox{.925}[.925]{{\mbox{{\raisebox{.25\ht\strutbox}{\tmp#1\tmp}}
		\kern-5pt\scalebox{1.5}[1.3]{\raisebox{-.1\ht\strutbox}{/}}\kern-0.5pt
		\mbox{\raisebox{-.15\ht\strutbox}{\tmp#2\tmp}}}}}
}


\renewcommand{\P}{\mathbb{P}}
\newcommand{\T}{\mathbb{T}}
\newcommand{\e}{\varepsilon}
\newcommand{\<}{\left\langle}
\renewcommand{\>}{\right\rangle}

\newcommand{\um}[1]{\color{blue}{#1}\color{black}}
\newcommand{\remove}[1]{\color{red}{#1}\color{black}}
\newcommand{\add}[1]{\color{green}{#1}\color{black}}

\newcommand{\keps}{\altfrac{k[\e]}{\left(\e^2\right)}}
\newcommand{\kepsi}{\altfraci{k[\e]}{\left(\e^2\right)}}
\newcommand{\mf}[1]{\mathfrak{#1}}

\DeclareMathOperator{\Bl}{Bl}
\DeclareMathOperator{\Sec}{Sec}
\DeclareMathOperator{\spec}{Spec}
\DeclareMathOperator{\res}{res}
\DeclareMathOperator{\Ima}{Im}

\begin{document}

\maketitle


\section{Introduction}
\label{sec:intro}
Our objective is to determine an enumerate formula which counts the
number of times a general cubic surface occurs in a general
$4$-parameter family $ \pi: X \to B$ of cubic surfaces.  We work
throughout with finite-type schemes over an algebraically closed,
characteristic ground field $k$.

\begin{definition}
  Let $\pi :X \to B$ be a flat, projective morphism of relative
  dimension $2$, with $B$ a $4$-dimensional proper scheme. We say
  $\pi$ is {\bf a good family of cubic surfaces} if for all $b \in B$,
  the fiber $X_{b}$ is isomorphic to a cubic surface which is not
  contained in a general cubic surface's $PGL_4$-orbit closure.
\end{definition}

The base of a good family carries the natural rank $4$
vector bundle
\begin{align}
  \label{eq:V}
  V := \pi_{*}\omega_{X/B}^{-1}.
\end{align}
The base $B$ of a good family of cubic surface admits a rational map
\begin{align}
  \label{eq:mu}
  \mu: B \dashrightarrow M
\end{align}
to the coarse moduli variety $M$ of cubic surfaces.  By general
equivariant-geometric arguments, there exists a universal
weighted-degree $4$ polynomial
\begin{align}
  \label{eq:P}
  P(v_1,v_2,v_3,v_4) = a_{1^4}v_{1}^{4} + a_{1^{2}2}v_{1}^{2}v_{2} + a_{13}v_{1}v_{3} + a_{2^2}v_{2}^{2} + a_{4}v_{4}
\end{align}
in the Chern classes $v_{i} = c_{i}V$, such that
\begin{align}
  \label{eq:equality}
  \deg P(v_{1},v_{2},v_{3},v_{4}) = \deg \nu
  \end{align}
Our goal is to determine $P$, which amounts to determining
the five coefficients $a_{\bullet}$.

This is essentially an exercise in the method of indetermined
coefficients. We devote each section to the construction of a
particular good family.  After each section, we reset notation.

\section{Family $B_1$}
\label{sec:first-test-family}

In this section, we let $S_7 \subset \P^7$ denote a degree $7$ del
Pezzo surface. We let $E_{1}, E_{2}$ denote the two disjoint lines on
$S_{7}$ and we let $L \subset S_{7}$ denote the third line on $S_{7}$,
which meets both $E_{1}$ and $E_{2}$. Finally, we choose a general
rational normal quintic curve $C \subset S_{7}$ meeting $E_{2}$ at a
single point other than $L \cap E_{2}$.

The base of our first family is
\begin{align}
  \label{eq:B1}
  B_1 = C^{[4]} \simeq \P^{4}.
\end{align}
 $B_1$ carries a good family of cubic surfaces over it.
To see this, one must only note that any length four subscheme
$Z \subset C$ satisfies the property that the three dimensional span
$\langle Z \rangle$ intersects $S_7$ at precisely the scheme
$Z$. Therefore, for each such $Z$, the projection
$p_{\langle Z \rangle}: S_7 \dashrightarrow \P^{3}$ has as image an
irreducible cubic surface.  Furthermore, one can check that each of
these has a finite automorphism group.

The Chern classes of the resulting vector bundle $V$ on
$B_{1} = \P^{4}$ are easily determined. In fact we have:
$v_{1} = -2h$, $v_{2}=h^2$, and $v_{3}= v_{4}=0$, where $h$ denotes
the hyperplane class on the projective space $B_1$.

\subsection{Enumerating cubics in $B_1$}
\label{sec:enum-cubics-b_1}


For $B_1$ to be of any use, we must {\sl a priori} know
$\deg P(v_{1},v_{2},v_{3},v_{4})$ for $B_1$. This is easily
achievable, given basic facts about the configuration of lines on a
general cubic surface. We let
\begin{align}
  \label{eq:nu}
  \nu: M^{\dagger} \to M
  \end{align}
  denote the variety of smooth {\sl marked} cubic surfaces:
  $M^{\dagger}$ parametrizes tuples $(X, E_{1}, \dots, E_{6})$ where
  the $E_{i}$ are six disjoint lines on $X$. It is a classical fact
  that $\deg \nu = 6! \times 72$. The symmetric group $\Sigma_6$ acts
  on $M^{\dagger}$ by permuting the marked lines.

  We claim that the natural rational map to moduli
  $\mu : B_1 \dashrightarrow M$ factors through a birational map
  $\mu^{\dagger}: B_1 \dashrightarrow M^{\dagger}/\Sigma_{4}$, where
  $\Sigma_{4}$ acts by permuting the last four lines
  $E_{3}, \dots, E_{6}$.

Assuming this fact, we arrive at the first relation:

\begin{align}
  \label{eq:relation1}
  16 \cdot a_{1^4} + 4 \cdot a_{1^2 2} + a_{2^2} = 6! \times 3 = 2160.
\end{align}

\section{Family $B_2$}
\label{sec:family-b_2}

Our next family is a slight variant of $B_1$. Let $S$ be a degree $7$
del Pezzo surface, viewed as the blow up of the plane at two points,
with its two exceptional lines $E_1, E_2$ and its third exceptional
line $L$.  Next, fix a general conic $C \subset S$ which is {\sl
  tangent} to $L$ at a point $p$ distinct from the pair
$L \cap (E_1 \cup E_2)$.

The configuration space $C^{[4]} \simeq \P^{4}$ carries a family of
cubic surfaces $X/\P^{4}$.  Unfortunately, the configurations of the
form $x+y+2p$, $x,y \in C$, register reducible cubic surfaces, and so
$X/\P^{4}$ is not good.  This locus is a $2$-plane
$\Lambda \subset \P^{4}$.

\begin{proposition}
  \label{proposition:resolveB2} The blow up of $\P^{4}$ along
  $\Lambda$ carries a good family extending
  $X/(\P^{4} \setminus \Lambda)$.
\end{proposition}
(Actually, hold off: One needs to blow up at least one more time,
corresponding to the configuration $4p$.)
\subsection{Enumerating cubics in $B_2$}
\label{sec:enum-cubics-b_2}

Next we must compute the degree of the map to moduli
$\mu: B_2 \dashrightarrow M$. As in the case of $B_1$, $\mu$ factors
through $\mu^{\dagger}: B_2 \dashrightarrow M^\dagger/\Sigma_{4}$.
However,  there is a distinction:

\begin{proposition}
  \label{proposition:deg-mudagger2}
  We have: $\deg \mu^{\dagger} = 2$.
\end{proposition}

\section{Family $B_3$}
\label{sec:family-b_3}

We construct our third family $B_3$ be starting with a quintic del
Pezzo surface $S$, which is the blow up of the plane at four general
points, and then ``blowing up all pairs of points on $S$.''

The surface $S$ has $10$ exceptional lines $L_{1}, \dots, L_{10}$.  We
let $S^{[2]}$ denote the Hilbert scheme of pairs of points on $S$.
$S^{[2]}$ contains $10$ disjoint $\P^{2}$'s,
$\Lambda_1, \dots, \Lambda_{10}$, namely the Hilbert schemes of pairs
of points contained in the lines $L_{i}$. Let
$\Lambda = \cup_{i}\Lambda_{i}$.

\subsection{Construction of $B_3$}
\label{sec:construction-b_3}

Let $Z \subset S^{[2]} \times S$ denote the universal degree two
subscheme; $Z$ is smooth and the first projection
$\pi_1 : Z \to S^{[2]}$ is finite, flat, and has degree two.



First, observe that the open set
\begin{align}
  \label{eq:UB3}
  U := S^{[2]} \setminus \Lambda
\end{align}
supports a good family of cubic surfaces, namely
\begin{align}
  \label{eq:XUB3}
  X^{\circ} := \Bl_{Z}(S^{[2]} \times S).
\end{align}

(We encourage the reader to check that $X^{\circ}/U$ is indeed good.)

However, when we extend $X^{\circ}$ in the natural way over $S^{[2]}$
the family acquires bad fibers, namely unions of quadrics and planes,
and therefore we must modify $S^{[2]}$ along $\Lambda$.

Let
\begin{align}
  \label{eq:S2tilde}
  \widetilde{S^{[2]}} := \Bl_{\Lambda}S^{[2]}
\end{align}
denote the blow-up with blow-down map
$\beta: \widetilde{S^{[2]}} \to S^{[2]}$.  Let
$E_{i} = \beta^{-1}(\Lambda_{i})$, $i=1, \dots, 10$ denote the
components of the of the exceptional divisor of the blow-up; $\beta$
expresses $E_{i}$ as a $\P^{1}$-bundle over $\Lambda_{i}$.

Next, let $\widetilde{Z} := \widetilde{S^{[2]}} \times_{S^{[2]}} Z$
denote the fiber-product; $\widetilde{Z}$ is a closed subscheme of the
product $\widetilde{S^{[2]}} \times S$, and is a finite, flat, degree
two cover of $\widetilde{S^{[2]}}$ under the first projection
$\tilde{\pi}_{1} : \widetilde{Z} \to \widetilde{S^{[2]}}.$

Inside the product $\widetilde{S^{[2]}} \times S$ we define the
smooth, closed subschemes
\begin{align}
  \label{eq:Fi}
  F_{i} := E_{i} \times L_{i};
\end{align}
we let $F := \cup_{i}F_{i}$.  We perform one more blow-up to get 
\begin{align}
  \label{eq:Xtilde}
  \widetilde{Y} := \Bl_{F}\widetilde{S^{[2]}} \times S.
\end{align}
The natural projection $ Y \to \widetilde{S^{[2]}}$ is a family of
surfaces with simple normal crossing singular fibers over $E_{i}$.
The next lemma is critical.

\begin{lemma}
  The closed immersion
  $\widetilde{Z} \subset \widetilde{S^{[2]}} \times S$ lifts to a
  closed immersion $\widetilde{Z} \subset Y$.
\end{lemma}

\begin{proof}
  The main observation is that we have an equality of schemes
  \begin{align}
    \label{eq:Cartier}
    F_i \cap \widetilde{Z} = \widetilde{\pi}_{1}^{-1}(E_i) \cap \widetilde{Z}
  \end{align}
  for all $i=1, \dots, 10$, {\sl and therefore $F \cap \widetilde{Z}$
    is a Cartier divisor on $\widetilde{Z}$}. The claim now follows
  from a basic property of blow-ups.
\end{proof}



We can finally define our proposed family of cubics.  We let
\begin{align}
  \label{eq:XB3}
  X := \Bl_{\widetilde{Z}}Y
\end{align}
denote the blow-up,
\begin{align}
  \label{eq:phiB3}
  \varphi: X \to Y
\end{align}
the blow-down map, and
\begin{align}
  \label{eq:piB3}
  \pi: X \to \widetilde{S^{[2]}}
\end{align}
the natural projection. 
\begin{proposition}
  \label{proposition:resolv-b3}
 The family 
\end{proposition}

(This one I still am confident in -- one simple blow-up resolves the situation.)

\section{Family $B_4$}
\label{sec:family-b_4}

For this family, we fix a degree $6$ del Pezzo surface $S$, the blow
up of the plane at three non-collinear points, and then we fix a
general ``line'' (which is really a twisted cubic) $L$ in $S$ disjoint
from the three points blown up.  An open subset $U$ of the variety
$L \times L \times S$ carries a good family of cubic surfaces in the
obvious way. However, as in the case of $B_3$, the complement of $U$
is a union of surfaces which need to be blown up before obtaining a
good family of cubic surfaces.


\section{Family $B_5$}
\label{sec:family-b_5-1}

This family is similar to $B_1$. Fix a degree $7$ del Pezzo surface
$S$, and let $D \subset S$ denote a general hyperplane section; $D$ is
a degree $7$ elliptic normal curve.  Next observe that any degree four
subscheme $Z \subset D$ imposes independent conditions on sections of
$O_{S}(1)$, and that $\langle Z \rangle \cap S = Z$.  Furthermore, for
each $Z$, the projection map
$p_{\langle Z \rangle}: S \dashrightarrow \P^{3}$ has as image an
irreducible cubic, with finite automorphism group (check this!).


Therefore, the HIlbert scheme $B_{5} = D^{[4]}$ carries a good family
of cubic surfaces.  (The reason I like this family is that, unlike
$B_1$'s case, the natural bundle $V$ has more non-trivial chern
classes. Therefore, the relation we get this time is probably not the
same as for $B_1$.

%\bibliographystyle{plain}
%\bibliography{references.bib}
\end{document}

%%% Local Variables:
%%% mode: latex
%%% TeX-master: t
%%% End:

