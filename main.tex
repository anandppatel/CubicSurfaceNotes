\documentclass[12pt,reqno]{amsart}
\usepackage[margin = 1.35 in]{geometry}
\usepackage[frenchmath,defaultmathsizes]{mathastext}
\usepackage{
  hyperref,
  amsmath,
  amssymb,
  tikz,
  amsthm,
  thmtools,
  microtype,
  enumitem,
  stmaryrd,
  tikz-cd,
  mathrsfs,
}
\usepackage{graphicx}

\usepackage{import}
\usepackage{xifthen}
\usepackage{pdfpages}
\usepackage{transparent}
\usepackage{booktabs}

\newcommand{\incfig}[1]{%
    \def\svgwidth{\columnwidth}
    \import{./figures/}{#1.pdf_tex}
}


\linespread{1.1}
\usepackage[all]{xy}
\usepackage{eucal}

\title{Counting Cubic Surfaces}
\author{Anand Deopurkar, Anand Patel
  \& Dennis Tseng}




%-----------------------------------------------
\def\labelitemi{--}
\newcommand{\todo}[1]{\fbox{ToDo: #1}}
\renewcommand{\k}{k}
\DeclareMathOperator{\id}{id}
\DeclareMathOperator{\Bl}{Bl}
\DeclareMathOperator{\Orb}{\overline{Orb}}
\DeclareMathOperator{\Polar}{Polar}
\DeclareMathOperator{\res}{Res}
\DeclareMathOperator{\Pole}{Pole}
\DeclareMathOperator{\Hilb}{Hilb}
\DeclareMathOperator{\Sing}{Sing}
\DeclareMathOperator{\M}{\mathcal{M}}
\renewcommand{\to}{{\longrightarrow}}

\renewcommand{\sectionautorefname}{\S}
\renewcommand{\subsectionautorefname}{\S}
% Let us keep this minimial
% Let us also define things only if they are previously undefined.

% Common theorem-like environments
\ifcsname theorem\endcsname{}\else\declaretheorem[parent=subsection]{theorem}\fi
\ifcsname corollary\endcsname{}\else\declaretheorem[sibling=theorem]{corollary}\fi
\ifcsname lemma\endcsname{}\else\declaretheorem[sibling=theorem]{lemma}\fi
\ifcsname proposition\endcsname{}\else\declaretheorem[sibling=theorem]{proposition}\fi
\ifcsname conjecture\endcsname{}\else\declaretheorem[sibling=theorem]{conjecture}\fi
\ifcsname problem\endcsname{}\else\declaretheorem[sibling=theorem]{problem}\fi
\ifcsname question\endcsname{}\else\declaretheorem[sibling=theorem]{question}\fi
\ifcsname definition\endcsname{}\else\declaretheorem[sibling=theorem, style=definition]{definition}\fi
\ifcsname exercise\endcsname{}\else\declaretheorem[sibling=theorem, style=definition]{exercise}\fi
\ifcsname example\endcsname{}\else\declaretheorem[sibling=theorem, style=definition]{example}\fi
\ifcsname remark\endcsname{}\declaretheorem[sibling=theorem, style=remark]{remark}\fi

% Common abbreviations

% Absolutely standard rings and fields
\providecommand {\N}{{\bf N}}
\providecommand {\Z}{{\bf Z}}
\providecommand {\Q}{{\bf Q}}
\providecommand {\R}{{\bf R}}
\providecommand {\C}{{\bf C}}

% Common spaces grassmannian
\renewcommand {\P}{{\bf P}}
\providecommand {\Gr}{{\bf Gr}}
\providecommand {\A}{{\bf A}}

% Groups
\providecommand{\SL}{\operatorname{SL}}
\providecommand{\GL}{\operatorname{GL}}
\providecommand{\PGL}{\operatorname{PGL}}
\providecommand{\Gm}{{\bf G}_m}

% f \from G \to H reads much better than f \colon G \to H
\providecommand {\from}{{\colon}}

% Absolutely standard notation
\providecommand{\spec}{\operatorname{Spec}}
\providecommand{\proj}{\operatorname{Proj}}
\providecommand{\coker}{\operatorname{coker}}
% Kernel is already defined
\providecommand{\Blowup}{\operatorname{Bl}}
\providecommand{\Hom}{\operatorname{Hom}}
\providecommand{\Ext}{\operatorname{Ext}}
\providecommand{\Tor}{\operatorname{Tor}}
\providecommand{\End}{\operatorname{End}}
\providecommand{\Aut}{\operatorname{Aut}}
\providecommand{\codim}{\operatorname{codim}}
% Dim is already defined
\providecommand{\Pic}{\operatorname{Pic}}
\providecommand{\Sym}{\operatorname{Sym}}
\providecommand{\rk}{\operatorname{rk}}
\declaretheorem[sibling=theorem,style=remark]{remark}
\numberwithin{equation}{section}
\declaretheorem[title=Theorem]{maintheorem}
\renewcommand{\themaintheorem}{\Alph{maintheorem}}


\renewcommand{\O}{\mathcal O}
\newcommand{\G}{\mathbf G}
\newcommand{\td}{\widetilde}
\newcommand{\V}{\mathcal V}
\newcommand{\cP}{\mathcal P}
\newcommand{\cC}{\mathcal{C}}
\newcommand{\hL}{\widehat{\mathcal{L}}}
\newcommand{\hpsi}{\widehat{\psi}}
\newcommand{\fm}{\mathfrak m}
\newcommand{\smvee}{\raise0.5ex\hbox{$\scriptscriptstyle\vee$}}
\newcommand{\hM}{\widehat{\M}}
\newcommand{\compl}[1]{\widehat{#1}}
\newcommand{\Spec}{{\text{\rm Spec}\,}}
\newcommand{\Spf}{{\text{\rm Spf}\,}}
\renewcommand {\o}[1]{\overline{#1}}
\newcommand{\Proj}{{\text{\rm Proj}\,}}
\newcommand{\git}{\sslash}
% -----------------------------------------------

%%% BEGIN DOCUMENT

\begin{document}




\maketitle

\section{Introduction}
\label{sec:intro}

The beautiful enumerative geometry inherent to cubic surfaces has
captured the imagination ever since, in 1849, Cayley and Salmon
illuminated the structure of the $27$ lines.  We add to this long
history by finding an enumerative formula which counts the number of
times the {\sl moduli} of a general cubic surface arises in suitably
generic, complete, $4$ parameter families. Applying the formula to a
few simple examples yields:
\begin{enumerate}
\item[(A)] \label{fact:RanestadSturmfelds} {\sl A general cubic
    surface's moduli arises $96120$ times in a general $4$ dimensional
    linear system of cubic surfaces,} and
\item[(B)] {\sl A general cubic surface's moduli arises $42120$ times
    as a hyperplane section of a general cubic threefold.}
\end{enumerate}

Our formula, along with its precise hypotheses, is stated in the
following theorem.

\begin{theorem}\label{theorem:main}
  Let $\pi: X \to B$ be a good family of cubic surfaces over a proper
  base $B$, with semi-stable generic fiber. If
  $v_{i} \in A^{i}(B), i=1, 2,3, 4$, are the Chern classes of its
  natural rank $4$ vector bundle $\V := \pi_{*}(\omega_{\pi}^{-1})$,
  then
  \begin{align}
    \label{eq:MAIN}
   \int_{B} 1080v_{1}^{2}v_{2} - 1080v_{1}v_{3}+9720v_{4} = \deg \mu,
  \end{align}
  where $\mu: B \dashrightarrow \M$ is the induced map to the moduli
  space of cubic surfaces.
\end{theorem}

The exact notion of a good family can be found in
\autoref{sec:good-families}.  We simply mention here that it includes
the case where all fibers are semi-stable cubic surfaces (in the sense
of GIT), and the relative dualizing sheaf $\omega_{\pi}$ exists and is
a line bundle.

The formula has a curious consequence:

\begin{corollary}
  \label{cor:ineq} If $\pi: X \to B$ is a good family of cubic
  surfaces parametrized by a $4$ dimensional proper variety $B$,
  then $$\int_{B} v_{1}^{2}v_{2} - v_{1}v_3 + 9v_{4} \geq 0$$ with
  equality holding if and only if a general cubic surface does not
  arise as a fiber.
\end{corollary}



\subsection{Strategy}
\label{sec:strategy}

The overall strategy of the proof is straightforward. We argue that a
priori there must exist a formula of the form

\begin{align}
  \label{eq:P}
  \int_{B} a_{1^4}v_{1}^{4} + a_{1^{2}2}v_{1}^{2}v_{2} + a_{13}v_{1}v_{3} + a_{2^2}v_{2}^{2} + a_{4}v_{4} = \deg \mu,
\end{align}
as found in \autoref{theorem:main} -- this is merely an instance of a
foundational theorem in equivariant enumerative geometry \cite[].
Then, we determine the five coefficients $a_{\bullet}$ by exhibiting
several good families for which we know all $v_{i}$ {\sl and}
$\deg \mu$.  The construction of these good families requires some
effort and occupies most of the paper.  The bases $B$ of our families
are : $\P^4$, $\overline{\M}_{0,7}$ and a Hassett-weighted variant of
it, a blow up of the Hilbert scheme of two points on a quintic del
Pezzo surface, and amusingly, the classifying stack $B\G_m$.


In this way, we obtain a system of linear equations in the five
unknowns $a_{\bullet}$, which turns out to have a unique solution.  In
fact, we exhibit {\sl nine} total families/equations; the redundancy
provides extra checks on the end result.
  

\subsection{History and context}
\label{sec:context}

The enumerative problem (A) was first mentioned in print in
K. Ranestad and B. Sturmfeld's \cite[], and the number $96120$ was
first arrived at by \cite[] using numerical methods, though the method
was one verification short of constituting a proof.  R. Laza brought
(B) to the first two authors' attention during a meeting at
Oberwolfach in 2017.


\autoref{theorem:main} actually represents a frontier case of a much
more substantial ongoing enumerative story, which we briefly advertise
here. If $E$ is an $n+1$ dimensional vector space and if
$F \in \Sym^dE^{\smvee}$ is a non-zero degree $d$ homogeneous form on
$E$, then we let $$\Orb(F) \subset \P \Sym^d E^{\smvee}$$ denote the
closure of the $\GL(E)$-orbit of the point
$[F] \in \P \Sym^{d}E^{\smvee}$.  Simple questions about the geometry
of the variety $\Orb(F)$, and its relationship to the geometry of the
hypersurface defined by $F$, remain unanswered.  Here, perhaps the
most well-studied problem is {\sl the calculation of the degree of the
  orbit-closure $\Orb(F)$}.  Aluffi and Faber, in a series of
remarkable papers \cite[], initially brought attention to this
enumerative problem and completely solved it when $n=1, 2$ and $d$
arbitrary.

We propose a slight shift in perspective on the problem: We should
instead aim to compute the {\sl equivariant fundamental class} of the
orbit closure, denoted
$$[\Orb(F)] \in A_{\GL(E)}^{\bullet}(\P \Sym^{d}E^{\smvee}).$$
This class not only determines the degree of the orbit closure, but
also tracks many more enumerative problems like (B).  As is
demonstrated in the appendix \cite[], \autoref{theorem:main} contains
the same content as the calculation of $[\Orb(F)]$ when $F$ is assumed
to be a {\sl general} degree $3$ form in four variables.

This change in perspective opens up our avenue of attack.  As will be
seen in \autoref{section:???} we exploit certain {\sl isotrivial}
families of cubic surfaces to obtain quick relations on the
coefficients $a_{\bullet}$.  In order to justify the use of these
isotrivial families, it is necessary to know whether a cubic surface
is represented by a point in the closure of $\Orb(F)$, with $F$
general.  This is an instance of the much more general ``Orbit Closure
Problem'' \cite[].



%---------------------------------------------------------------------------


\section{Preliminaries}
\label{sec:good}

After establishing notation, conventions, and classical facts about
cubic surfaces, this section introduces the notion of a {\sl good}
family of cubic surfaces and then provides the main tool
(\autoref{prop:good}) for verifying the goodness of several families
occurring in the paper.


\subsection{Notation and conventions}
\label{sec:notation-conventions}
All schemes are assumed of finite-type over an algebraically closed,
characteristic zero field, $\k$.

\begin{itemize}

\item If $\pi: X \to Y$ is a map of schemes and if $y \in Y$ is a
  point, then we let $X_y$ denote the fiber scheme $\pi^{-1}(y)$, and
  if both $X$ and $Y$ have dualizing line bundles
  $\omega_{X}, \omega_{Y}$ respectively, then we write $\omega_{X/Y}$,
  or $\omega_{\pi}$, for the line bundle
  $\omega_{X} \otimes \pi^{*}\omega_{Y}^{-1}$.
\item If $\mathcal{W}$ is a
  vector bundle, then we let $\P \mathcal{W}$ denote its
  projectivization, which parametrizes $1$-dimensional {\sl subspaces}
  in the fibers of $\mathcal{W}$. $\P \mathcal{W}$ comes naturally
  equipped with the line bundle $\O_{\P \mathcal{W}}(1)$, whose
  push-forward to the base scheme is $\mathcal{W}^{\smvee}$.

\item If a smooth conic $C \subset \P^{2}$ is given, and if
  $p \in \P^{2}$ is any point, we let $\Polar(p)$ denote the line
  polar to $p$ with respect to $C$.  Similarly, if $L \subset \P^{2}$
  is any line, we let $\Pole(L)$ denote the point polar to $L$.

\end{itemize}


\subsection{Classical facts about cubic surfaces}
\label{sec:classical-facts}

\begin{theorem}[Cayley, Salmon]
  \label{theorem:cayleysalmon} Every smooth cubic surface contains
  exactly $27$ lines.
\end{theorem}

An ordered tuple of six lines $(\ell_1, \dots, \ell_6)$ on a cubic
surface is called a {\sl six} if they are pairwise disjoint.  An
unordered pair of sixes
$\{(\ell_1, \dots, \ell_6), (\ell_1', \dots, \ell_6')\}$ is called a
{\sl double-six} if
\begin{itemize}
\item $\ell_{i}$ and $\ell'_{i}$ are disjoint for all $i$, and
\item $\ell_{i}$ and $\ell'_{j}$ meet at a point whenever $i \neq j$.
  \end{itemize}

\begin{theorem}[Schl\"{a}fli]
  \label{theorem:schlafli} Each six on a smooth cubic surface has a
  unique extension to a double-six, and every smooth cubic surface
  contains $36$ double-sixes.
\end{theorem}

\subsubsection{Moduli spaces of cubic surfaces}
\label{sec:modulispacescubic}
We let $\M$ denote the {\sl coarse moduli space of cubic surfaces},
obtained by taking the GIT quotient of the linear system of cubic
surfaces in $\P^3$ by the action of $\Aut(\P^3) = \PGL_{4}(\k)$. By
the classical calculation of invariants \cite[], we know that $\M$ is
a $4$ dimensional variety isomorphic to the weighted projective space
$\P(1,2,3,4,5)$.

We let $\M^{\dagger}$ denote the {\sl moduli space
  of marked cubic surfaces} which parametrizes pairs
$(S, \underline{\ell})$ where $\underline{\ell}$ is a six on a smooth
cubic surface $S \subset \P^3$, taken up to the action of
$\Aut(\P^3)$.  There is a natural quasi-finite, dominant forgetful
morphism
\begin{align}
  \label{eq:modulispaces}
  \sigma: \M^{\dagger} \to \M\\
  (S, \underline{\ell}) \mapsto S. \nonumber
\end{align}

\autoref{theorem:schlafli} implies that
\begin{align}
  \label{eq:degreesigma}
  \deg (\sigma) = 72 \times 6! = 51840.
\end{align}


\begin{theorem}[Schubert]
  \label{theorem:Schubert} There are $20$ cuspidal cubic curves
  containing all of five general points in $\P^{2}$, and flexed at a
  sixth general point.
\end{theorem}


\subsection{Good families}
\label{sec:good-families}

Let $E$ denote a four dimensional $\k$-vector space, and hereafter fix
a {\sl general} degree $3$ form $F \in \Sym^{3}E^{\smvee}$.  Let
$$\Orb(F) \subset \P \Sym^{3}E^{\smvee} \simeq \P^{19}$$
denote the $\GL(E)$-orbit closure of the point $[F]$.  We let
$$\partial \Orb(F) = \Orb(F) \setminus \GL(E) \cdot [F]$$
denote the {\sl boundary}, i.e. the complement of the orbit of $[F]$
in its closure. $\Orb(F)$ is a $15$ dimensional variety whose singular
locus lies in the proper subvariety $\partial \Orb(F)$.  

\begin{definition}
\label{def:goodfamily}
Let $\pi :X \to B$ be a flat, projective morphism of relative
dimension $2$, with $B$ a $4$-dimensional scheme. We say $\pi$ is {\sl
  a good family of cubic surfaces} if:
\begin{itemize}
\item There exists a $\P^3$-bundle $\rho: P \to B$ and a closed
  embedding $\iota: X \hookrightarrow P$ embedding each fiber $X_b$,
  $b \in B$, as a cubic surface in $P_b$ and $\pi = \rho \circ \iota$,
\item There does {\sl not} exist a $b \in B$ such that the cubic
  surface $X_b$ is represented by a point in
  $\partial \Orb(F) \subset \P^{19}$.
\end{itemize}
\end{definition}

The base $B$ of a good family carries the {\sl natural} rank $4$
vector bundle
\begin{align}
  \label{eq:V}
  \V := \pi_{*}\omega_{X/B}^{-1}.
\end{align}
Furthermore, assuming the general fiber of $\pi$ is smooth, $B$ admits
a rational map
\begin{align}
  \label{eq:mu}
  \mu: B \dashrightarrow \M
\end{align}
to the coarse moduli variety of cubic surfaces
$\mathcal{M} := \Sym^{3}E^{\smvee} \sslash GL(E)$.




\begin{definition}
  \label{def:admissible} A length $6$ subscheme $Z \subset \P^{2}$
  with ideal sheaf $\mathcal{I}_{Z}$ is {\sl admissible} if:
  \begin{enumerate}
  \item $Z$ is curvilinear, i.e $Z$ is contained in a smooth curve,
  \item $h^{0}(\mathcal{I}_Z(3)) = 4$, and
  \item $Z$ is the scheme cut out by the vector space of cubics
    $H^0(\P^2, \mathcal{I}_Z(3))$.
  \end{enumerate}
\end{definition}

\begin{definition}
  \label{def:cubicZ} Let $Z \subset \P^2$ be an admissible length $6$
  subscheme.  The {\sl cubic surface associated to $Z$}, denoted
  $X_Z \subset \P^3$ is the image of the anti-canonical map
  $\kappa: \Bl_{Z}\P^2 \to \P^3$ induced by the linear system
  $H^0(\P^2,\mathcal{I}_Z(3))$.
\end{definition}

\begin{proposition}
  \label{prop:good} Let $Z \subset \P^2$ be an admissible, length $6$
  subscheme with $X_Z \subset \P^3$ its associated cubic surface. Then
  $\dim \Aut(\P^2,Z) > 0$ if and only if $\dim \Aut(\P^3,X_Z) > 0$.
\end{proposition}

\begin{proof} The implication
  $\dim \Aut(\P^2,Z) > 0 \implies \dim \Aut(\P^3,X_Z) > 0$ is simpler:
  Any automorphism $g \in \Aut(\P^2,Z)$ naturally lifts to an
  automorphism $g' \in \Aut \Bl_{Z}\P^2$. This latter automorphism
  must preserve the anti-canonical line bundle
  $\omega_{\Bl_{Z}(\P^2)}^{-1}$, and therefore induces an element in
  $\Aut(\P^3, X_Z)$. This element is nontrivial if $g$ is, because
  $\kappa: \Bl_{Z}(\P^2) \to X_Z$ is birational. The implication
  follows.


  The opposite implication
  $\dim \Aut(\P^3,X_Z) > 0 \implies \dim \Aut(\P^2,Z) > 0$ is more
  intricate. First, note that $\Bl_{Z}\P^2$ is a surface with finitely
  many $A_n$ singularities, thanks to the curvilinearity of $Z$; let
  $\tau: Y \to \Bl_{Z}\P^2$ denote its minimal desingularization.
  Since $A_n$ singularities are canonical, we conclude that
  $\tau^{*}(\omega_{\Bl_{Z}\P^2}) = \omega_{Y}$.  The composite
  $\kappa':= \kappa \circ \tau: Y \to X_{Z}$ can only contract a curve
  $C \subset Y$ to a singular point of $X_Z$.  This follows because   
  $\kappa'^{*}\omega_{X_{Z}} = \omega_{Y}$. Thus,
  $\kappa': \kappa'^{-1}(X_{Z}^{sm}) \to X_{Z}^{sm}$ is an
  isomorphism. Our claim is that $\kappa':Y \to X_{Z}$ is also the
  minimal desingularization of $X_Z$.

  Since $Y$ is {\sl some} desingularization of $X_Z$,
  $\kappa':Y \to X_{Z}$ factors through the minimal desingularization
  $\widetilde{X_{Z}} \to X_{Z}$ of $X_{Z}$; let
  $\alpha: Y \to \widetilde{X_{Z}}$ denote the induced map. $\alpha$
  is a birational morphism between smooth surfaces. Therefore, if it
  is not biregular, there must be a smooth $(-1)$ rational curve
  $E \subset Y$ contracted by $\alpha$.  Since $\omega_{Y}$ is pulled
  back from $\widetilde{X_{Z}}$, it follows that $\omega_{Y}|_{E}$ is
  trivial. This, along with $E \cdot E = -1$, contradicts the
  adjunction formula.  Hence $\alpha$ is an isomorphism.

  Every automorphism of a surface lifts uniquely to its minimal
  desingularization. Therefore, if $\dim \Aut(\P^3,X_Z) >0 $, it
  follows that $\dim \Aut(Y) >0$. Under this assumption, let
  $G \subset \Aut(Y)$ denote the connected component containing
  identity.

  The Picard group $\Pic(Y)$ is discrete, so the connected group $G$
  acts trivially on $\Pic(Y)$.  In particular, the line bundle $L$ on
  $\Bl_{Z}\P^2$ inducing the morphism $Y \to \Bl_{Z}\P^2 \to \P^2$ is
  preserved by $G$. Therefore, $G$ induces an action on
  $\P H^0(Y,L)= \P^2$.  This action renders the blow-down map
  $\Bl_{Z}\P^2 \to \P^2$ $G$-equivariant, from which it follows that
  the action of $G$ on $\P^2$ preserves the subscheme $Z$, which is
  what we sought to show.

  
\end{proof}

%-------------------------------------------------------------------------


\section{Family $B_1$}
\label{sec:first-test-family}

In this section, we construct a good family over $B_{1} = \P^{4}$.
Let $S$ denote the blow up of $\P^{2}$ at two points $y_1,y_2$. We let
$E_{1}, E_{2}$ denote the corresponding exceptional lines on $S$ and
we let $L \subset S$ denote the proper transform of a line in $\P^2$
passing through the two points $y_1,y_2$. Finally, we pick a general
conic through $y_{2}$ and let $C \subset S$ denote its proper
transform. Let $\omega_{S}^{-1}$ denote the anti-canonical line bundle
on $S$.  

\subsection{Construction of $B_1$}
\label{sec:construction-b_1}


The base of our first family is
\begin{align}
  \label{eq:B1}
  B_1 := \Hilb^{4}C \simeq \P^{4}.
\end{align}
$B_1$ carries a good family of cubic surfaces over it.  In short, to
each degree $4$ divisor $Z \subset C$, we assign the cubic surface
$X_{Z}$ which is the image of $\Bl_{Z}S$ under its anti-canonical
map. The following proposition justifies this construction:

\begin{proposition}
  \label{prop:B1} Let $Z \subset S$ be any length $4$ subscheme of $C$
  with ideal sheaf $\mathcal{I}_{Z} \subset \mathcal{O}_{S}$. Then
  \begin{enumerate}
  \item $h^{0}(S, \mathcal{I}_{Z} \otimes \omega_{S}^{-1}) = 4$, and
  \item $Z \subset S$ is the subscheme defined by the vector space of
    sections $H^{0}(S, \mathcal{I}_{Z} \otimes \omega_{S}^{-1})$.
  \end{enumerate}
\end{proposition}

\begin{proof}
  We translate (1) and (2) into geometric statements about the
  anti-canonical embedding $S \subset \P^{7}$.  The line bundle
  $\omega_{S}^{-1}(-C)$ on $S$ is basepoint free, as its divisor class
  is represented by proper transforms of lines through
  $y_{1} \in \P^2$, which become separated in the blow up.  The
  geometric consequence of this is: {\sl If
    $\langle C \rangle \simeq \P^5 \subset \P^7$ is the linear span of
    $C$, then $\langle C \rangle \cap S = C$, scheme theoretically.}

  Now $C$ is a rational normal quintic curve in its span
  $\langle C \rangle$. The following is a basic fact about rational
  normal curves: {\sl Every length $m \leq n$ subscheme of rational
    normal curve of degree $n$ spans a linear space of dimension $m-1$
    which, in turn, intersects the rational normal curve precisely at
    the length $m$ scheme.}  Both (1) and (2) follow directly from
  this.
\end{proof}

Let $\mathcal{Z} \subset B_1 \times C$ denote the universal
length $4$ subscheme, and consider $\mathcal{Z}$ as a closed subscheme
of the product $B_1 \times S$. Let $\pi_{1}, \pi_{2}$ denote the
projections of $B_1 \times S$ to its respective factors.
Then we set
\begin{align}
  \label{eq:X1}
  X' := \Bl_{\mathcal{Z}} \left( B_1 \times S \right).
\end{align}
Abusing notation, we continue to let $\pi_{1}$ and $\pi_{2}$ denote
the natural maps from $X'$ to $B_1$ and $S$, respectively.  We
let $E \subset X'$ denote the exceptional divisor over $\mathcal{Z}$.


Consider the line bundle
$\mathcal{L} := \pi_{2}^{*}\omega_{S}^{-1} (-E) =
\omega_{X'/B_{1}}^{-1}$ on $X'$.  By \autoref{prop:B1},
\begin{align}
  \label{eq:V1}
  \V := \pi_{1 *}\mathcal{L}
\end{align}
is a rank $4$ vector bundle on $B_1$, and the canonical evaluation map
\begin{align}
  \label{eq:eval1}
  \pi_{1}^{*}\V \to \mathcal{L}
\end{align}
is surjective.  Therefore, we obtain a map
\begin{align}
  \label{eq:X11}
  \kappa: X' \to \P \V^{\smvee}
\end{align}
whose image $X := \kappa(X')$ is a family of cubic surfaces
parametrized by $B_1$.  Let $\pi: X \to B_1$ denote the natural map.
By tracing through the construction, one can see that the vector
bundle $\V$ is equal to $\pi_{*} \omega_{X/B_1}'$, i.e. $\V$ is the
natural bundle for $\pi: X \to B_1$.

\subsection{Chern classes of $\V$}
\label{sec:chern-classes-b_1}

In order to calculate the Chern classes of $\V$, we express it as an
extension of a trivial rank $2$ vector bundle by the Lazarsfeld-Mukai
bundle on $B_1$ associated to the line bundle
$\omega_{S}^{-1}|_{C}$. By identifying the conic $C$ with $\P^{1}$,
this latter line bundle is $\O_{\P^{1}}(5)$ -- we will switch back and
forth between $C$ and $\P^{1}$ in what follows.  We also identify
$B_{1}$ with $\P^{4}$.


The trivial sub-bundle of $\V$ arises as follows: The fiber $\V_{[Z]}$
over a point $[Z] \in B_1$, corresponding to a subscheme
$Z \subset C$, is the vector space
$H^{0}(S, \mathcal{I}_{Z} \otimes \omega_{S}^{-1})$.  The latter
vector space contains the fixed $2$ dimensional vector space
$H^{0}(S, \omega_{S}^{-1}(-C))$.  This yields a $2$-dimensional
trivial subbundle of $\V$, whose quotient we denote by $\mathcal{K}$.


The bundle $\mathcal{K}$ is naturally identified with the kernel in
the exact sequence
\begin{align}
  \label{eq:LM}
  0 \to \mathcal{K} \to H^{0}(\P^1, \O_{\P^{1}}(5)) \otimes \O_{\P^4} \to \O_{\P^{1}}(5)^{[4]} \to 0 
\end{align}

Here, the rank four bundle $\O_{\P^{1}}(5)^{[4]}$ is the tautological
bundle whose fiber over a point $[Z] \in B_{1}$, corresponding to a
length $4$ subscheme $Z \subset \P^{1}$, is the vector space
$H^{0}(Z, \O_{\P^{1}}(5)|_{Z})$.  The map on the right side of
\eqref{eq:LM} is simply restriction of sections.

It is an easy exercise to show that
$\mathcal{K} \simeq \O_{\P^{4}}(-1) \oplus \O_{\P^{4}}(-1)$.  The
following lemma summarizes the relevant Chern class computations of
$\V$ -- for brevity's sake, we equate a $0$-cycle with its degree.

\begin{lemma}
  \label{lemma:chernB1}
  For the natural bundle $\V$ over $B_1= \P^{4}$, we have:
  \begin{align}\nonumber
    v_{1}^{4} = 16,\\\nonumber
    v_{1}^{2}v_{2} = 4,\\\nonumber
    v_{1}v_{3} =0,\\\nonumber
    v_2^{2} = 1,\\\nonumber
    v_{4} = 0.
  \end{align}
\end{lemma}

\begin{proof}
  By the Whitney sum formula, the Chern classes of $\V$ are
  $v_{1} = -2h$, $v_{2}=h^2$, and $v_{3}= v_{4}=0$, where $h$ denotes
  the hyperplane class on the projective space $B_1 = \P^{4}$.
\end{proof}









\subsection{Enumerating cubics in $B_1$}
\label{sec:enum-cubics-b_1}


\begin{figure}[!]
  \centering
%  \incfig{FamilyB1}
  \caption{Schematic for the family parametrized by $B_1$. The points
    $Z$ vary freely on the conic $C$, and we blow up the plane at the
    six points $Z \cup \{y_{1},y_{2}\}$.}
    \label{fig:FamilyB1}
\end{figure}


For $B_1$ to be of any use, we must {\sl a priori} know the degree of
the map to moduli $\mu: B_1 \dashrightarrow \M$.  First note that
$\mu$ factors through the quasi-finite, degree
$72 \times \frac{6!}{4!}$ cover $\M^{\dagger}/S_{4} \to \M$.  Here,
the symmetric group $S_4$ acts by permuting the last four lines in a
six $(\ell_{1}, \dots, \ell_{6})$ on a cubic surface.


The moduli map $\mu: B_{1} \dashrightarrow \M$ naturally factors
through a map $\mu': B_1 \dashrightarrow \M^{\dagger}/S_{4}$ by
sending a general subscheme $Z \subset C$ to the marked cubic surface
$(\Bl_{Z}S, E_1, E_2, E_{Z})$. Here, $E_Z$ denotes the union of four
exceptional lines above the points of $Z$.


Thus, in order to compute $\deg \mu$, we merely need to compute
$\deg \mu'$.

  \begin{lemma}
    \label{lemma:degreemudaggerB1}
    $\mu': B_1 \dashrightarrow \M^{\dagger}/S_{4}$ has degree $2$.
  \end{lemma}

  \begin{proof}
    Beginning with a general marked cubic surface
    $(X, \ell_{1}, \ell_2, \dots, \ell_{6})$ we can blow down the six
    to produce a map $\varphi: X \to \P^2$. The map $\varphi$ is only
    well-defined up to the action of $\Aut(\P^{2})$.  By composing
    with an automorphism, we may assume that
    $\varphi(\ell_{1}) = y_{1}$ and $\varphi(\ell_{2}) = y_{2}$.

    The $4$ dimensional subgroup $H \subset \Aut(\P^{2})$ fixing the
    points $y_{1}, y_{2}$ acts on the $\P^{4}$ of conics passing
    through $y_{2}$. Let $G \subset H$ denote the stabilizer of $C$,
    viewed as a conic through $y_{2}$.  We will now show that $|G|=2$,
    from which the lemma follows.

    Any automorphism $h \in H$ preserving $C$ must fix the three
    points $y_{1},y_{2}$ and the point
    $y_{3}:= (\overline{y_{1},y_{2}} \cap C) \setminus \{y_{2}\}.$
    Therefore, $h$ must also fix $p := \Pole(y_{2},y_{3}) \in \P^{2}$.
    At the same time, $h$ must preserve the un-ordered pair of points
    $\Polar(y_{1}) \subset C$. It is now a simple exercise to see that
    there is a unique involution on $\P^{2}$ fixing $p$, preserving
    $C$ and interchanging the points of $\Polar(y_{1})$. Hence
    $|G|=2$, as desired.
  \end{proof}




\subsection{Goodness}
\label{sec:goodness1}

\begin{proposition}
  \label{prop:B1good}
  The family $\pi: X \to B_1$ is good.
\end{proposition}

\begin{proof}
  We use \autoref{prop:good}. If $[Z] \in B_{1}$ is any point,
  corresponding to a length $4$ subscheme $Z \subset C$, we
  immediately get a length $6$ admissible subscheme (see \cite[] )
  $Z' \subset \P^2$ by the rule
  \begin{align*}
    Z' = (Z+y_2) \cup \{y_{1}\}.
  \end{align*}
  Here, we view $C$ as a conic in $\P^{2}$ containing $y_{2}$, and the
  sum in $Z + y_{2}$ is as divisors on $C$.  The cubic surface
  $X_{[Z]} = \pi^{-1}([Z])$ is the same as the cubic surface $X_{Z'}$
  associated to $Z'$.

  In order to use \autoref{prop:good}, we next verify that the group
  $\Aut(\P^{2},Z')$ is finite.  First observe that $C \subset \P^{2}$
  must be the unique conic containing the length $5$ subscheme
  $ Z + y_{2} \subset Z'$.  Therefore, any element
  $h \in \Aut(\P^{2},Z')$ must preserve $C$.  As $y_{1}$ does not lie
  on $C$, we conclude that $h(y_{1}) = y_{1}$, and therefore $h$ must
  preserve the pair of points $\Polar(y_{1}) \subset C$. Furthermore,
  if we assume $h$ also fixes $y_{2}$, which we can do by passing to
  the connected component of the identity element in
  $\Aut(\P^{2},Z')$, then $h$ would have to preserve three distinct
  points on $C$, from which finiteness follows.
\end{proof}


\subsection{First relation}
\label{sec:firstrelation}

The results in this section, taken together, yield the following
relation on the indetermined coefficients in \eqref{eq:P}:
  

\begin{align}
  \label{eq:relation1}
  16 \cdot a_{1^4} + 4 \cdot a_{1^2 2} + a_{2^2} = 72 \times \frac{6!}{4!} \times 2 = 4320.
\end{align}

%------------------------------------------------------------------------------------

\section{Family $B_2$}
\label{sec:family-b_2}


This section constructs and analyzes a good family over the base
$$B_2 := \overline{\M}_{0,7},$$
the moduli space of Deligne-Mumford stable $7$-pointed genus $0$
curves.  To avoid unnecessary accumulation of symbols, we reset
notation such as $C,S,X, \dots$ from previous sections, and start
anew.



Before diving into details, we summarize the construction. Let
$C \subset \P^{2}$ be a smooth conic with $7$ distinct marked points
$s_{0}, s_{1}, \dots, s_{5}, s_{\infty}$ on $C$, and let
$t \in \P^{2} \setminus C$ denote the intersection point of the
tangent lines of $C$ at $s_{0}$ and $s_{\infty}$, i.e.
$t = \Pole(s_{0},s_{\infty})$.  Then the scheme
$Z := \{s_{1}, \dots , s_{5}, t \}$ is an admissible length $6$
subscheme of $\P^{2}$, and its associated cubic surface $X_{Z}$ has a
finite automorphism group.  The main objective of this section is to
demonstrate how this construction {\sl can be extended over the
  boundary of $\overline{\M}_{0,7}$, yielding a good family of cubic
  surfaces.}

\subsection{Basics of $\overline{\M}_{0,7}$}
\label{sec:basics-overlinem_0-7}

We denote a general stable $7$-pointed genus zero curve by
$$(C, s_{0}, s_{1}, \dots, s_{5}, s_{\infty}),$$
and we let 
\begin{align}
  \label{eq:PM07}
  \varphi: \mathcal{C} \to \overline{\M}_{0,7}
\end{align}
denote the universal stable curve, with universal marking sections
denoted $$\sigma_{0}, \sigma_{1}, \dots, \sigma_{5}, \sigma_{\infty} \subset \mathcal{C}.$$

We let $\mathcal{L}$ denote the line bundle
$\O_{\mathcal{C}}(-\sigma_{\infty})|_{\sigma_{\infty}} \simeq
\omega_{\varphi}|_{\sigma_{\infty}}$ -- we view it as a line bundle on
$\overline{\M}_{0,7}$, as is customary.  Following standard notation,
we set $\psi_{\infty} := c_{1}\mathcal{L}$.  The sole enumerative
result about $\overline{\M}_{0,7}$ we will use in this section is:

\begin{theorem}[ ???2 ]
  \label{theorem:psi}
  $$\int_{\overline{\M}_{0,7}} \psi_{\infty}^{4} = 1.$$
\end{theorem}

\subsection{Construction of the family}
\label{sec:construction-familyB2}


We move on to the construction of our good family over $B_2$. Start
with the rank $2$ vector bundle
$$\mathcal{A} := \varphi_{*}\O_{\mathcal{C}}(\sigma_{\infty})$$ on
$\overline{\M}_{0,7}$.  From the fact that the line bundle
$\O_{\mathcal{C}}(\sigma_{\infty})$ restricts to $\mathcal{L}^{-1}$ on
$\sigma_{\infty}$ and $\O_{\sigma_{0}}$ on $\sigma_{0}$, it follows
that
\begin{align*}
  \mathcal{A} \simeq \O_{\overline{\M}_{0,7}} \oplus \mathcal{L}^{-1}.
\end{align*}

Next, put $\o{\mathcal{C}} := \P \mathcal{A}^{\smvee}$;
$\overline{\mathcal{C}}$ is a $\P^{1}$-bundle over
$\overline{\M}_{0,7}$ with structural map
$$\o{\varphi}: \o{\cC} \to \o{\M}_{0,7}$$ and the surjection
$\varphi^{*}\mathcal{A} \to \O_{\mathcal{C}}(\sigma_{\infty})$ induces
a {\sl contraction map}

\begin{align}
  \label{eq:contract}
  \gamma: \mathcal{C} \to \overline{\mathcal{C}}.
\end{align}

The effect of $\gamma$ on a stable curve $(C, s_0, \dots, s_{\infty})$
is to contract all irreducible components not containing the marked
point $s_{\infty}$, leaving the unique $\P^{1}$ containing the marked
point $s_{\infty}$.  This remaining $\P^{1}$ is the corresponding
fiber of the $\P^{1}$-bundle $\overline{\mathcal{C}}$.

Under $\gamma$, the sections $\sigma_{i}$ map to sections, which we
denote by $\overline{\sigma_{i}}$, of $\o{\varphi}$.  These sections
have the following properties:
\begin{itemize}
\item $\overline{\sigma}_{\infty}$ is disjoint from all other sections, and
\item in each fiber of $\o{\varphi}$, the union of all seven sections
  is supported on at least three points.
\end{itemize}




The next step is to invoke the relative $2$-veronese embedding
$$\iota: \overline{\mathcal{C}} \hookrightarrow \cP:= \P \Sym^{2}\mathcal{A}^{\smvee}.$$
We let $\rho: \cP \to \o{\M}_{0,7}$ denote the structural map of this
$\P^{2}$-bundle.  In this way, $\overline{\mathcal{C}} \subset \cP$ is
a family of smooth conics with $7$ marked points
$\overline{\sigma}_{0}, \dots, \overline{\sigma}_{\infty}$ satisfying
the two properties just mentioned above.

\begin{definition}
  \label{def:tau} Let $\tau \subset \cP$ denote the section obtained
  by taking, fiber-by-fiber, the point polar (with respect to the
  conic $\iota(\overline{\mathcal{C}})$) to the line spanned by
  $\overline{\sigma}_{0}$ and $\overline{\sigma}_{\infty}$ in each
  fiber.
\end{definition}

Observe that, since $\o{\sigma}_{0}$ and $\o{\sigma}_{\infty}$ are {\sl
    disjoint}, it follows that the three sections
  $\tau, \o{\sigma}_{0},$ and $\o{\sigma}_{\infty}$ are non-collinear
  in every fiber of $\rho: \cP \to \o{\M}_{0,7}$.

\begin{lemma}
  \label{lemma:tau}
  The section $\tau \subset \cP$ is induced by an inclusion of the
  form
  $${\mathcal{L}} \hookrightarrow \Sym^{2} \mathcal{A}^{\smvee}.$$
\end{lemma}

\begin{proof}
  Let $\mathcal{T} \hookrightarrow \Sym^{2} \mathcal{A}^{\smvee}$
  denote the line sub-bundle corresponding to the section $\tau$ -- we
  must show $\mathcal{T} \simeq \mathcal{L}$.

  The sections
  $\o{\sigma_{0}}$ and $\o{\sigma}_{\infty}$ correspond, respectively,
  to inclusions of the form
  \begin{align*}
    \O \hookrightarrow \Sym^{2} \mathcal{A}^{\smvee}, \\
    \mathcal{L}^{2} \hookrightarrow \Sym^{2} \mathcal{A}^{\smvee}. \\
  \end{align*}

  Since $\tau$, $\o{\sigma}_{0}$, and $\o{\sigma}_{\infty}$ are
  non-collinear in every fiber of $\rho$, we deduce that the induced
  map
  $$\mathcal{T} \oplus \mathcal{L}^{2} \oplus \O \to \Sym^{2}
  \mathcal{A}^{\smvee}$$ is an isomorphism.  Since
  $\Sym^{2} \mathcal{A}^{\smvee} \simeq \O \oplus \mathcal{L} \oplus
  \mathcal{L}^{2}$, the claim now follows by considering determinants.
\end{proof}


\begin{definition}
  \label{def:ZB2} We let $\Sigma \subset \iota(\o{\mathcal{C}})$
  denote the union of the sections
  $\o{\sigma}_{1}, \dots, \o{\sigma}_{5}$.  We let
  $\mathcal{Z} \subset \cP$ denote the (disjoint) union
  $\Sigma \bigsqcup \tau$.
\end{definition}

The reader can immediately check that $\mathcal{Z} \subset \cP$
registers an admissible length $6$ subscheme in each fiber of
$\rho$. In light of this, we let
\begin{align}
  \label{eq:VB2}
  \mathcal{V} := \rho_{*}(\mathcal{I}_{\mathcal{Z}} \otimes \omega_{\rho}^{-1});
\end{align}
$\V$ is the natural rank $4$ vector bundle for the family of cubic
surfaces $$\pi: X \to \o{\M}_{0,7}$$ obtained by taking, for each
fiber of $\rho$, the cubic associated to the admissible subscheme
$\mathcal{Z} \subset \cP$.

\begin{proposition}
  \label{prop:goodnessB2} The family $\pi: X \to \o{\M}_{0,7}$ is
  good.
\end{proposition}

\begin{proof}
  Apply \autoref{prop:good}, after noting that in
  each fiber of $\rho$ the subscheme
  $\Sigma \subset \iota(\o{\mathcal{C}})$ is supported on at least
  three points.
\end{proof}

\subsection{Chern classes of $\V$}
\label{sec:chern-classes-v2}

We must now determing the Chern polynomial of the vector bundle
$\V = \rho_{*}(\mathcal{I}_{\mathcal{Z}} \otimes
\omega_{\rho}^{-1})$. In light of the exact sequence
\begin{align}
  \label{eq:exact2}
  0 \to \V \to \rho_{*}\omega_{\rho}^{-1} \to \rho_{*}(\omega_{\rho}^{-1}|_{\mathcal{Z}}) \to 0
\end{align}
it clearly suffices to determine the Chern polynomials of the rank
$10$ bundle $\rho_{*}(\omega_{\rho}^{-1})$ and the rank $6$ bundle
$\rho_{*}(\omega_{\rho}^{-1}|_{\mathcal{Z}})$.

\begin{lemma}
  \label{lemma:rank10}
  The bundle $\omega_{\rho}^{-1}$ is isomorphic to
  $\O_{\mathcal{P}}(3) \otimes \rho^{*}\mathcal{L}^{3}$, and
  \begin{align}
    \label{eq:rnk10}
    \rho_{*}(\omega_{\rho}^{-1}) \simeq \Sym^{3}(\Sym^{2} \mathcal{A})\otimes \mathcal{L}^{3}.
  \end{align}
\end{lemma}

\begin{proof}
  It suffices to prove the first assertion, that
  $\omega_{\rho}^{-1} \simeq \O_{\mathcal{P}}(3) \otimes
  \rho^{*}\mathcal{L}^{3}$.  This follows directly from the relative
  Euler exact sequence for $\rho$:
  \begin{align}
    \label{eq:eulerexact}
    0 \to \Omega_{\mathcal{P}/\o{\M}_{0,7}} \to \rho^{*}(\Sym^{2}\mathcal{A}^{\smvee})(-1) \to \O_{\cP} \to 0.
  \end{align}
\end{proof}

\begin{corollary}
  \label{cor:chern10} The Chern polynomial of
  $\rho_{*}(\omega_{\rho}^{-1})$ is
  \begin{align}
    \label{eq:chern10}
    (1-9\psi_{\infty}^{2}t^{2})(1-4\psi_{\infty}^{2}t^{2})(1-\psi_{\infty}^{2}t^{2})^{2}.
  \end{align}
\end{corollary}

\begin{proposition}
  \label{prop:omegarestrictZ} The Chern polynomial of $\rho_{*}(\omega_{\rho}^{-1}|_{\mathcal{Z}})$ is
  \begin{align}
    \label{eq:rank6}
    (1+3\psi_{\infty}t) (1+2\psi_{\infty}t) (1+\psi_{\infty}t)  (1-\psi_{\infty}t)
  \end{align}
\end{proposition}

\begin{proof}
  First, because $\mathcal{Z}$ is the disjoint union of
  $\Sigma \subset \o{\mathcal{C}}$ and $\tau$, we get a direct sum
  decomposition
  \begin{align}
    \label{eq:dirsum}
    \rho_{*}(\omega_{\rho}^{-1}|_{\mathcal{Z}}) = \rho_{*}(\omega_{\rho}^{-1}|_{\Sigma}) \oplus \rho_{*}(\omega_{\rho}^{-1}|_{\tau}).
  \end{align}
  The line bundle $\rho_{*}(\omega_{\rho}^{-1}|_{\tau})$ is easy to
  identify -- in fact it is trivial.  Indeed, \autoref{lemma:tau}
  implies that $\O_{\cP}(1)|_{\tau} \simeq \mathcal{L}^{-1}$, and the
  first claim in \autoref{lemma:rank10} then implies
  $\rho_{*}(\omega_{\rho}^{-1}|_{\tau}) \simeq \mathcal{L}^{-3}
  \otimes \mathcal{L}^{3} = \O_{\o{\M}_{0,7}}$.

  Thus, the Chern polynomial we need is the same as the Chern
  polynomial of $\rho_{*}(\omega_{\rho}^{-1}|_{\Sigma})$ -- we spend
  the rest of the proof extracting it.  Since
  \begin{align*}
    \rho_{*}(\omega_{\rho}^{-1}|_{\Sigma}) = \o{\varphi}_{*}(\iota^{*}\omega_{\rho}^{-1}|_{\Sigma})
  \end{align*}
  we may instead work within the $\P^{1}$-bundle $\o{\cC}$, where
  $\Sigma$ is a divisor.  Since
  $$\iota^{*}\O_{\cP}(1) = \O_{\o{\cC}}(2) = \O_{\o{\cC}}(2\o{\sigma}_{\infty}),$$
  and since $\Sigma$ is disjoint from $\o{\sigma}_{\infty}$, it follows that
  $$\omega_{\rho}^{-1}|_{\Sigma} \simeq \o{\varphi}^{*}(\mathcal{L}^{3})|_{\Sigma},$$ and therefore that
  $$\rho_{*}(\omega_{\rho}^{-1}|_{\Sigma}) \simeq \o{\varphi}_{*}(\o{\varphi}^{*}(\mathcal{L}^{3})|_{\Sigma}) = (\o{\varphi}_{*}\O_{\Sigma})\otimes \mathcal{L}^{3}.$$

  This means we must find the Chern polynomial of
  $\o{\varphi}_{*}\O_{\Sigma}$.  For this, we use a little trick.  In
  the $\P^{1}$-bundle $\o{\cC}$, every section $\o{\sigma}_{i}$ other
  than $\o{\sigma}_{\infty}$ is linearly equivalent to any other such
  section -- we leave it to the reader to deduce this from the
  observation that $\o{\sigma}_{\infty}$ is disjoint from all other
  sections $\o{\sigma}_{i}$.  So, there exists a one-parameter family
  of divisors interpolating between $\Sigma$ and, say, the divisor
  $5\o{\sigma}_{0}$.  Since the Chow ring of $\o{\M}_{0,7}$ is
  finitely generated, it follows that the Chern polynomials of
  $\o{\varphi}_{*}(\O_{\Sigma})$ and
  $\o{\varphi}_{*}(\O_{5 \o{\sigma}_{0}})$ are identical.

  In order to identify the Chern polynomial of
  $\o{\varphi}_{*}(\O_{5 \o{\sigma}_{0}})$, we iteratively use the sequences
  $$0 \to \mathcal{I}_{\o{\sigma}_{0}}^{n-1}/\mathcal{I}_{\o{\sigma}_{0}}^{n} \to \O_{n\o{\sigma}_{0}} \to \O_{(n-1)\o{\sigma}_{0}} \to 0$$ along with the isomorphism of line bundles
  $$\mathcal{I}_{\o{\sigma}_{0}}^{n-1}/\mathcal{I}_{\o{\sigma}_{0}}^{n} \simeq (\mathcal{I}_{\o{\sigma}_{0}}/\mathcal{I}_{\o{\sigma}_{0}}^{2})^{\otimes n-1}$$ to deduce that
  \begin{align}
    \label{eq:rk5}
    c(\o{\varphi}_{*}(\O_{5 \o{\sigma}_{0}})) = c(\O_{\o{\M}_{0,7}})c(\mathcal{I}_{\o{\sigma}_{0}}/\mathcal{I}_{\o{\sigma}_{0}}^{2})c((\mathcal{I}_{\o{\sigma}_{0}}/\mathcal{I}_{\o{\sigma}_{0}}^{2})^{2})\cdots c((\mathcal{I}_{\o{\sigma}_{0}}/\mathcal{I}_{\o{\sigma}_{0}}^{2})^{4}).
  \end{align}

  It remains to identify the conormal bundle
  $\mathcal{I}_{\o{\sigma}_{0}}/\mathcal{I}_{\o{\sigma}_{0}}^{2}$ and
  its relationship with the class $\psi_{\infty}$. For this, we leave
  it to the reader to verify: {\sl if a $\P^{1}$-bundle has two
    disjoint sections, then their conormal bundles are inverse to
    each other.}  Therefore,
  $$c(\o{\varphi}_{*}(\O_{5 \o{\sigma}_{0}})) = c(\O)c(\mathcal{L}^{-1})c(\mathcal{L}^{-2})c(\mathcal{L}^{-3})c(\mathcal{L}^{-4}).$$

  The proposition now follows by tracing backwards, starting with this
  latest identity.
\end{proof}

\begin{corollary}
  \label{cor:chernpolyV2} The Chern polynomial of $\V$ is
  $(1-3\psi_{\infty}t)(1-2
  \psi_{\infty}t)(1-\psi_{\infty}^{2}t^{2})$. Therefore,
  \begin{align}
    \label{eq:chernV2}
    v_1 = -5\psi_{\infty}\\
    v_2 = 5 \psi_{\infty}^{2}\\ \nonumber
    v_3 = 5 \psi_{\infty}^{3}\\ \nonumber
    v_{4} = -6 \psi_{\infty}^{4}.\\ \nonumber
  \end{align}
\end{corollary}

\subsection{Enumerating cubics in $B_{2}$}
\label{sec:enum-cubics-b_2}


\begin{figure}[!]
    \centering
%    \incfig{FamilyB2}
    \caption{Schematic for our second family. We blow up the plane at
      the points $\{s_{1}, \dots, s_{5}, t\}$.}
    \label{fig:FamilyB2}
\end{figure}


Now that we've identified the $v_{i}$, we need only establish
$\deg \mu$, where $\mu: B_{2} \dashrightarrow \M$ is the map to
moduli.  Clearly, $\mu$ factors through
$\mu':B_{2} \dashrightarrow \M^{\dagger}$: $\mu'$ sends a general
$7$-pointed stable curve $(C,s_0,s_1, \dots, s_5, s_{\infty})$ to the
marked cubic surface
$(Bl_{\{s_1, \dots, s_{5}, t\}}\P^{2}, E_1, \dots, E_5, E_t)$. Here,
$C$ is embedded in the plane as a conic, and $E_i$ denotes the
exceptional divisor over the corresponding point.

Clearly, $\deg \mu = 51840 \times \deg \mu'$.  Thus our analysis will
be complete once we show:

\begin{lemma}
  \label{lemma:muprime2} $\deg \mu' = 2.$
\end{lemma}

\begin{proof}
  We need to show that a general marked cubic surface
  $(X, \ell_{1}, \dots, \ell_{6}) \in \M^{\dagger}$ has two preimages
  in $\o{\M_{0,7}}$.
  
  
  Starting with $(X, \ell_{1}, \dots, \ell_{6})$, let
  $(s_1, \dots, s_5, t) \subset \P^{2}$ denote the images of
  $\ell_{i}$ under the map blowing down the six, well defined up to
  $\Aut \P^{2}$.  The five points $s_{i}$ lie on a unique smooth conic
  $C$ on which there is a unique un-ordered pair of points $\{x,y\}$
  whose tangent lines intersect at $t$.  The two stable curves
  $(C,x,s_1, \dots, s_5, y)$ and $(C,y,s_1, \dots, s_5, x)$ are then
  the two claimed preimage points in $\o{\M}_{0,7}$, proving the
  lemma.
\end{proof}


\subsection{Second relation}
\label{sec:secondrelation}

By assembling the information in this section, using
\autoref{theorem:psi} to compute the various monomials in the $v_{i}$,
we find our second relation:
\begin{align}
  \label{eq:relation2}
  625 a_{1^{4}} + 125 a_{1^{2}2} - 25a_{13} + 25a_{2^2} - 6 a_{4} = 2 \times 51840.
\end{align}

%------------------------------------------------------------------------------


\section{Family $B_3$}
\label{sec:family-b_3}


\begin{figure}[!]
    \centering
%    \incfig{FigureB3}
    \caption{Third family schematic. The point $\nu(s_{\infty})$ is
      the point at infinity. It, along with
      $\nu(s_{i}), i=1, \dots, 5$ are to be blown up.}
    \label{fig:FigureB3}
\end{figure}


Let us reset notation once again.  Our third family is a variant of
the family in the previous section \autoref{sec:family-b_2}.  It is
parametrized by the base variety
$$B_{3} := \hM_{0,7}.$$

Here, $\hM_{0,7}$ is a particular Hassett space parametrizing
${\bf w}$-stable $7$-pointed genus $0$ curves, where
$${\bf w} = \left(1-\epsilon, \epsilon, \dots, \epsilon, 1-3 \epsilon \right), 0 < \epsilon \ll 1$$
is the weight vector on the marked points
$(s_{0}, s_{1}, \dots, s_{5}, s_{\infty})$ on a genus $0$ curve.

Before getting buried in details, we explain the basic construction of
the family of cubic surfaces over the interior, $\M_{0,7}$.  Begin
with $(C, s_{0}, s_{1}, \dots, s_{5}, s_{\infty})$, a smooth rational
curve with $7$ distinct marked points, and choose a map
$$\nu: C \to \P^{2},$$ birational onto its image
$\nu(C)$, which is a {\sl cuspidal cubic curve with cusp point
  $\nu(s_{0})$ and unique flex point $\nu(s_{\infty})$.}  Let
$Z = \nu(\{s_{1}, \dots, s_{5}, s_{\infty}\}) $, and now assign the
cubic surface $\Bl_{Z}\P^{2}$.  The main objective of this section is
to demonstrate that this assignment extends over the boundary of
$\hM_{0,7}$, yielding a good family of cubic surfaces from which we
can extract one more relation.

Before starting the analysis, we make some remarks: The need for using
the alternate compactification $\hM_{0,7}$ ultimately traces to the
requirement that our family must be good.  There are several possible
variants of the construction, but the reader will eventually realize
that, very often, there will be a few non-good fibers present.


Finally, in the interest of brevity, {\sl we assume the reader is
  familiar with the intersection theory of $\o{\M}_{0,n}$, as
  explained in, for example \cite[??].}  Therefore, we will at time
translate intersection products on $\hM_{0,7}$ into calculations on
$\o{\M}_{0,7}$, and then leave it to the reader to check the results
there.

\subsection{The spaces $\hM_{0,n}$}
\label{sec:spaces-hm_0-n}

In order to carry out the calculations needed for our third relation,
we need to investigate the basic geometry of a series of
Hassett-spaces which we denote by $\hM_{0,n}$.  We do that in this
section.

\begin{definition}
  \label{def:hM0n} For each $n \geq 4$, we define $\hM_{0,n}$ to be
  the compactification of $\M_{0,n}$ by ${\bf w}$-stable, $n$-pointed,
  genus $0$ curves, where
  $${\bf w} = \left(1- \epsilon, \epsilon, \dots, \epsilon, 1-(n-4)\epsilon \right), 0 < \epsilon \ll 1.$$
\end{definition}

We will denote a ${\bf w}$-stable curve by
$(C, s_{0}, s_{1}, \dots, s_{n-2}, s_{\infty})$.  The indices are
meant to indicate that the two points $s_{0}$ and $s_{\infty}$ are two
``more interesting'' points, as will become apparent in the next
paragraph.

The pointed curves $(C, s_{0}, s_{1}, \dots, s_{n-2}, s_{\infty})$
parametrized by $\hM_{0,n}$ are actually very simple to describe:

\begin{itemize}
\item $C$ is a {\sl smooth} rational curve,
\item $s_{0}$ and $s_{\infty}$ must be distinct,
\item at most one of the $s_{i}$, $i=1, \dots, n-2$ can coincide with
  $s_{0}$,
\item at most $n-4$ of the $s_{i}$, $i=1, \dots, n-2$ can coincide
  with $s_{\infty}$.
\item there are no constraints on how the points $s_{i}$,
  $i=1, \dots, n-2$ can coincide with each other.
\end{itemize}
Therefore, the universal curve $$\varphi: \cC \to \hM_{0,n}$$ is in
fact a $\P^{1}$-bundle, the projectivization of a particular rank $2$
vector bundle $\mathcal{W}$ on $\hM_{0,n}$. We let
$\sigma_{i} \subset \cC$,
$i \in \left\{0,1,2, \dots, n-2, \infty\right\}$ denote the universal
sections providing the marked points, and we denote by
\begin{align}
  \label{eq:stab}
  \zeta: \overline{\M}_{0,n} \to \hM_{0,n}
\end{align}
the stabilization morphism.  We will sometimes use $\zeta$ to
pull-back enumerative calculations on $\hM_{0,n}$ back to
$\overline{\M}_{0,n}$, where they are better understood.

\subsubsection{Divisor classes on $\hM_{0,n}$}
\label{sec:divisor-classes-hm_0}

Certain divisors on $\hM_{0,n}$  play an essential role in our
computations, so we take this section to introduce them.

\begin{definition}
  \label{def:psihats} We let $\hL_{0}, \hL_{\infty}$ denote the line
  bundles
  $\omega_{\varphi}|_{\sigma_{0}},
  \omega_{\varphi}|_{\sigma_{\infty}}$, respectively, and let
  $\hpsi_{0}, \hpsi_{\infty} \in A^{1}(\hM_{0,n})$ denote their first
  Chern classes.
\end{definition}

Since the sections $\sigma_{0}$ and $\sigma_{\infty}$ are disjoint, it
follows that
\begin{align}
  \label{eq:opposites}
  \hpsi_{0} + \hpsi_{\infty} = 0.
\end{align}


Finally, we indicate the simple relationship between $\hpsi_{0}$ on
$\hM_{0,n}$ with $\psi_{0}$ on $\o{\M}_{0,n}$.  Let
$\delta_{\{i,0\}}, i=1, \dots, n-2$ denote the boundary divisor in
$\o{\M}_{0,n}$ whose general point corresponds to the union of two
marked $\P^{1}$'s, one of which contains only $0$-th and $i$-th marked
points.



Then,
\begin{align}
  \label{eq:relationpsis}
\zeta^{*}(\hpsi_{0}) = \psi_{0} - \sum_{i \in \{1, \dots, n-2\}} \delta_{\{i,0\}}.  
\end{align}













\begin{definition}
  \label{def:boundary}
  For each distinct pair of indices
  $i,j \in \{0, 1, 2, \dots, n-2, \infty \}$, we define
  $\Delta_{i,j} \subset \hM_{0,n}$ to be the divisor over which the
  sections $\sigma_{i}$ and $\sigma_{j}$ intersect, and we set
  $$\Delta_{\infty} = \sum_{i \in \{1, \dots, n-2 \}}
  \Delta_{i,\infty}.$$
\end{definition}

The specific boundary divisors $\Delta_{i, \infty}$, where the section
$\sigma_{i}$ collides with $\sigma_{\infty}$ will play an especially
important role in our calculations will be important for us, so we
briefly collect some observations about them:
\begin{itemize}\label{item:boundaryprops}
\item Each boundary divisor $\Delta_{i,\infty} \subset \hM_{0,n}$ is
  naturally isomorphic to $\hM_{0,n-1}$ via a map
  $$\eta_{i}: \hM_{0,n-1} \to \hM_{0,n}$$ that sends
  $(C, s_{0}, \dots, s_{n-3}, s_{\infty})$ to
  $(C, s_{0}, \dots, s_{i-1}, s_{i}=s_{\infty}, s_{i+1} \dots,
  s_{\infty})$.

  In the same way, if $i \neq j$, then the intersection
  $\Delta_{i,\infty} \cap \Delta_{j, \infty}$ is isomorphic to
  $\hM_{0,n-2}$ via a similarly defined map
  $$\eta_{i,j}: \hM_{0,n-2} \to \hM_{0,n},$$ and so on.  This provides a
  nice inductive structure among the Hassett spaces $\hM_{0,n}$.
  
\item Under $\eta$, we clearly have $\eta^{*}\hpsi_{0} = \hpsi_{0}$
  and $\eta^{*}\hpsi_{\infty} = \hpsi_{\infty}.$
\item We have:
\begin{align}\label{eq:pullbackdelta}
  \eta^{*}([\Delta_{i,\infty}]) = -\hpsi_{\infty} \in A^{1}(\hM_{0,n-1}).
\end{align}
To see this, first note that
$\Delta_{i, \infty} = \varphi_{*}(\sigma_{i}\sigma_{\infty})$, and so
$\eta^{*}([\Delta_{i,\infty}]) = \varphi_{*}(\sigma_{\infty}^{2})$,
where the latter equation is happening in the divisor class group of
$\hM_{0,n-1}$.  But
$\varphi_{*}(\sigma_{\infty}^{2}) = - \hpsi_{\infty}$, and the claim
follows.
\end{itemize}


\subsubsection{Computing integrals in $\hM_{0,n}$}

These observations are enough to compute, using an inductive method,
any top degree integral involving $\hpsi_{0}, \hpsi_{\infty},$ and the
boundary divisors $\Delta_{i, \infty}$.

\begin{example}
  \label{ex:toppsi}
  Let's first indicate how to
  compute $$\int_{\hM_{0,n}}\hpsi_{0}^{n-3}.$$ Using
  \eqref{eq:relationpsis}, this amounts to computing
  $$\int_{\o{\M}_{0,n}} \left(\psi_{0} - \sum_{i \in \{1, \dots, n-2\}} \delta_{\{i,0\}}\right)^{n-3}.$$
  Now, in the Chow ring of $\o{\M}_{0,n}$, recall that:
  \begin{itemize}
  \item $\psi_{0} \cdot \delta_{\{i,0\}} = 0$, 
  \item $\delta_{\{i,0\}} \cdot \delta_{\{j,0\}} = 0$ for $i \neq j$,
    and
  \item $\int_{\o{\M}_{0,n}}\delta_{\{i,0\}}^{n-3} = (-1)^{n}$, for
    all $i$, and
  \item $\int_{\o{\M}_{0,n}} \psi_{0}^{n-3} = 1$.
  \end{itemize}

  Using these, we deduce:
  \begin{align}
    \label{eq:hpsitop}
    \int_{\hM_{0,n}}\hpsi_{0}^{n-3} = -(n-3). 
  \end{align}
\end{example}

Next, we do an example involving $\Delta_{i, \infty}$'s.

\begin{example}
  Consider
  $$\int_{\hM_{0,7}} \Delta_{i, \infty}^{2}\Delta_{j, \infty}^{2},$$
  where $i \neq j$.  Using the parametrization
  $\eta_{i,j} : \hM_{0,5} \to \Delta_{i,\infty} \cap
  \Delta_{j,\infty}$, we see that this integral is the same as
  $$\int_{\hM_{0,5}}\eta_{i,j}^{*}[\Delta_{i, \infty}] \cdot \eta_{i,j}^{*}[\Delta_{j, \infty}],$$
  which in turn is equal to
  $$\int_{\hM_{0,5}}(-\hpsi_{\infty})^{2} = \int_{\hM_{0,5}}\hpsi_{0}^{2} = -2,$$
  as seen by combining \eqref{eq:hpsitop} with previous observations
  on how boundary divisors pull back along $\eta_{i}$ maps.
\end{example}

\subsection{Construction of the family over $\hM_{0,7}$}
\label{sec:constr-family-hM07}

Let us proceed to the formal construction of our family, starting with
the universal curve
$$\varphi: \cC \to \hM_{0,7}.$$
The idea is to map $\cC$ to a relative cuspidal cubic curve in a
$\P^{2}$-bundle so that $\sigma_{0}$ becomes the cusp point and
$\sigma_{\infty}$ becomes the flex point.


Consider the rank $2$
vector bundle $\mathcal{A} = \varphi_{*}\O_{\cC}(\sigma_{\infty})$ --
since $\sigma_{\infty}$ and $\sigma_{0}$ are disjoint, and since
$\O_{\cC}(\sigma_{\infty})|_{\sigma_{\infty}} = \hL_{\infty}^{-1}$ and
$\O_{\cC}(\sigma_{0})|_{\sigma_{0}} \simeq \O_{\hM_{0,7}}$, it follows
that
$$\mathcal{A} \simeq \O_{\hM_{0,7}} \oplus \hL_{\infty}^{-1}.$$
We emphasize that the above direct sum decomposition is induced by
applying $\varphi_{*}$ to the restriction map
$\O_{\cC}(\sigma_{\infty}) \to \O_{\cC}(\sigma_{\infty})|_{\sigma_{0}
  \cup \sigma_{\infty}}.$

  Next, consider the subbundle
  \begin{align}
    \label{eq:S}
    \mathcal{S} = :\O_{\hM_{0,7}} \oplus \hL_{\infty}^{-2} \oplus \hL_{\infty}^{-3} \subset \Sym^{3} \mathcal{A} = \O_{\hM_{0,7}} \oplus \hL_{\infty}^{-1} \oplus \hL_{\infty}^{-2} \oplus \hL_{\infty}^{-3},
  \end{align}
  and denote by
  \begin{align}
    \label{eq:nu}
    \nu: \cC \to \P \mathcal{S}^{\smvee}
  \end{align}
  the $\hM_{0,7}$-morphism induced by the surjection
  $\varphi^{*}\mathcal{S} \to \O_{\cC}(3\sigma_{\infty})$. Denote by
  $\rho: \P \mathcal{S}^{\smvee} \to \hM_{0,7}$ the $\P^{2}$-bundle
  structural map.  Fiber-by-fiber, the effect of $\nu$ is to map the
  fibers of $\cC$ to a cuspidal cubic curve, with $\nu(\sigma_{0})$
  the cusp point and $\nu(\sigma_{\infty})$ the flex point.

  We denote by $\mathcal{Z} \subset \cC$ the union of the six sections
  $\sigma_{1}, \dots, \sigma_{\infty}$.  Because at most one of the
  sections $\sigma_{i}, i \in \{1, \dots, \infty\}$ is allowed to
  intersect $\sigma_{0}$ (which becomes the cusp under $\nu$), it
  follows that $\nu|_{\mathcal{Z}}:\mathcal{Z} \to \nu(\mathcal{Z})$
  is an isomorphism. As will be demonstrated in
  \autoref{sec:goodness3}, every member of the family of length $6$
  subschemes $\nu(\mathcal{Z}) \subset \P \mathcal{S}^{\smvee}$ is
  admissible, and has a {\sl good} associated cubic surface.
  Therefore, our attention is brought upon the natural rank $4$ bundle
  \begin{align}
    \label{eq:V3}
    \V := \rho_{*}\left(\omega_{\rho}^{-1} \otimes \mathcal{I}_{\nu(\mathcal{Z})}\right)
  \end{align}
  of our claimed good family $$\pi: X \to \hM_{0,7},$$ with
  $X := \Bl_{\nu(\mathcal{Z})}\P \mathcal{S}^{\smvee}$.


\subsection{Goodness}
\label{sec:goodness3}

\begin{proposition}
  \label{prop:goodcusp}
  Let $C \subset \P^{2}$ be a cuspidal cubic, and let
  $s_{i} \in C, i \in \{1, \dots, 5, \infty \}$ be six, not
  necessarily distinct, points satisfying:
  \begin{enumerate}
  \item At most one $s_{i}$ is the cusp point,
  \item $s_{\infty}$ is the flex point of $C$, and
  \item at most three $s_{i}, i \leq 5$ are equal to $s_{\infty}$.
  \end{enumerate}
  Then the length $6$ subscheme $Z := \sum_{i}s_{i}$, interpreted as a
  divisor on $C$, is admissible and $\Aut(\P^{2},Z)$ is finite. NOT TRUE!!!?
\end{proposition}

\begin{proof}
  
\end{proof}

  

%-------------------------------------------------------------------------



\section{Family $B_4$}




\label{sec:family-b_4}

We construct our third family $B_3$ be starting with a quintic del
Pezzo surface $S$, which is the blow up of the plane at four general
points, and then ``blowing up all pairs of points on $S$.''

The surface $S$ has $10$ exceptional lines $L_{1}, \dots, L_{10}$.  We
let $S^{[2]}$ denote the Hilbert scheme of pairs of points on $S$.
$S^{[2]}$ contains $10$ disjoint $\P^{2}$'s,
$\Lambda_1, \dots, \Lambda_{10}$, namely the Hilbert schemes of pairs
of points contained in the lines $L_{i}$. Let
$\Lambda = \cup_{i}\Lambda_{i}$.

\subsection{Construction of $B_4$}
\label{sec:construction-b_4}

Let $Z \subset S^{[2]} \times S$ denote the universal degree two
subscheme; $Z$ is smooth and the first projection
$\pi_1 : Z \to S^{[2]}$ is finite, flat, and has degree two.



First, observe that the open set
\begin{align}
  \label{eq:UB3}
  U := S^{[2]} \setminus \Lambda
\end{align}
supports a good family of cubic surfaces, namely
\begin{align}
  \label{eq:XUB3}
  X^{\circ} := \Bl_{Z}(S^{[2]} \times S).
\end{align}

(We encourage the reader to check that $X^{\circ}/U$ is indeed good.)

However, when we extend $X^{\circ}$ in the natural way over $S^{[2]}$
the family acquires bad fibers, namely unions of quadrics and planes,
and therefore we must modify $S^{[2]}$ along $\Lambda$.

Let
\begin{align}
  \label{eq:S2tilde}
  \widetilde{S^{[2]}} := \Bl_{\Lambda}S^{[2]}
\end{align}
denote the blow-up with blow-down map
$\beta: \widetilde{S^{[2]}} \to S^{[2]}$.  Let
$E_{i} = \beta^{-1}(\Lambda_{i})$, $i=1, \dots, 10$ denote the
components of the of the exceptional divisor of the blow-up; $\beta$
expresses $E_{i}$ as a $\P^{1}$-bundle over $\Lambda_{i}$.

Next, let $\widetilde{Z} := \widetilde{S^{[2]}} \times_{S^{[2]}} Z$
denote the fiber-product; $\widetilde{Z}$ is a closed subscheme of the
product $\widetilde{S^{[2]}} \times S$, and is a finite, flat, degree
two cover of $\widetilde{S^{[2]}}$ under the first projection
$\tilde{\pi}_{1} : \widetilde{Z} \to \widetilde{S^{[2]}}.$

Inside the product $\widetilde{S^{[2]}} \times S$ we define the
smooth, closed subschemes
\begin{align}
  \label{eq:Fi}
  F_{i} := E_{i} \times L_{i};
\end{align}
we let $F := \cup_{i}F_{i}$.  We perform one more blow-up to get 
\begin{align}
  \label{eq:Xtilde}
  Y := \Bl_{F}\widetilde{S^{[2]}} \times S.
\end{align}
The natural projection $ Y \to \widetilde{S^{[2]}}$ is a family of
surfaces with simple normal crossing singular fibers over $E_{i}$.
The next lemma is critical.

\begin{lemma}
  The closed immersion
  $\widetilde{Z} \subset \widetilde{S^{[2]}} \times S$ lifts to a
  closed immersion $\widetilde{Z} \subset Y$.
\end{lemma}

\begin{proof}
  The main observation is that we have an equality of schemes
  \begin{align}
    \label{eq:Cartier}
    F_i \cap \widetilde{Z} = \widetilde{\pi}_{1}^{-1}(E_i) \cap \widetilde{Z}
  \end{align}
  for all $i=1, \dots, 10$, {\sl and therefore $F \cap \widetilde{Z}$
    is a Cartier divisor on $\widetilde{Z}$}. The claim now follows
  from a basic property of blow-ups.
\end{proof}



We are close to describing a family of cubic surfaces over
$\widetilde{S^{[2]}}$.  We first let
\begin{align}
  \label{eq:XB3}
  Y' := \Bl_{\widetilde{Z}}Y
\end{align}
denote the blow-up,
\begin{align}
  \label{eq:phiB3}
  \varphi: Y' \to Y
\end{align}
the blow-down map, and
\begin{align}
  \label{eq:piB3}
  \pi: Y' \to \widetilde{S^{[2]}}
\end{align}
the natural projection. Finally, let $F' \subset Y'$ denote the
exceptional divisor $\varphi^{-1}(F)$.
\begin{proposition}
  \label{proposition:resolv-b3}
  There exists a line bundle $\mathcal{L}$ on $Y'$ which induces a
  rational map
  \begin{align*}
    \rho: Y' \dashrightarrow \P(\pi_{*}\mathcal{L})
  \end{align*}
  whose image $X \subset \P(\pi_{*}\mathcal{L})$ is a good family of
  cubic surfaces over $\widetilde{S^{[2]}}$ with natural bundle
  $V = \pi_{*} \mathcal{L}$.
\end{proposition}

(By image, we mean ``closure of image''.)

(This one I still am confident in -- one simple blow-up resolves the situation.)


\subsection{Relation $B_4$}
\label{sec:relation-b_4}

The relation we get from family $B_3$ ultimately appears to be:
\begin{align}
  \label{eq:relationB4}
  6a_{1^{4}} + 21a_{1^{2}2}+6a_{13}+16a_{2^{2}}+a_{4} = 36 \times 6! = 25920.
\end{align}

(There is an 80 percent chance I made a mistake in the long
calculation, so be careful.)

\section{Isotrivial families}
In this section, we exhibit a number of good isotrivial families.  All
of these families will arise from a cubic surface $X$ with
automorphism group $\mathbb G_m$.  To make sure that these families
are good, we must prove that $X$ does not lie in the closure of the
orbit of a generic cubic surface.  If $X$ is a semi-stable point of
the $\SL_4$ action on the space of cubics $\P^{19}$, then this follows
from GIT.  We now explain an extension of this idea that allows for
even more examples.

Let $V$ be a 4-dimensional vector space.  Consider the natural action
of $\SL(V)$ on $\P(\Sym^3V) \times \P V$ (here $\P$ denotes the space
of one-dimensional subspaces).  That is, consider the action of
$\SL_4$ on the set of pairs $(X, H)$, where $X \subset \P^3$ is a
cubic surface and $H \subset \P^3$ is a hyperplane.  For a pair of
positive integers $a, b$, the action lifts uniquely to an action on
the ample line bundle $\mathcal O(a)\boxtimes \mathcal O(b)$; the GIT
of the resulting linearized action depends only on the ratio
$t = a/b$.

\begin{proposition}\label{prop:nolimit}
  Let $t$ be a positive rational number.
  Suppose $S \subset \P^3$ is a cubic surface such that for every hyperplane $H$, the pair $(S,H)$ is $t$-stable.
  Let $X \subset \P^3$ be a cubic surface whose stabilizer group $T \subset \PGL(4)$ is reductive.
  Let $T_0 \circ T$ be the connected component of the identity, and assume that there exists a $T_0$-fixed hyperplane $H \subset \mathbb P^3$ such that the pair $(X, H)$ is $t$-semistable.
  Then $X$ does not lie in the closure of the $\PGL_4$-orbit of $S$.
\end{proposition}

For the proof, we need a lemma.
\begin{lemma}\label{prop:oneparam}
  Let \(U\) be a smooth scheme over $\k$ with an action of a linear algebraic group $G$.
  Let $x \in U(\k)$ be a point whose stabilizer group $T \subset G$ is reductive.
  If $x$ lies in the closure of the $G$-orbit of a point $s \in U(\k)$, then there exists a one-parameter subgroup $\lambda \colon \mathbb G_m \to T$ and a point $s'$ in the $G$-orbit of $s$ such that
  \[ x = \lim_{t \to 0} \lambda(t) \cdot s'.\]
\end{lemma}
\begin{proof}
  This is a consequence of the Hilbert--Mumford criterion and the \'etale local structure of algebraic stacks near a point with a reductive stabilizer (see \cite[\S~1.3 Immediate consequences (5)]{alp.hal.ryd:20}).
  For the convenience of the reader, we include the details.
  By \cite[Theorem~3]{alp:10}, there exists a $T$-invariant locally closed affine $W \subset U$ containing the point $x$ such that the map
  \[ \pi \colon [W/T] \to [U / G]\]
  is affine and \'etale.
  (Note that this result is substantially easier than the main theorem of \cite{alp.hal.ryd:20}.)
  Define $Z$ as the fiber product
  \[
    \begin{tikzcd}
      Z \ar{r}\ar{d}& {[W/T]}\ar{d}{\pi}\\
      \spec \k \ar{r}{[S]}& {[U/G]}.
    \end{tikzcd}
  \]
  Since $\pi$ is representable, \'etale, and of finite type, $Z$ is a finite disjoint of copies of $\spec \k$.
  Let $z_1, \dots, z_n \in [W/T](\k)$ be the points so that $Z = \sqcup s_i$.
  Since the point $x \in [U/G](\k)$ lies in the closure of $s \in [U/G](\k)$ and the map $\pi$ is \'etale (and hence open), the point $x \in [W/T](\k)$ lies in the closure of the set $\{z_1,\dots, z_n\}$.
  But then $x$ lies in the closure of $z_i$ for some $i$.
  In other words, the point $x \in W(\k)$ lies in the closure of the $T$-orbit of a lift $s_i \in W(\k)$ of $z_i \in [W/T](\k)$.
  By the Hilbert--Mumford criterion \cite[Theorem~1.4]{kem:78}, there exists a one-parameter subgroup $\lambda \colon \mathbb G_m \to T$ such that we have the equation
  \begin{equation}\label{eqn:specialization}
    x = \lim_{t \to 0} \lambda (t) \cdot s_i
  \end{equation}
  in $W(\k)$.
  Since $s_i \in [W/T](\k)$ maps to $s \in [U/G](\k)$, the image $s' \in U(\k)$ of $s_i \in W(\k)$ is in the same $G$-orbit of $s$.
  Mapping both sides of equation~\eqref{eqn:specialization} to $U(\k)$ yields the claim.
\end{proof}

\begin{proof}[Proof of \autoref{prop:nolimit}]
  Suppose $X$ lies in the orbit closure of $S$.
  We prove that for any $T_0$-stable hyperplane $H$, the pair $(X, H)$ must be $t$-unstable.
  By applying \autoref{prop:oneparam} to $U = \mathbb P \Sym^3V$ and $G = \PGL V$, we deduce that there exists a one-parameter subgroup $\lambda \colon \mathbb G_m \to T_0$ and a cubic surface $S' \subset \P^3$ such that $S'$ is in the $\PGL_4$-orbit of $S$ and
  \[ X = \lim_{t \to 0} \lambda_t \cdot S'.\]
  Consider any $\lambda$-fixed hyperplane $H \subset \P^3$.
  We have
  \[ \lim_{t \to 0} \lambda_t(S', H) = (X, H).\]
  Since $(S', H)$ is $t$-stable, the limit $(X, H)$ must be $t$-unstable.
  The proof is then complete.
\end{proof}
\begin{corollary}\label{cor:nolimit}
  Let $X$ be as in \autoref{prop:nolimit} and assume $0 < t \leq 3/7$.
  Let $S$ be a smooth cubic surface without an Eckardt point.
  Then $X$ is not in the closure of the $\PGL_4$ orbit of $S$.
\end{corollary}
\begin{proof}
  We use the classification of $t$-stable pairs $(S,H)$ due to Gallardo and Martinez-Garcia \cite{gal.mar:19}.
  For a smooth cubic surface $S$ without an Eckardt point and any hyperplane $H$, the curve $D = S \cap H$ is reduced and has $A_n$ singularities for $n \leq 3$ (at worst tacnodes).
  From \cite[Theorem~2]{gal.mar:19}, it follows that $(S, H)$ is $t$-stable.
  We now use \autoref{prop:nolimit}.
\end{proof}

\begin{corollary}
  The following surfaces are not in the closure of the $\PGL_4$-orbit of a generic cubic surface (the parenthesis describes the singularities):
  \begin{enumerate}
  \item $x_0x_1x_3 = x_2^3$ \quad $(3A_2)$
  \item $x_3(x_0x_2-x_1^2) = x_0x_1^2$ \quad $(A_3 + 2A_1)$
  \item $x_3(x_0x_2-x_1^2) = x_0^2x_1$ \quad $(A_4 + A_1)$
  \item $x_3x_0^2 = x_1^3 + x_2^3$ \quad $(D_4)$
  \end{enumerate}
\end{corollary}
\begin{proof}
  Each surface $X$ in this list has a reductive stabilizer group $T$, and there exists a $T_0$-fixed hyperplane section $H$ such that the pair $(X, H)$ is $t$-semistable for some $0 < t \leq 3/7$.
  We now apply \autoref{cor:nolimit}.

  In the following table, we list the automorphism groups, the hyperplane $H$, and the relevant value of $t$.
  The automorphism groups are from \cite[Theorem~3]{sak:10} and the data of the $t$-semistable hyperplane is from \cite[Table~2]{gal.mar:19}.
  We denote the symmetric group on $n$ letters by $\Sigma_n$.
  \begin{center}
  \begin{tabular}{l l l l}
    \toprule
    $X$ & $\Aut(X)$ & $H$ & $t$\\
    \midrule
    $x_0x_1x_3 = x_2^3$& $\mathbb G_m^2 \rtimes \Sigma_3$ & $x_2 = 0$ & All $0 < t < 1$\\
    $x_3(x_0x_2-x_1^2) = x_0x_1^2$& $\mathbb G_m \rtimes \Sigma_2$ & $x_2 = 0$ & $1/5$ \\
    $x_3(x_0x_2-x_1^2) = x_0^2x_1$& $\mathbb G_m$ & $x_2 = 0$ & $1/3$\\
    $x_3x_0^2 = x_1^3 + x_2^3$& $\mathbb G_m \rtimes \Sigma_3$ & $x_3 = 0$ & $3/7$\\
    \bottomrule
  \end{tabular}
\end{center}
\end{proof}

Let $X$ be a good cubic surface (not in the closure of a general orbit) with stabilizer group $\mathbb G_m$.
We have a good family
\[ \mathcal X = [X / \mathbb G_m] \xrightarrow{\pi} [. / \mathbb G_m] = B\mathbb G_m,\]
yielding a moduli map
\[\mu \colon B \mathbb G_m \to [\mathbb P^{19}/\SL_4].\]
By construction, the pull-back under $\mu$ of the closure of a generic orbit is empty.
Therefore, we must have
\[ a_{1^4} v_1^4 + a_{1^22}v_1^2v_2 + a_{13}v_1v_3 + a_{2^2}v_2^2 + a_4v_4 = 0.\]
By computing $v_1^4, \dots, v_4$, we get a linear relationship among the coefficients.

To compute $v_1^4, \dotso, v_4$, we must identify the vector bundle $\pi_* \left( \omega_{\mathcal X/B\mathbb G_m}^{-1}\right)$.
To do so, let $\mathcal V$ be the rank 4 vector bundle on $B \mathbb G_m$ corresponding to the representation $\mathbb G_m \to \SL_4$.
Then $\mathcal X \subset \P \mathcal V$ is a divisor, given by a global section of a line bundle
\[ \mathcal L = \mathcal O_{\P \mathcal V}(3) \otimes \pi^* M\]
for some line bundle $M$ on $B\mathbb G_m$.
By adjunction, we have
\begin{align*}
  \omega_{\mathcal X/B\mathbb G_m} &= \omega_{\P \mathcal V / \mathbb G_m} \otimes \mathcal L \big|_{\mathcal X}\\
                                   &= \mathcal O_{\P \mathcal V}(-4) \otimes \mathcal O_{\P \mathcal V}(3) \otimes \pi^* M \big|_{\mathcal X} \\
                                   &= \mathcal O_{\P \mathcal V}(-1) \otimes \pi^* M \big |_{\mathcal X},
\end{align*}
and hence
\begin{align*}
  \pi_* \left( \omega_{\mathcal X/B\mathbb G_m}  \right) &= \pi_* \left( \mathcal O_{\P \mathcal V}(1) \big|_{\mathcal X} \right) \otimes M^\vee \\
  &= \mathcal V \otimes M^\vee.
\end{align*}
Hence, by identifying $\mathcal V$ and $M$, we can easily compute the chern classes $v_i$.

Given $a \in \mathbb Z$, we denote by $\chi(a)$ the line bundle on $B\mathbb G_m$ corresponding to the character $t \mapsto t^a$.
Denote by $q \in A^1(B\mathbb G_m)$ the generator of the cohomology ring given by $q = c_1(\chi(1))$.
\subsection{The family $3A_2$}
For
\[ X: x_0x_1x_3 = x_2^3\]
a stabilizing $\mathbb G_m \subset \SL_4$ is given by the diagonal matrix with exponents $(a,b,0,c)$ where $a+b+c = 0$.
Thus, we have
\[\mathcal V = \chi(a) \oplus \chi(b) \oplus \chi(0) \oplus \chi(c).\]
The element $x_0x_1x_3 - x_2^3$ is a global section of $\Sym^3 \mathcal V$.
In other words, $\mathcal X$ is a section of $\mathcal O_{\P \mathcal V}(3)$ and hence we have
\[ M = \chi(0).\]
We get
\[ v_1 = 0, \quad v_2 = (ab+bc+ca) \cdot q^2, \quad v_3 = abc \cdot q^3, \quad v_4 = 0,\]
and hence
\[ v_1^4 = 0, \quad v_1^2v_2 = 0, \quad v_1v_3 = 0,\quad v_2^2 \neq 0, \quad v_4 = 0.\]
Therefore, this family gives the relation
\begin{equation}\label{eqn:iso1}
  a_{2^2} = 0.
\end{equation}

\subsection{The family $A_3 + 2A_1$}
For
\[ X: x_3(x_0x_2-x_1^2) = x_0x_1^2,\]
the stabilizing $\mathbb G_m \subset \SL_4$ is given by the diagonal matrix with exponents $(-3,1,5,-3)$.
Thus, we have
\[ \mathcal V = \chi(-3) \oplus \chi(1) \oplus \chi(5) \oplus \chi(-3).\]
The element $x_3(x_0x_2-x_1^2) - x_0x_1^2$ defines a global section of $\Sym^3 \mathcal V \otimes \chi(1)$.
In other words, we have
\[ M = \chi(1).\]
%((-4 -16 64 0) (256 -256 -256 256 0))
We get
\[ v_1 = -4q, \quad v_2 = -16q^2, \quad v_3 = 64q^3, \quad v_4 = 0,\]
and hence
\[ v_1^4 = 256 \quad v_1^2v_2 = -256, \quad v_1v_3 = -256, \quad v_2^2 = 256, \quad v_4 = 0.\]
Therefore, this family gives the relation
\begin{equation}\label{eqn:iso2}
  a_{1^4} - a_{1^2 2}- a_{13} +  a_{2^2}  = 0.
\end{equation}

\subsection{The family $A_4 + A_1$}
For
\[ X: x_3(x_0x_2-x_1^2) = x_0^2x_1,\]
the stabilizing $\mathbb G_m \subset \SL_4$ is given by the diagonal matrix with exponents $(1,-1,-3,3)$.
Thus, we have
\[ \mathcal V = \chi(1) \oplus \chi(-1) \oplus \chi(-3) \oplus \chi(3).\]
The element $x_3(x_0x_2-x_1^2) - x^2_0x_1$ defines a global section of $\Sym^3 \mathcal V \otimes \chi(-1)$.
In other words, we have
\[ M = \chi(-1).\]
We get
\[ v_1 = 4q, \quad v_2 = -4q^2, \quad v_3 = -16q^3, \quad v_4 = 0,\]
and hence
\[ v_1^4 = 256 \quad v_1^2v_2 = -64, \quad v_1v_3 = -64, \quad v_2^2 = 16, \quad v_4 = 0.\]
Therefore, this family gives the relation
\begin{equation}\label{eqn:iso3}
  16a_{1^4} -4a_{1^2 2}- 4a_{13} + 1 a_{2^2}  = 0.
\end{equation}

\subsection{The family $D_4$}
For
\[ X: x_3x_0^2 = x_1^3+x_2^3,\]
the stabilizing $\mathbb G_m \subset \SL_4$ is given by the diagonal matrix with exponents $(5,1,1,-7)$.
Thus, we have
\[ \mathcal V = \chi(5) \oplus \chi(1) \oplus \chi(1) \oplus \chi(-7).\]
The element $x_3x_0^2-x_1^3-x_2^3$ defines a global section of $\Sym^3 \mathcal V \otimes \chi(-3)$.
In other words, we have
\[ M = \chi(-3).\]
We get
\[ v_1 = -12q, \quad v_2 = -16q^2, \quad v_3 = 192q^3, \quad v_4 = -512q^4,\]
and hence
\[ v_1^4 = 20736 \quad v_1^2v_2 = 2304, \quad v_1v_3 = -2304, \quad v_2^2 = 256, \quad v_4 = -512.\]
Therefore, this family gives the relation
\begin{equation}\label{eqn:iso4}
  81a_{1^4} +9a_{1^2 2}- 9a_{13} +  a_{2^2}-2a_4  = 0.
\end{equation}

\subsection{Families over a base variety}
By choosing a suitable base $B$ and a map $B \to B \mathbb G_m$, we
can construct an isotrivial family over $B$ by pulling back one of the
families above.  We describe this explicitly in an example.

Let $B = \mathbb P^4$.  Set
\[V = \mathcal O(5) \oplus \mathcal O(1) \oplus \mathcal O(1) \oplus
  \mathcal O(-7).\] Let $x_0, x_1, x_2, x_3$ be generators of the four
summands on the standard $\mathbb A^4 \subset \mathbb P^4$.  We have
the section $(x_3x_0^2-x_1^3-x_2^3)$ of
$\Sym^3 V = \mathcal O(3)^{\oplus 20}$.  Note that this section has a
pole of order $3$ along the hyperplane at infinity.  As a result, it
defines a section $\xi$ of $\Sym^3 V \otimes \mathcal O(-3)$, which is
nowhere vanishing.  Equivalently, $\xi$ is a section of
$\mathcal O_{\P V}(3) \otimes \pi^* \mathcal O(-3)$.  Our family
$\mathcal X \to B$ is defined by the zero-locus of $\xi$ in
$\P V \to B$.

\appendix

\section{Existance of a universal formula}

In this section, we recall statements from equivariant intersection
theory that guarantee the existence of the formula \eqref{eq:P}. On
this first pass, we will let $\mathbb{P}(V)$ denote the projective
bundle of 1-dimensional subspaces in the fibers of $V$ and correct for
the sign later if people want to work with duals.

\begin{lemma}
\label{lem:Q_Zclass}
Let $W$ be an $r+1$-dimensional vector space and $Z\subset \operatorname{Sym}^d W^{\vee}$ be a subvariety. Then, the equivariant class $[\mathbb{P}Z]$ inside the equivariant Chow ring 
\begin{align}
    A^{\bullet}(\mathbb{P}_{GL_{r+1}}(\operatorname{Sym}^d W^{\vee}))\cong \frac{\mathbb{Z}[H, t_0,\ldots,t_{r}]^{S^{r+1}}}{\left(\displaystyle\prod_{\substack{i_0+\cdots+i_{r}=d\\ i_0,\ldots,i_r\geq 0}}(H-i_0 t_0 - \cdots - i_r t_r)\right)}\label{eq:equivariantSym}
\end{align}
can be expressed as
\begin{align*}
    Q_Z(t_0-\frac{H}{d},\ldots,t_{r}-\frac{H}{d}),
\end{align*}
for polynomial homogeneous $Q_Z$ with degree equal to the codimension of $Z$. 
\end{lemma}

The variables $t_i$'s in \autoref{lem:Q_Zclass} can either be viewed as chern roots or via the inclusion into the torus-equivariant Chow ring 
\begin{align*}
    A^{\bullet}(\mathbb{P}_{T}(\operatorname{Sym}^d W^{\vee}))\cong \frac{\mathbb{Z}[H, t_0,\ldots,t_{r}]}{\left(\displaystyle\prod_{\substack{i_0+\cdots+i_{r}=d\\ i_0,\ldots,i_r\geq 0}}(H-i_0 t_0 - \cdots - i_r t_r)\right)}
\end{align*}
where $T\subset GL_{r+1}$ is the standard torus.

\begin{proof}[Proof of \autoref{lem:Q_Zclass}]
The formula for the Chow ring of $A^{\bullet}(\mathbb{P}_{GL_{r+1}}(\operatorname{Sym}^d W^{\vee}))$ in \eqref{eq:equivariantSym} follows from the Chow ring of a projective bundle. %applied to the construction of an equivariant Chow ring, didn't find a quick reference for this than other people saying its obvious

The formula for $[\mathbb{P}Z]$ in terms of a polynomial $Q_Z$ follows from \cite[Theorem 6.1]{FNR05}, where $Q_Z$ is the formula for the class of $[Z]$ in $A^{\bullet}_{GL_r+1}(\operatorname{Sym}^d W^{\vee})\cong \mathbb{Z}[H, t_0,\ldots,t_{r}]$. 
\end{proof}

\begin{proposition}
In the context of \autoref{def:goodfamily} a universal polynomial $P$ as in \eqref{eq:P} exists.
\end{proposition}

\begin{proof}
Following the setup in \eqref{eq:P}, we start with a good family $\pi: X\to B$ and let $V=\pi_{*}\omega_{X/B}^{-1}$. Since each fiber of $X\to B$ is embedded into $\mathbb{P}(V)$ as a cubic surface via the anticanonical map, we get a section $\sigma$ of the bundle $\mathbb{P}(\operatorname{Sym}^3 V^{\vee})\to B$. The section $\sigma$ maps each point $b$ to the equation of the cubic surface $X_b$ lying above it.

Inside of $\mathbb{P}(\operatorname{Sym}^3 V^{\vee})$, the section $\sigma$ is induced by the inclusion of a line bundle $L=\det(V^{\vee})\hookrightarrow \operatorname{Sym}^3 V^{\vee}$. {\color{red} Check this, the main thing we need is that the line bundle can be expressed in terms of $c_1(V)$ only. I don't have a clean proof yet.}

Now, we apply \autoref{lem:Q_Zclass}, where $Z\subset \operatorname{Sym}^3(\mathbb{C}^{4\ast})$ is the $GL_4$ orbit closure of a general degree 3 homogenous equation in 4 variables. The vector bundle $V\to B$ induces a map $B\to [\operatorname{pt}/GL_4]$ under which $[\operatorname{Sym}^3(\mathbb{C}^{4\ast})/GL_4]$ pulls back to $\mathbb{P}(\operatorname{Sym}^3 V^{\vee})$. Under the pullback the equivariant class of the projectivized orbit closure $[\mathbb{P}Z]$ pulls back to the class of the relative orbit closure $\mathcal{Z}\subset \mathbb{P}(\operatorname{Sym}^3 V^{\vee})$, where each fiber of $\mathcal{Z}\to B$ is given by $Z$. By \autoref{lem:Q_Zclass}, the class of $\mathcal{Z}$ depends only on the chern classes of $V$. 

The polynomial $P$ in \eqref{eq:P} is the intersection number of $\sigma$ with the class of $\mathcal{Z}$. The class of $\sigma$, the Chow ring of $\mathbb{P}(\operatorname{Sym}^3 V^{\vee})$, and the class of $\mathcal{Z}$ are depend only on the chern classes of $V$. Thus, the final intersection number depends only on the chern classes of $V$ as desired. 
{\color{red} I don't think it'll be any harder to actually determine the formula of $P$ in \eqref{eq:P} in terms of the equivariant class of an orbit closure of a cubic surface, but we don't seem to need the formula. The point is that $P$ and the equivariant class of the orbit closure are equivalent by some substitution.}
\end{proof}

\section{Congruence relations from hypersurfaces with finite automorphisms}
As another another check for the class, I think we can get congruence relations from isotrivial families of hypersurfaces with finite linear automorphisms. This is the same idea as the $3A_2$ family, except the condition holds in $H_{G}(\operatorname{pt})$ with $G$ a finite group instead of $\mathbb{G}_m$. 

\begin{proposition}
\label{prop:isotrivial2}
Let $X\subset \mathbb{P}^r$ be a hypersurface with a linear automorphism group $G\subset \operatorname{Aut}(X)\subset GL_{r+1}$. Let $Q$ be the polynomial $Q_Z$ in \autoref{lem:Q_Zclass} where $Z$ \emph{any} equivariant cycle in the space of hypersurfaces not containing $X$. Then, under the linear representation $G\hookrightarrow GL_{r+1}$, $Q$ is sent to zero under the characteristic class map:
\begin{align*}
    A^{\bullet}_{GL_{r+1}}\cong \mathbb{Z}[t_0,\ldots,t_{r}]\to A^{\bullet}_{G}(\operatorname{pt}). 
\end{align*}
\end{proposition}

% \begin{proof}
% Continue: the key point is that the twisted family of hypersurfaces comes from descending a trivial section of Sym^d of the universal vector bundle pulled back from BG\to BGL_{r+1}, so intersecting this section with the orbit closure picks out constant term of the projective equivariant class. This constant term is exactly $Q$ in the proposition.

% This is just the general construction of the twisted family:
% Given a hypersurface $X\subset \mathbb{P}^r$ with a finite linear automorphism group $G=\operatorname{Aut}(X)\subset GL_{r+1}$, we want to construct a twisted version of $X$ over $BG$. To this end, take $X\times ((\mathbb{C}^{r+1})^N\backslash Z)\subset \mathbb{P}^r \times ((\mathbb{C}^{r+1})^N\backslash Z)$, where $N>>0$ and $Z$ is some closed set of high codimension. Here $G$ acts on $X$ and $\mathbb{P}^{r+1}$ by linear automorphisms and on each factor of $\mathbb{C}^{r+1}$ via $GL_{r+1}$. The closed set $Z$ is chosen to $G$ acts freely on $(\mathbb{C}^{r+1})^N\backslash Z$. 

% Taking the quotient by $G$ yields a nontrivial family of hypersurfaces $\mathcal{X}\to ((\mathbb{C}^{r+1})^N\backslash Z)/G$ inside the projectivization of some vector bundle $W$ over $((\mathbb{C}^{r+1})^N\backslash Z)/G$.
% \end{proof}
\begin{example}
\autoref{prop:isotrivial2} specializes to \autoref{sec:isotrivial3A2} when applied to the cubic surface $X^3=YZW$.
\end{example}

\begin{example}
Every coefficient of every class of a quartic plane curve we computed in \cite{LPT19} is divisible by 4 (even after dividing by the automorphism groups). I think this is a consequence of the Fermat quartic having automorphism group containing $(\mu_4)^3\hookrightarrow GL_3$. Similarly, every equivariant class of a cubic curve orbit is divisible by 3 in \cite[Appendix C]{LPT19}, and this should follow from the Fermat cubic. We will try to prove this in \autoref{prop:divd} below.
\end{example}

Given the example, we will try to flesh out the application of \autoref{prop:3A2} to the Fermat hypersurface applied to the case $G=(\mu_d)^{r+1}\subset GL_{r+1}$. In general, I don't know how to compute the map $A^{\bullet}_{GL_{r+1}}\to A^{\bullet}_{G}(\operatorname{pt})$ (google ``characteristic class of a linear representation''). 

\subsection{Fermat hypersurfaces}
A Fermat hypersurface $X\subset \mathbb{P}^r$ of degree $d$ has a linear automorphism group containing $(\mu_d)^{r+1}$. {\color{red} You can also permute the coordinates, curious what condition in $A^{\bullet}_{S_{r+1}}$ that gives...}

\begin{lemma}
\label{lem:mu_d}
Under the inclusion $\mu_d\hookrightarrow \mathbb{G}_m$, the map in group cohomology is given by 
\begin{align*}
    \mathbb{Z}[t]=A^{\bullet}_{\mathbb{G}_m}(\operatorname{pt})\to A^{\bullet}_{\mu_d}(\operatorname{pt})=\mathbb{Z}[t]/(dt)
\end{align*}
\end{lemma}

\begin{proof}
The ring structure of $A^{\bullet}_{\mu_d}(\operatorname{pt})$ is given by the cohomology ring of the Lens space 
\href{https://mathoverflow.net/questions/133974/reference-for-ring-structure-on-group-cohomology}{(mathoverflow link)}. To compute the map, we take an approximation of $B\mathbb{G}_m$ and $B\mu_d$ to compute the map in $A^{1}_{\mathbb{G}_m}(\operatorname{pt})\to A^{1}_{\mu_d}(\operatorname{pt})$. This reduces to a familiar geometric example of a cone over a rational normal curve.

We approximate using the map $B\mu_d\to B\mathbb{G}_m$ using
\begin{align*}
    \pi: S=\operatorname{Spec}(\mathbb{C}[x^d, x^{d-1}y,\ldots,y^d])=(\mathbb{A}^2\backslash\{0\})/\mu_d\to (\mathbb{A}^2\backslash\{0\})/\mathbb{G}_m=\mathbb{P}^1. 
\end{align*}
The map $\pi$ is the projection of the cone over a degree $d$ rational normal curve (minus the origin) onto the rational normal curve. The content of \autoref{lem:mu_d} is that a ruling of the cone $S$ over a rational normal curve generates the class group of the cone. 

This is true because $S\backslash L\cong \mathbb{A}^2$ where $L$ is a ruling. (The more nontrivial fact is that $A^{1}(S)\cong \mathbb{Z}/d\mathbb{Z}$, so $[L]$ is exactly $d$-torsion. This is implied by the cohomology of the lens space above, but one can probably prove it directly. For example, $[L]$ is nontrivial in the class group because it cannot be cut out by a single equation (since the tangent space at the cone is at least 3-dimensional for $d>1$).) 
\end{proof}

\begin{proposition}
\label{prop:divd}
If $Q$ is the polynomial associated to the equivariant class of \emph{any} equivariant class in the space of degree $d$ hypersurfaces not containing the Fermat hypersurface, then every coefficient of $Q$ is divisible by $d$.
\end{proposition}

\begin{proof}
This follows from \autoref{lem:mu_d} and \autoref{prop:isotrivial2}.
\end{proof}

\begin{example}
For a general cubic surface, our current guess is that:
$$Q = 25920 c_1^4+12960 c_1^2 c_2-6480 c_1 c_3+9720 c_4$$
 and
$$P = 1080 c_1^2 c_2-1080 c_1 c_3+9720 c_4,$$
where the $P$ and $Q$ are related by
\begin{align*}
P(t_0,t_1,t_2,t_3) &= Q(t_0 - \frac{t_0+\cdots+t_3}{3},\ldots, t_3 - \frac{t_0+\cdots+t_3}{3})\\
P(t_0 - (t_0+\cdots+t_3), \ldots, t_3 - (t_0+\cdots+t_3)) &= Q(t_0,\ldots,t_3)
\end{align*}
The coefficients of $Q$  are all divisible by 3 as predicted.
\end{example}
\bibliographystyle{plain}
\bibliography{references.bib}
\end{document}

%%% Local Variables:
%%% mode: latex
%%% TeX-master: t
%%% End:

